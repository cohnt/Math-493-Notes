\documentclass[10pt,letterpaper]{article}
\usepackage[utf8]{inputenc}
\usepackage[intlimits]{amsmath}
\usepackage{amsfonts}
\usepackage{amssymb}
\usepackage{ragged2e}
\usepackage[letterpaper, margin=1in]{geometry}
\usepackage{graphicx}
\usepackage{cancel}
\usepackage{mathtools}
\usepackage{tabularx}
\usepackage{arydshln}
\usepackage{tensor}
\usepackage{array}
\usepackage{xcolor}
\usepackage[boxed]{algorithm}
\usepackage[noend]{algpseudocode}
\usepackage{listings}
\usepackage{textcomp}
\usepackage[pdf,tmpdir,singlefile]{graphviz}
\usepackage{mathrsfs}
\usepackage{bbm}
\usepackage{tikz}
\usepackage{enumitem}
\usepackage{arydshln}
\usepackage{relsize}
\usepackage{multicol}
\usepackage{scalerel}

\usetikzlibrary{bayesnet}

%%%%%%%%%%%%%%%%%%%%%%%%%%%%%
% Formatting commands
%%%%%%%%%%%%%%%%%%%%%%%%%%%%%
\newcommand{\n}{\hfill\break}
\newcommand{\up}{\vspace{-\baselineskip}}
\newcommand{\hangblock}[2]{\par\noindent\settowidth{\hangindent}{\textbf{#1: }}\textbf{#1: }\!\!\!#2}
\newcommand{\lemma}[1]{\hangblock{Lemma}{#1}}
\newcommand{\defn}[1]{\hangblock{Defn}{#1}}
\newcommand{\thm}[1]{\hangblock{Thm}{#1}}
\newcommand{\cor}[1]{\hangblock{Cor}{#1}}
\newcommand{\prop}[1]{\hangblock{Prop}{#1}}
\newcommand{\ex}[1]{\hangblock{Ex}{#1}}
\newcommand{\exer}[1]{\hangblock{Exer}{#1}}
\newcommand{\fact}[1]{\hangblock{Fact}{#1}}
\newcommand{\remark}[1]{\hangblock{Remark}{#1}}
\newcommand{\proven}{\;$\square$\n}
\newcommand{\problem}[1]{\par\noindent{#1}\n}
\newcommand{\problempart}[2]{\par\noindent\indent{}\settowidth{\hangindent}{\textbf{(#1)} \indent{}}\textbf{(#1)} #2\n}
\newcommand{\ptxt}[1]{\textrm{\textnormal{#1}}}
\newcommand{\inlineeq}[1]{\centerline{$\displaystyle #1$}}
\newcommand{\pageline}{\noindent\rule{\textwidth}{0.1pt}}

%%%%%%%%%%%%%%%%%%%%%%%%%%%%%
% Math commands
%%%%%%%%%%%%%%%%%%%%%%%%%%%%%
% Set Theory
\newcommand{\card}[1]{\left|#1\right|}
\newcommand{\set}[1]{\left\{#1\right\}}
\newcommand{\setmid}{\;\middle|\;}
\newcommand{\ps}[1]{\mathcal{P}\left(#1\right)}
\newcommand{\pfinite}[1]{\mathcal{P}^{\ptxt{finite}}\left(#1\right)}
\newcommand{\naturals}{\mathbb{N}}
\newcommand{\N}{\naturals}
\newcommand{\integers}{\mathbb{Z}}
\newcommand{\Z}{\integers}
\newcommand{\rationals}{\mathbb{Q}}
\newcommand{\Q}{\rationals}
\newcommand{\reals}{\mathbb{R}}
\newcommand{\R}{\reals}
\newcommand{\complex}{\mathbb{C}}
\newcommand{\C}{\complex}
\newcommand{\halfPlane}{\mathbb{H}}
\let\H\relax
\newcommand{\H}{\halfPlane}
\newcommand{\comp}{^{\complement}}
\DeclareMathOperator{\Hom}{Hom}
\newcommand{\Ind}{\mathbbm{1}}
\newcommand{\cut}{\setminus}
\DeclareMathOperator{\elem}{elem}

% Graph Theory
\let\deg\relax
\DeclareMathOperator{\deg}{deg}
\newcommand{\degp}{\ptxt{deg}^{+}}
\newcommand{\degn}{\ptxt{deg}^{-}}
\newcommand{\precdot}{\mathrel{\ooalign{$\prec$\cr\hidewidth\hbox{$\cdot\mkern0.5mu$}\cr}}}
\newcommand{\succdot}{\mathrel{\ooalign{$\cdot\mkern0.5mu$\cr\hidewidth\hbox{$\succ$}\cr\phantom{$\succ$}}}}
\DeclareMathOperator{\cl}{cl}
\DeclareMathOperator{\affdim}{affdim}

% Probability
\newcommand{\parSymbol}{\P}
\newcommand{\Prob}{\mathbb{P}}
\renewcommand{\P}{\Prob}
\newcommand{\Avg}{\mathbb{E}}
\newcommand{\E}{\Avg}
\DeclareMathOperator{\Var}{Var}
\DeclareMathOperator{\cov}{cov}
\DeclareMathOperator{\Unif}{Unif}
\DeclareMathOperator{\Binom}{Binom}
\newcommand{\CI}{\mathrel{\text{\scalebox{1.07}{$\perp\mkern-10mu\perp$}}}}

% Standard Math
\newcommand{\inv}{^{-1}}
\newcommand{\abs}[1]{\left|#1\right|}
\newcommand{\ceil}[1]{\left\lceil{}#1\right\rceil{}}
\newcommand{\floor}[1]{\left\lfloor{}#1\right\rfloor{}}
\newcommand{\conj}[1]{\overline{#1}}
\newcommand{\of}{\circ}
\newcommand{\tri}{\triangle}
\newcommand{\inj}{\hookrightarrow}
\newcommand{\surj}{\twoheadrightarrow}
\newcommand{\ndiv}{\nmid}
\renewcommand{\epsilon}{\varepsilon}
\newcommand{\divides}{\mid}
\newcommand{\ndivides}{\nmid}
\DeclareMathOperator{\lcm}{lcm}
\DeclareMathOperator{\sgn}{sgn}
\newcommand{\map}[4]{\!\!\!\begin{array}[t]{rcl}#1 & \!\!\!\!\to & \!\!\!\!#2\\ #3 & \!\!\!\!\mapsto & \!\!\!\!#4\end{array}}
\newcommand{\bigsum}[2]{\smashoperator[lr]{\sum_{\scalebox{#1}{$#2$}}}}

% Linear Algebra
\newcommand{\Id}{\textrm{\textnormal{Id}}}
\newcommand{\im}{\textrm{\textnormal{im}}}
\newcommand{\norm}[1]{\abs{\abs{#1}}}
\newcommand{\tpose}{^{T}}
\newcommand{\iprod}[1]{\left<#1\right>}
\DeclareMathOperator{\trace}{tr}
\newcommand{\chgBasMat}[3]{\!\!\tensor*[_{#1}]{\left[#2\right]}{_{#3}}}
\newcommand{\vecBas}[2]{\tensor*[]{\left[#1\right]}{_{#2}}}
\DeclareMathOperator{\GL}{GL}
\DeclareMathOperator{\Mat}{Mat}
\DeclareMathOperator{\vspan}{span}
\DeclareMathOperator{\rank}{rank}
\newcommand{\V}[1]{\vec{#1}}
\DeclareMathOperator{\proj}{proj}
\DeclareMathOperator{\compProj}{comp}
\DeclareMathOperator{\row}{row}
\newcommand{\smallPMatrix}[1]{\paren{\begin{smallmatrix}#1\end{smallmatrix}}}
\newcommand{\smallBMatrix}[1]{\brack{\begin{smallmatrix}#1\end{smallmatrix}}}

% Multilinear Algebra
\newcommand{\Lsym}{\L}
\let\L\relax
\DeclareMathOperator{\L}{\mathscr{L}}
\DeclareMathOperator{\A}{\mathcal{A}}
\DeclareMathOperator{\Alt}{Alt}
\DeclareMathOperator{\Sym}{Sym}
\newcommand{\ot}{\otimes}
\newcommand{\ox}{\otimes}
\DeclareMathOperator{\asc}{asc}
\DeclareMathOperator{\asSet}{set}
\DeclareMathOperator{\sort}{sort}
\DeclareMathOperator{\ringA}{\mathring{A}}

% Topology
\newcommand{\closure}[1]{\overline{#1}}
\newcommand{\uball}{\mathcal{U}}
\DeclareMathOperator{\Int}{Int}
\DeclareMathOperator{\Ext}{Ext}
\DeclareMathOperator{\Bd}{Bd}
\DeclareMathOperator{\rInt}{rInt}
\DeclareMathOperator{\ch}{ch}
\DeclareMathOperator{\ah}{ah}
\newcommand{\LargerTau}{\mathlarger{\mathlarger{\mathlarger{\mathlarger{\tau}}}}}
\newcommand{\Tau}{\mathcal{T}}

% Analysis
\DeclareMathOperator{\Graph}{Graph}
\DeclareMathOperator{\epi}{epi}
\DeclareMathOperator{\hypo}{hypo}
\DeclareMathOperator{\supp}{supp}
\newcommand{\lint}[2]{\underset{#1}{\overset{#2}{{\color{black}\underline{{\color{white}\overline{{\color{black}\int}}\color{black}}}}}}}
\newcommand{\uint}[2]{\underset{#1}{\overset{#2}{{\color{white}\underline{{\color{black}\overline{{\color{black}\int}}\color{black}}}}}}}
\newcommand{\alignint}[2]{\underset{#1}{\overset{#2}{{\color{white}\underline{{\color{white}\overline{{\color{black}\int}}\color{black}}}}}}}
\newcommand{\extint}{\ptxt{ext}\int}
\newcommand{\extalignint}[2]{\ptxt{ext}\alignint{#1}{#2}}
\newcommand{\conv}{\ast}
\newcommand{\pd}[2]{\frac{\partial{}#1}{\partial{}#2}}
\newcommand{\del}{\nabla}
\DeclareMathOperator{\grad}{grad}
\DeclareMathOperator{\curl}{curl}
\let\div\relax
\DeclareMathOperator{\div}{div}
\DeclareMathOperator{\vol}{vol}

% Complex Analysis
\let\Re\relax
\DeclareMathOperator{\Re}{Re}
\let\Im\relax
\DeclareMathOperator{\Im}{Im}
\DeclareMathOperator{\Res}{Res}

% Abstract Algebra
\DeclareMathOperator{\ord}{ord}
\newcommand{\generated}[1]{\left<#1\right>}
\newcommand{\cycle}[1]{\begin{pmatrix}#1\end{pmatrix}}
\newcommand{\subscriptcycle}[1]{\scaleto{\cycle{#1}}{5pt}}
\newcommand{\id}{\ptxt{id}}
\newcommand{\iso}{\cong}
\DeclareMathOperator{\Aut}{Aut}
\DeclareMathOperator{\SL}{SL}

% Convex Optimization
\newcommand{\sectionSymbol}{\S}
\let\S\relax
\newcommand{\S}{\mathbb{S}}

% Proofs
\newcommand{\st}{s.t.}
\newcommand{\unique}{!}
\newcommand{\iffdef}{\overset{\ptxt{def}}{\Leftrightarrow}}
\newcommand{\eqdef}{\overset{\ptxt{def}}{=}}
\newcommand{\eqVertical}{\rotatebox[origin=c]{90}{=}}
\newcommand{\mapsfrom}{\mathrel{\reflectbox{\ensuremath{\mapsto}}}}
\newcommand{\mapsdown}{\rotatebox[origin=c]{-90}{$\mapsto$}\mkern2mu}
\newcommand{\mapsup}{\rotatebox[origin=c]{90}{$\mapsto$}\mkern2mu}
\newcommand{\from}{\!\mathrel{\reflectbox{\ensuremath{\to}}}}

% Brackets
\newcommand{\paren}[1]{\left(#1\right)}
\renewcommand{\brack}[1]{\left[#1\right]}
\renewcommand{\brace}[1]{\left\{#1\right\}}
\newcommand{\ang}[1]{\left<#1\right>}

% Algorithms
\algrenewcommand{\algorithmiccomment}[1]{\hskip 1em \texttt{// #1}}
\algrenewcommand\algorithmicrequire{\textbf{Input:}}
\algrenewcommand\algorithmicensure{\textbf{Output:}}
\newcommand{\algP}{\ptxt{\textbf{P}}}
\newcommand{\algNP}{\ptxt{\textbf{NP}}}
\newcommand{\algNPC}{\ptxt{\textbf{NP-Complete}}}
\newcommand{\algNPH}{\ptxt{\textbf{NP-Hard}}}
\newcommand{\algEXP}{\ptxt{\textbf{EXP}}}

%%%%%%%%%%%%%%%%%%%%%%%%%%%%%
% Other commands
%%%%%%%%%%%%%%%%%%%%%%%%%%%%%
\newcommand{\flag}[1]{\textbf{\textcolor{red}{#1}}}
\newcommand{\uSym}{\u}
\let\u\relax
\newcommand{\u}[1]{\underline{#1}}
\newcommand{\bSym}{\b}
\let\b\relax
\newcommand{\b}[1]{\textbf{#1}}
\newcommand{\iSym}{\i}
\let\i\relax
\newcommand{\i}[1]{\textit{#1}}

%%%%%%%%%%%%%%%%%%%%%%%%%%%%%%%%%%%%%%%
% Make l's curvy in math environments %
%%%%%%%%%%%%%%%%%%%%%%%%%%%%%%%%%%%%%%%
\mathcode`l="8000
\begingroup
\makeatletter
\lccode`\~=`\l
\DeclareMathSymbol{\lsb@l}{\mathalpha}{letters}{`l}
\lowercase{\gdef~{\ifnum\the\mathgroup=\m@ne \ell \else \lsb@l \fi}}%
\endgroup

%%%%%%%%%%%%%%%%%%%%%%%%%
% Fix \vdots and \ddots %
%%%%%%%%%%%%%%%%%%%%%%%%%
\usepackage{letltxmacro}
\LetLtxMacro\orgvdots\vdots
\LetLtxMacro\orgddots\ddots

\makeatletter
\DeclareRobustCommand\vdots{%
	\mathpalette\@vdots{}%
}
\newcommand*{\@vdots}[2]{%
	% #1: math style
	% #2: unused
	\sbox0{$#1\cdotp\cdotp\cdotp\m@th$}%
	\sbox2{$#1.\m@th$}%
	\vbox{%
		\dimen@=\wd0 %
		\advance\dimen@ -3\ht2 %
		\kern.5\dimen@
		% remove side bearings
		\dimen@=\wd2 %
		\advance\dimen@ -\ht2 %
		\dimen2=\wd0 %
		\advance\dimen2 -\dimen@
		\vbox to \dimen2{%
			\offinterlineskip
			\copy2 \vfill\copy2 \vfill\copy2 %
		}%
	}%
}
\DeclareRobustCommand\ddots{%
	\mathinner{%
		\mathpalette\@ddots{}%
		\mkern\thinmuskip
	}%
}
\newcommand*{\@ddots}[2]{%
	% #1: math style
	% #2: unused
	\sbox0{$#1\cdotp\cdotp\cdotp\m@th$}%
	\sbox2{$#1.\m@th$}%
	\vbox{%
		\dimen@=\wd0 %
		\advance\dimen@ -3\ht2 %
		\kern.5\dimen@
		% remove side bearings
		\dimen@=\wd2 %
		\advance\dimen@ -\ht2 %
		\dimen2=\wd0 %
		\advance\dimen2 -\dimen@
		\vbox to \dimen2{%
			\offinterlineskip
			\hbox{$#1\mathpunct{.}\m@th$}%
			\vfill
			\hbox{$#1\mathpunct{\kern\wd2}\mathpunct{.}\m@th$}%
			\vfill
			\hbox{$#1\mathpunct{\kern\wd2}\mathpunct{\kern\wd2}\mathpunct{.}\m@th$}%
		}%
	}%
}
\makeatother

\newcommand{\B}{
	\begin{tikzpicture}
	\filldraw [fill=red, draw=black] (0, 0) rectangle (0.37, 0.45);
	\draw [line width=0.5mm, white ] (0.1,0.08) -- (0.1,0.38);
	\draw[line width=0.5mm, white ] (0.1, 0.35) .. controls (0.2, 0.35) and (0.4, 0.2625) .. (0.1, 0.225);
	\draw[line width=0.5mm, white ] (0.1, 0.225) .. controls (0.2, 0.225) and (0.4, 0.1625) .. (0.1, 0.1);
	\end{tikzpicture}
}

\author{Professor Andrew Snowden\\ \small\i{Transcribed by Thomas Cohn}}
\title{Math 493 Lecture 3}
\date{9/11/2019} % Can also use \today

\begin{document}
\maketitle
\setlength\RaggedRightParindent{\parindent}
\RaggedRight

\ex{
	Consider $\det:\GL_{n}(\R)\to\R^{\times}$ ($\R^{\times}$ is the nonzero real numbers under multiplication).\n
	$\det(AB)=\det(A)\det(B)$, so $\det$ is a group homomorphism.\n
	$\ker(\det)=\set{A\in\GL_{n}(\R)|\det(A)=1}=\SL_{n}(\R)$, so $\SL_{n}(\R)$ is a normal subgroup of $\GL_{n}(\R)$.\n
}

\ex{
	Given $\sigma\in{}S_{n}$, define a linear map $\map{A_{\sigma}:\R^{n}}{\R^{n}}{e_{i}}{e_{\sigma_{i}}}$.\n
	$A_{\sigma}\in\GL_{n}(\R)$ -- we can check that $A_{\sigma}A_{\tau}=A_{\sigma\tau}$.\n
	So we have a group homomorphism $\map{A:S_{n}}{\GL_{n}(\R)}{\sigma}{A_{\sigma}}$.\n
	This is clearly injective, so $A$ is an isomorphism between $S_{n}$ and its image $A(S_{n})\subseteq\GL_{n}(\R)$.
}

\defn{
	Matrices of the form $A_{\sigma}$ for some $\sigma\in{}S_{n}$ are called \textbf{permutation matrices}.
}

\defn{
	$\map{\sgn:S_{n}}{\set{\pm{}1}}{\sigma}{\det(A_{\sigma})}$.
}

\ex{
	$S_{2}=\set{1,\cycle{1 & 2}}$.\n
	$A_{1}=\smallBMatrix{1 & 0\\ 0 & 1}\quad\det{}A_{1}=1\quad\sgn(1)=1$.\n
	$A_{\scaleto{\cycle{1 & 2}}{5pt}}=\smallBMatrix{0 & 1\\ 1 & 0}\quad\det{}A_{\scaleto{\cycle{1 & 2}}{5pt}}=-1\quad\sgn(\cycle{1 & 2})=-1$.\n
}

\ex{
	$S_{3}=\set{1, \cycle{1 & 2}, \cycle{1 & 3}, \cycle{2 & 3}, \cycle{1 & 2 & 3}, \cycle{1 & 3 & 2}}$.\n
	$A_{1}=\smallBMatrix{1 & 0 & 0\\ 0 & 1 & 0\\ 0 & 0 & 1}\quad\det{}A_{1}=1\quad\sgn(1)=1$.\n
	$A_{\subscriptcycle{1 & 2}}=\smallBMatrix{0 & 1 & 0\\ 1 & 0 & 0\\ 0 & 0 & 1}\quad\det{}A_{\subscriptcycle{1 & 2}}=-1\quad\sgn(\cycle{1 & 2})=-1$.\n
	$A_{\subscriptcycle{1 & 3}}=\smallBMatrix{0 & 0 & 1\\ 0 & 1 & 0\\ 1 & 0 & 0}\quad\det{}A_{\subscriptcycle{1 & 3}}=-1\quad\sgn(\cycle{1 & 3})=-1$.\n
	$A_{\subscriptcycle{2 & 3}}=\smallBMatrix{1 & 0 & 0\\ 0 & 0 & 1\\ 0 & 1 & 0}\quad\det{}A_{\subscriptcycle{2 & 3}}=-1\quad\sgn(\cycle{2 & 3})=-1$.\n
	$A_{\subscriptcycle{1 & 2 & 3}}=\smallBMatrix{0 & 0 & 1\\ 1 & 0 & 0\\ 0 & 1 & 0}\quad\det{}A_{\subscriptcycle{1 & 2 & 3}}=1\quad\sgn(\cycle{1 & 2 & 3})=1$.\n
	$A_{\subscriptcycle{1 & 3 & 2}}=\smallBMatrix{0 & 1 & 0\\ 0 & 0 & 1\\ 1 & 0 & 0}\quad\det{}A_{\subscriptcycle{1 & 3 & 2}}=1\quad\sgn(\cycle{1 & 3 & 2})=1$.\n
}

\fact{
	Transpositions generate $S_{n}$.\n
}

\fact{
	For any $n$ and any transposition $\sigma\in{}S_{n}$, $\sgn(\sigma)=1$.\n
	So if $\sigma\in{}S_{n}$, write $\sigma=\tau_{1}\cdots\tau_{m}$, where each $\tau_{i}$ is a transposition.\n
	Then $\sgn(\sigma)=\sgn(\tau_{1})\cdots\sgn(\tau_{m})$.\n
}

\defn{
	$A_{n}=\ker(\sgn:S_{n}\to\set{\pm{}1})$. $A_{n}$ is called the \textbf{alternating group}, and is a normal subgroup of $S_{n}$.\n
}

\ex{
	$A_{2}=\set{1}$.\n
	$A_{3}=\set{1,\cycle{1 & 2 & 3},\cycle{1 & 3 & 2}}$.\n
	$\#A_{n}=\frac{1}{2}n!$ for $n\ge{}2$.\n
}

\defn{
	Let $S$ be a set. An \textbf{equivalence relation} on $S$ is a binary relation $\sim$ \st{}
	\begin{enumerate}
		\item Reflexivity: $\forall{}x\in{}S$, $x\sim{}x$.
		\item Symmetry: $\forall{}x,y\in{}S$, $x\sim{}y\Leftrightarrow{}y\sim{}x$.
		\item Transitivity: $\forall{}x,y,z\in{}S$, $x\sim{}y\land{}y\sim{}z\Rightarrow{}x\sim{}z$.
	\end{enumerate}
}

\ex{
	\begin{enumerate}
		\item Define $x\sim{}y$ iff $x=y$.
		\item Define $x\sim{}y$, $\forall{}x,y$.
		\item Let $f:S\to{}T$ is a function. Define $x\sim{}y$ iff $f(x)=f(y)$.
		\item Define an equivalence relation on $\Z$ by $n\sim{}m$ iff $z\divides{}n-m$, ie., $n\equiv{}m\pmod{2}$.\n
		Note: if we define $f:\Z\to\set{\ptxt{even}, \ptxt{odd}}$, where $f(n)=\ptxt{even}$ if $n$ is even and $f(n)=\ptxt{odd}$ if $n$ is odd, then $f$ induces the above equivalence relation.
	\end{enumerate}
}

\defn{
	Let $S$ be a set with an equivalence relation $\sim$. Let $x\in{}S$.\n
	The \textbf{equivalence class} of $x$ is $C_{x}=\set{y\in{}S|x\sim{}y}$.
}

\ex{
	$S=\Z$, $n\sim{}m$ iff $n\equiv{}m\pmod{2}$.\n
	$C_{1}=\set{\ldots,-3,-1,1,3,\ldots}$\n
	$C_{2}=\set{\ldots,-2,0,2,\ldots}$\n
	$C_{3}=\set{\ldots,-3,-1,1,3,\ldots}$\n
	$C_{4}=\set{\ldots,-2,0,2,\ldots}$\n
}

\prop{
	If two equivalence classes have any common element, they're equal.\n
	Proof: Suppose $z\in{}C_{x}\cap{}C_{y}$. Let $w\in{}C_{x}$. Then $w\sim{}x\sim{}z\sim{}y$. So $w\sim{}y$, so $w\in{}C_{y}$. Thus, $C_{x}\subseteq{}C_{y}$. A similar argument gives us $C_{y}\subseteq{}C_{x}$, so $C_{x}=C_{y}$.\proven
}

\defn{
	Let $S$ be a set. A \textbf{partition} of $S$ is a collection $\mathcal{P}$ of non-empty subsets of $S$ \st{} every element of $S$ belongs to a unique member of $\mathcal{P}$.\n
}

\par\noindent
From the previous proposition, we know that a collection of equivalence classes form a partition.\n
We can reverse this: suppose $\mathcal{P}$ is a partition. Define an equivalence relation on $S$ by $x\sim{}y$ if $x$ and $y$ are in the same element of $\mathcal{P}$.\n

\defn{
	Let $S$ be a set with an equivalence relation. Define $\overline{S}$ to be the set of equivalence classes. For $x\in{}S$, we'll write $\overline{x}=C_{x}\in\overline{S}$.
}

\par\noindent
$\overline{x}=\overline{y}\Leftrightarrow{}x\sim{}y$.\n
So we can define a function $\map{\pi:S}{\overline{S}}{x}{\overline{x}}$. $x\sim{}y$ iff $\pi(x)=\pi(y)$, so $\sim$ is induced by $\pi$.\n

\ex{
	$S=\Z$, with $n\sim{}m$ iff $n\equiv{}m\pmod{2}$. Then $\overline{S}=\set{\overline{0},\overline{1}}$.\n
}

\ex{
	Let $G$ be a group, $H\subset{}G$ a subgroup. Define an equivalence relation on $G$ by $g\equiv{}g'\pmod{H}$ if $g=g'h$ for some $h\in{}H$ (so $(g')\inv{}g\in{}H$).\n
	Check:
	\begin{enumerate}
		\item Reflexivity: $g=g\cdot{}1$, and $1\in{}H$, So $g\equiv{}g\pmod{H}$.
		\item Symmetry: if $g\equiv{}g'\pmod{H}$, then $g=g'h$ for some $h\in{}H$. So $g'=gh\inv$.\n Since $h\inv\in{}H$, $g'\equiv{}g\pmod{H}$.
		\item Transitivity: if $g\equiv{}g'\pmod{H}$ and $g'\equiv{}g''\pmod{H}$, then $g=gh'$ and $g'=g''h'$, for some $h,h'\in{}H$. So $g=(g''h')h=g''(h'h)$. $h'h\in{}H$, so $g\equiv{}g''\pmod{H}$.
	\end{enumerate}
}

\ex{
	$G=\Z$, $H=d\Z$ ($d>0$). $n,m\in\Z$, $n\equiv{}m\pmod{H}$, according to this definition, iff\n $n-m\in{}H\Leftrightarrow{}n\equiv{}m\pmod{d}$.\n
}

\par\noindent
What is $\overline{g}=C_{g}$? Well,
\begin{align*}
	\overline{g} & =\set{g'\in{}G|g'\equiv{}g\pmod{H}}\\
	& =\set{g'|\exists{}h\in{}H\ptxt{ st{} }g'=gh}\\
	& =\set{gh|h\in{}H}\\
	& =gH
\end{align*}

\defn{
	$gH$ is the \textbf{left coset} of $H$ defined by $g$.\n
}

\par\noindent
By our previous considerations the left cosets of $H$ form a partition of $G$.\n

\defn{
	The \textbf{index} of $H$ in $G$ is the number of left cosets, denoted $[G:H]$.\n
}

\prop{
	$[G:H]$ is the number of right cosets.\n
	Proof: $\map{\set{\ptxt{left cosets}}}{\set{\ptxt{right cosets}}}{gH}{Hg\inv}$.\n
}

\par\noindent
Observe: For any element $g\in{}G$, $\#(gH)=\#H$.\n

\thm{
	$\#G=\#H\cdot[G:H]$.\n
}

\cor{
	(Lagrange's Theorem) If $\#G$ is finite, then $\#H|\#G$.\n
}

\cor{
	If $G$ is finite $g\in{}G$, then $\ord(g)|\#G$.\n
}

\par\noindent
Suppose $G$ is a group, $N$ is a normal subgroup. Then for any $g\in{}G$, we have $gN=Ng$, because for $n\in{}N$, $gn=\underbrace{(gng\inv)}_{\in{}N}g\in{}Ng$, so $gN\subseteq{}Ng$. (The other direction follows similarly.)\n

\defn{
	The \textbf{quotient group} $G/N$ is the set of cosets of $N$, where $(gN)(g'N)=(gg')N$.\n
}

\end{document}
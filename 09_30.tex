\documentclass[10pt,letterpaper]{article}
\usepackage[utf8]{inputenc}
\usepackage[intlimits]{amsmath}
\usepackage{amsfonts}
\usepackage{amssymb}
\usepackage{ragged2e}
\usepackage[letterpaper, margin=1in]{geometry}
\usepackage{graphicx}
\usepackage{cancel}
\usepackage{mathtools}
\usepackage{tabularx}
\usepackage{arydshln}
\usepackage{tensor}
\usepackage{array}
\usepackage{xcolor}
\usepackage[boxed]{algorithm}
\usepackage[noend]{algpseudocode}
\usepackage{listings}
\usepackage{textcomp}
\usepackage[pdf,tmpdir,singlefile]{graphviz}
\usepackage{mathrsfs}
\usepackage{bbm}
\usepackage{tikz}
\usepackage{tikz-cd}
\usepackage{enumitem}
\usepackage{arydshln}
\usepackage{relsize}
\usepackage{multicol}
\usepackage{scalerel}

\usetikzlibrary{bayesnet}
\setlist{noitemsep}

%%%%%%%%%%%%%%%%%%%%%%%%%%%%%
% Formatting commands
%%%%%%%%%%%%%%%%%%%%%%%%%%%%%
\newcommand{\n}{\hfill\break}
\newcommand{\up}{\vspace{-\baselineskip}}
\newcommand{\hangblock}[2]{\par\noindent\settowidth{\hangindent}{\textbf{#1: }}\textbf{#1: }\!\!\!#2}
\newcommand{\lemma}[1]{\hangblock{Lemma}{#1}}
\newcommand{\defn}[1]{\hangblock{Defn}{#1}}
\newcommand{\thm}[1]{\hangblock{Thm}{#1}}
\newcommand{\cor}[1]{\hangblock{Cor}{#1}}
\newcommand{\prop}[1]{\hangblock{Prop}{#1}}
\newcommand{\ex}[1]{\hangblock{Ex}{#1}}
\newcommand{\exer}[1]{\hangblock{Exer}{#1}}
\newcommand{\fact}[1]{\hangblock{Fact}{#1}}
\newcommand{\remark}[1]{\hangblock{Remark}{#1}}
\newcommand{\proven}{\;$\square$\n}
\newcommand{\problem}[1]{\par\noindent{#1}\n}
\newcommand{\problempart}[2]{\par\noindent\indent{}\settowidth{\hangindent}{\textbf{(#1)} \indent{}}\textbf{(#1)} #2\n}
\newcommand{\ptxt}[1]{\textrm{\textnormal{#1}}}
\newcommand{\inlineeq}[1]{\centerline{$\displaystyle #1$}}
\newcommand{\pageline}{\noindent\rule{\textwidth}{0.1pt}}

%%%%%%%%%%%%%%%%%%%%%%%%%%%%%
% Math commands
%%%%%%%%%%%%%%%%%%%%%%%%%%%%%
% Set Theory
\newcommand{\card}[1]{\left|#1\right|}
\newcommand{\set}[1]{\left\{#1\right\}}
\newcommand{\setmid}{\;\middle|\;}
\newcommand{\ps}[1]{\mathcal{P}\left(#1\right)}
\newcommand{\pfinite}[1]{\mathcal{P}^{\ptxt{finite}}\left(#1\right)}
\newcommand{\naturals}{\mathbb{N}}
\newcommand{\N}{\naturals}
\newcommand{\integers}{\mathbb{Z}}
\newcommand{\Z}{\integers}
\newcommand{\rationals}{\mathbb{Q}}
\newcommand{\Q}{\rationals}
\newcommand{\reals}{\mathbb{R}}
\newcommand{\R}{\reals}
\newcommand{\complex}{\mathbb{C}}
\newcommand{\C}{\complex}
\newcommand{\halfPlane}{\mathbb{H}}
\let\H\relax
\newcommand{\H}{\halfPlane}
\newcommand{\comp}{^{\complement}}
\DeclareMathOperator{\Hom}{Hom}
\newcommand{\Ind}{\mathbbm{1}}
\newcommand{\cut}{\setminus}
\DeclareMathOperator{\elem}{elem}

% Graph Theory
\let\deg\relax
\DeclareMathOperator{\deg}{deg}
\newcommand{\degp}{\ptxt{deg}^{+}}
\newcommand{\degn}{\ptxt{deg}^{-}}
\newcommand{\precdot}{\mathrel{\ooalign{$\prec$\cr\hidewidth\hbox{$\cdot\mkern0.5mu$}\cr}}}
\newcommand{\succdot}{\mathrel{\ooalign{$\cdot\mkern0.5mu$\cr\hidewidth\hbox{$\succ$}\cr\phantom{$\succ$}}}}
\DeclareMathOperator{\cl}{cl}
\DeclareMathOperator{\affdim}{affdim}

% Probability
\newcommand{\parSymbol}{\P}
\newcommand{\Prob}{\mathbb{P}}
\renewcommand{\P}{\Prob}
\newcommand{\Avg}{\mathbb{E}}
\newcommand{\E}{\Avg}
\DeclareMathOperator{\Var}{Var}
\DeclareMathOperator{\cov}{cov}
\DeclareMathOperator{\Unif}{Unif}
\DeclareMathOperator{\Binom}{Binom}
\newcommand{\CI}{\mathrel{\text{\scalebox{1.07}{$\perp\mkern-10mu\perp$}}}}

% Standard Math
\newcommand{\inv}{^{-1}}
\newcommand{\abs}[1]{\left|#1\right|}
\newcommand{\ceil}[1]{\left\lceil{}#1\right\rceil{}}
\newcommand{\floor}[1]{\left\lfloor{}#1\right\rfloor{}}
\newcommand{\conj}[1]{\overline{#1}}
\newcommand{\of}{\circ}
\newcommand{\tri}{\triangle}
\newcommand{\inj}{\hookrightarrow}
\newcommand{\surj}{\twoheadrightarrow}
\newcommand{\ndiv}{\nmid}
\renewcommand{\epsilon}{\varepsilon}
\newcommand{\divides}{\mid}
\newcommand{\ndivides}{\nmid}
\DeclareMathOperator{\lcm}{lcm}
\DeclareMathOperator{\sgn}{sgn}
\newcommand{\map}[4]{\!\!\!\begin{array}[t]{rcl}#1 & \!\!\!\!\to & \!\!\!\!#2\\ #3 & \!\!\!\!\mapsto & \!\!\!\!#4\end{array}}
\newcommand{\bigsum}[2]{\smashoperator[lr]{\sum_{\scalebox{#1}{$#2$}}}}

% Linear Algebra
\newcommand{\Id}{\textrm{\textnormal{Id}}}
\newcommand{\im}{\textrm{\textnormal{im}}}
\newcommand{\norm}[1]{\abs{\abs{#1}}}
\newcommand{\tpose}{^{T}\!}
\newcommand{\iprod}[1]{\left<#1\right>}
\DeclareMathOperator{\tr}{tr}
\DeclareMathOperator{\trace}{tr}
\newcommand{\chgBasMat}[3]{\!\!\tensor*[_{#1}]{\left[#2\right]}{_{#3}}}
\newcommand{\vecBas}[2]{\tensor*[]{\left[#1\right]}{_{#2}}}
\DeclareMathOperator{\GL}{GL}
\DeclareMathOperator{\Mat}{Mat}
\DeclareMathOperator{\vspan}{span}
\DeclareMathOperator{\rank}{rank}
\newcommand{\V}[1]{\vec{#1}}
\DeclareMathOperator{\proj}{proj}
\DeclareMathOperator{\compProj}{comp}
\DeclareMathOperator{\row}{row}
\newcommand{\smallPMatrix}[1]{\paren{\begin{smallmatrix}#1\end{smallmatrix}}}
\newcommand{\smallBMatrix}[1]{\brack{\begin{smallmatrix}#1\end{smallmatrix}}}

% Multilinear Algebra
\newcommand{\Lsym}{\L}
\let\L\relax
\DeclareMathOperator{\L}{\mathscr{L}}
\DeclareMathOperator{\A}{\mathcal{A}}
\DeclareMathOperator{\Alt}{Alt}
\DeclareMathOperator{\Sym}{Sym}
\newcommand{\ot}{\otimes}
\newcommand{\ox}{\otimes}
\DeclareMathOperator{\asc}{asc}
\DeclareMathOperator{\asSet}{set}
\DeclareMathOperator{\sort}{sort}
\DeclareMathOperator{\ringA}{\mathring{A}}

% Topology
\newcommand{\closure}[1]{\overline{#1}}
\newcommand{\uball}{\mathcal{U}}
\DeclareMathOperator{\Int}{Int}
\DeclareMathOperator{\Ext}{Ext}
\DeclareMathOperator{\Bd}{Bd}
\DeclareMathOperator{\rInt}{rInt}
\DeclareMathOperator{\ch}{ch}
\DeclareMathOperator{\ah}{ah}
\newcommand{\LargerTau}{\mathlarger{\mathlarger{\mathlarger{\mathlarger{\tau}}}}}
\newcommand{\Tau}{\mathcal{T}}

% Analysis
\DeclareMathOperator{\Graph}{Graph}
\DeclareMathOperator{\epi}{epi}
\DeclareMathOperator{\hypo}{hypo}
\DeclareMathOperator{\supp}{supp}
\newcommand{\lint}[2]{\underset{#1}{\overset{#2}{{\color{black}\underline{{\color{white}\overline{{\color{black}\int}}\color{black}}}}}}}
\newcommand{\uint}[2]{\underset{#1}{\overset{#2}{{\color{white}\underline{{\color{black}\overline{{\color{black}\int}}\color{black}}}}}}}
\newcommand{\alignint}[2]{\underset{#1}{\overset{#2}{{\color{white}\underline{{\color{white}\overline{{\color{black}\int}}\color{black}}}}}}}
\newcommand{\extint}{\ptxt{ext}\int}
\newcommand{\extalignint}[2]{\ptxt{ext}\alignint{#1}{#2}}
\newcommand{\conv}{\ast}
\newcommand{\pd}[2]{\frac{\partial{}#1}{\partial{}#2}}
\newcommand{\dd}[2]{\frac{d#1}{d#2}}
\newcommand{\del}{\nabla}
\DeclareMathOperator{\grad}{grad}
\DeclareMathOperator{\curl}{curl}
\let\div\relax
\DeclareMathOperator{\div}{div}
\DeclareMathOperator{\vol}{vol}

% Complex Analysis
\let\Re\relax
\DeclareMathOperator{\Re}{Re}
\let\Im\relax
\DeclareMathOperator{\Im}{Im}
\DeclareMathOperator{\Res}{Res}

% Abstract Algebra
\DeclareMathOperator{\ord}{ord}
\newcommand{\generated}[1]{\left<#1\right>}
\newcommand{\cycle}[1]{\smallPMatrix{#1}}
\newcommand{\id}{\ptxt{id}}
\newcommand{\iso}{\cong}
\DeclareMathOperator{\Aut}{Aut}
\DeclareMathOperator{\SL}{SL}
\DeclareMathOperator{\op}{op}
\newcommand{\isom}[4]{\!\!\!\begin{array}[t]{rcl}#1 & \!\!\!\!\overset{\sim}{\to} & \!\!\!\!#2\\ #3 & \!\!\!\!\mapsto & \!\!\!\!#4\end{array}}
\newcommand{\F}{\mathbb{F}}

% Convex Optimization
\newcommand{\sectionSymbol}{\S}
\let\S\relax
\newcommand{\S}{\mathbb{S}}
\DeclareMathOperator{\dist}{dist}
\DeclareMathOperator{\dom}{dom}
\DeclareMathOperator{\diag}{diag}
\DeclareMathOperator{\ones}{\mathbbm{1}}

% Proofs
\newcommand{\st}{s.t.}
\newcommand{\unique}{!}
\newcommand{\iffdef}{\overset{\ptxt{def}}{\Leftrightarrow}}
\newcommand{\eqdef}{\overset{\ptxt{def}}{=}}
\newcommand{\eqVertical}{\rotatebox[origin=c]{90}{=}}
\newcommand{\mapsfrom}{\mathrel{\reflectbox{\ensuremath{\mapsto}}}}
\newcommand{\mapsdown}{\rotatebox[origin=c]{-90}{$\mapsto$}\mkern2mu}
\newcommand{\mapsup}{\rotatebox[origin=c]{90}{$\mapsto$}\mkern2mu}
\newcommand{\from}{\!\mathrel{\reflectbox{\ensuremath{\to}}}}

% Brackets
\newcommand{\paren}[1]{\left(#1\right)}
\renewcommand{\brack}[1]{\left[#1\right]}
\renewcommand{\brace}[1]{\left\{#1\right\}}
\newcommand{\ang}[1]{\left<#1\right>}

% Algorithms
\algrenewcommand{\algorithmiccomment}[1]{\hskip 1em \texttt{// #1}}
\algrenewcommand\algorithmicrequire{\textbf{Input:}}
\algrenewcommand\algorithmicensure{\textbf{Output:}}
\newcommand{\algP}{\ptxt{\textbf{P}}}
\newcommand{\algNP}{\ptxt{\textbf{NP}}}
\newcommand{\algNPC}{\ptxt{\textbf{NP-Complete}}}
\newcommand{\algNPH}{\ptxt{\textbf{NP-Hard}}}
\newcommand{\algEXP}{\ptxt{\textbf{EXP}}}

%%%%%%%%%%%%%%%%%%%%%%%%%%%%%
% Other commands
%%%%%%%%%%%%%%%%%%%%%%%%%%%%%
\newcommand{\flag}[1]{\textbf{\textcolor{red}{#1}}}
\newcommand{\uSym}{\u}
\let\u\relax
\newcommand{\u}[1]{\underline{#1}}
\newcommand{\bSym}{\b}
\let\b\relax
\newcommand{\b}[1]{\textbf{#1}}
\newcommand{\iSym}{\i}
\let\i\relax
\newcommand{\i}[1]{\textit{#1}}

%%%%%%%%%%%%%%%%%%%%%%%%%%%%%%%%%%%%%%%
% Make l's curvy in math environments %
%%%%%%%%%%%%%%%%%%%%%%%%%%%%%%%%%%%%%%%
\mathcode`l="8000
\begingroup
\makeatletter
\lccode`\~=`\l
\DeclareMathSymbol{\lsb@l}{\mathalpha}{letters}{`l}
\lowercase{\gdef~{\ifnum\the\mathgroup=\m@ne \ell \else \lsb@l \fi}}%
\endgroup

%%%%%%%%%%%%%%%%%%%%%%%%%
% Fix \vdots and \ddots %
%%%%%%%%%%%%%%%%%%%%%%%%%
\usepackage{letltxmacro}
\LetLtxMacro\orgvdots\vdots
\LetLtxMacro\orgddots\ddots

\makeatletter
\DeclareRobustCommand\vdots{%
	\mathpalette\@vdots{}%
}
\newcommand*{\@vdots}[2]{%
	% #1: math style
	% #2: unused
	\sbox0{$#1\cdotp\cdotp\cdotp\m@th$}%
	\sbox2{$#1.\m@th$}%
	\vbox{%
		\dimen@=\wd0 %
		\advance\dimen@ -3\ht2 %
		\kern.5\dimen@
		% remove side bearings
		\dimen@=\wd2 %
		\advance\dimen@ -\ht2 %
		\dimen2=\wd0 %
		\advance\dimen2 -\dimen@
		\vbox to \dimen2{%
			\offinterlineskip
			\copy2 \vfill\copy2 \vfill\copy2 %
		}%
	}%
}
\DeclareRobustCommand\ddots{%
	\mathinner{%
		\mathpalette\@ddots{}%
		\mkern\thinmuskip
	}%
}
\newcommand*{\@ddots}[2]{%
	% #1: math style
	% #2: unused
	\sbox0{$#1\cdotp\cdotp\cdotp\m@th$}%
	\sbox2{$#1.\m@th$}%
	\vbox{%
		\dimen@=\wd0 %
		\advance\dimen@ -3\ht2 %
		\kern.5\dimen@
		% remove side bearings
		\dimen@=\wd2 %
		\advance\dimen@ -\ht2 %
		\dimen2=\wd0 %
		\advance\dimen2 -\dimen@
		\vbox to \dimen2{%
			\offinterlineskip
			\hbox{$#1\mathpunct{.}\m@th$}%
			\vfill
			\hbox{$#1\mathpunct{\kern\wd2}\mathpunct{.}\m@th$}%
			\vfill
			\hbox{$#1\mathpunct{\kern\wd2}\mathpunct{\kern\wd2}\mathpunct{.}\m@th$}%
		}%
	}%
}
\makeatother

\newcommand{\B}{
	\begin{tikzpicture}
	\filldraw [fill=red, draw=black] (0, 0) rectangle (0.37, 0.45);
	\draw [line width=0.5mm, white ] (0.1,0.08) -- (0.1,0.38);
	\draw[line width=0.5mm, white ] (0.1, 0.35) .. controls (0.2, 0.35) and (0.4, 0.2625) .. (0.1, 0.225);
	\draw[line width=0.5mm, white ] (0.1, 0.225) .. controls (0.2, 0.225) and (0.4, 0.1625) .. (0.1, 0.1);
	\end{tikzpicture}
}

\author{Professor Andrew Snowden\\ \small\i{Transcribed by Thomas Cohn}}
\title{Math 493 Lecture 8}
\date{9/30/19} % Can also use \today

\begin{document}
\maketitle
\setlength\RaggedRightParindent{\parindent}
\RaggedRight

\prop{
	Suppose $V$ is a vector space on field $K$. We have a bijection
	\[
		\begin{array}{rcl}
			\set{(v_{1},\ldots,v_{n}\in{}V^{n})} & \!\!\!\!\leftrightarrow & \!\!\!\!\set{\ptxt{linear transformation }K^{n}\to{}V}\\
			(v_{1},\ldots,v_{n}) & \!\!\!\!\mapsto & \!\!\!\!\paren{\smallBMatrix{a_{1}\\ \vdots\\ a_{n}}\to\sum_{i=1}^{n}a_{i}v_{i}}\\
			\smallBMatrix{T(e_{1}) & \cdots & T(e_{n})} & \!\!\!\!\mapsfrom & \!\!\!\!T
		\end{array}
	\]
}

\par\noindent
We write $i_{(v_{1},\ldots,v_{n})}$ for the linear transformation $K^{n}\to{}V$ corresponding to $(v_{1},\ldots,v_{n})$.
\[
	\im(i_{(v_{1},\ldots,v_{n})})=\vspan\set{v_{1},\ldots,v_{n}}
\]
\[
	\ker(i_{(v_{1},\ldots,v_{n})})=\set{\ptxt{linear relations between }v_{1},\ldots,v_{n}}=\set{(a_{1},\ldots,a_{n})\in{}K^{n}\;\middle|\;\sum_{i=1}^{n}a_{i}v_{i}=0}
\]

\par\noindent
$i_{(v_{1},\ldots,v_{n})}$ is surjective if and only if $v_{1},\ldots,v_{n}$ span, and injective if and only if $v_{1},\ldots,v_{n}$ are linearly independent. Hence, it's a bijection if and only if $v_{1},\ldots,v_{n}$ forms a basis.\n

\par\noindent
Conclusion: ordered bases of $V$ correspond with isomorphisms $K^{n}\to{}V$.\n

\par\noindent
Let $T:V\to{}W$ be a linear transformation. Let $B=(v_{1},\ldots,v_{m})$ and $C=(w_{1},\ldots,w_{n})$ be bases of $V$ and $W$, respectively.
\begin{center}
	\begin{tikzcd}
		V \rar{T} \dar{i_{B}}[sloped, below]{\sim} & W \dar{i_{C}}[sloped, below]{\sim}\\
		K^{m} \rar{T'} & K^{n}
	\end{tikzcd}
\end{center}
where $T'=(i_{C}\inv)\of(T)\of(i_{B})$.\n

\par\noindent
$T'$ is left multiplication by some $n\times{}n$ matrix, denoted $A_{p}$.\n

\defn{
	$A_{p}$ is the \textbf{matrix of $T$} with respect to $B$ and $C$.\n
}

\ex{
	Let $\map{T:P_{\le{}2}(x)}{P_{\le{}2}(x)}{f}{\dd{f}{x}}$. Let $B=C=(1,x,x^{2})$. Want to find $T'$.
	\[
		T'(e_{1})=\begin{bmatrix}0\\ 0\\ 0\end{bmatrix}\qquad{}T'(e_{2})=\begin{bmatrix}1\\ 0\\ 0\end{bmatrix}\qquad{}T'(e_{3})=\begin{bmatrix}0\\ 2\\ 0\end{bmatrix}
	\]
	So
	\[
		A=\begin{bmatrix}
			| & | & |\\
			T'(e_{1}) & T'(e_{2}) & T'(e_{3})\\
			| & | & |\\
		\end{bmatrix}=\begin{bmatrix}
			0 & 1 & 0\\
			0 & 0 & 2\\
			0 & 0 & 0
		\end{bmatrix}
	\]
	Now let $B=(2,x-x^{2},-x)$ and $C=(1,x,x^{2})$. Then
	\[
		T'(e_{1})=\begin{bmatrix}0\\ 0\\ 0\end{bmatrix}\qquad{}T'(e_{2})=\begin{bmatrix}1\\ -2\\ 0\end{bmatrix}\qquad{}T'(e_{3})=\begin{bmatrix}-1\\ 0\\ 0\end{bmatrix}
	\]
	So
	\[
		A=\begin{bmatrix}
			0 & 1 & -1\\
			0 & -2 & 0\\
			0 & 0 & 0
		\end{bmatrix}
	\]
}

\par\noindent
Now suppose $B=(v_{1},\ldots,v_{m})$ and $B'=(v'_{1},\ldots,v'_{m})$ are bases of $V$. Then
\begin{center}
	\begin{tikzcd}[column sep=2mm]
		& V \dlar[sloped,swap,pos=0.2]{\sim}[below,pos=0.4]{i_{B}} \drar[sloped,pos=0.2]{\sim}[below,pos=0.4]{i_{B'}} &\\
		K^{m} \arrow{rr}{T_{X}} & & K^{m}
	\end{tikzcd}
\end{center}
with $T_{X}=(i_{B'}\inv)\of(i_{B})$. $X$ is the matrix of $T_{X}$, associated with the standard basis of $K^{m}$.
\[
T_{X}(e_{1})=((i_{B'}\inv)\of(i_{B}))(e_{1})=(i_{B'}\inv)(v_{1})=\begin{bmatrix}
	a_{1}\\ \vdots\\ a_{m}
\end{bmatrix}\ptxt{ \st{} }v_{1}=\sum_{i=1}^{m}a_{i}v_{i}'
\]

\par\noindent
Now also suppose $C=(w_{1},\ldots,w_{n})$, $C'=(w_{1}',\ldots,w_{n}')$ are bases for $W$. Then we have
\begin{center}
	\begin{tikzcd}
		V \rar{\Id} \dar[pos=0.4]{i_{B'}}[sloped,below,pos=0.4]{\sim} & V \rar{T} \dar[pos=0.4]{i_{B}}[sloped,below,pos=0.4]{\sim} & W \dar[pos=0.4]{i_{C}}[sloped,below,pos=0.4]{\sim} \rar{\Id} & W \dar[pos=0.4]{i_{C'}}[sloped,below,pos=0.4]{\sim}\\
		K^{m} \arrow[bend right=30]{rrr}{T''} & \lar[swap]{T_{X}} K^{m} \rar{T'=A} & K^{n} \rar{T_{Y}} & K^{n}
	\end{tikzcd}
\end{center}
$T''=T_{Y}\of{}T'\of{}T_{X}\inv$. So the matrix for $T''$ is $M=YAX\inv$.\n

\par\noindent
Conclusion: If $A$ is the matrix for $T$ w.r.t. $B,C$, then $YAX\inv$ is the matrix for $T$ w.r.t. $B',C'$.\n

\defn{
	An \textbf{endomorphism} of $V$ (or a \textbf{linear operator} on $V$) is a linear transformation on $V$.\n
}

\par\noindent
Let $T$ be an endomorphism of $V$, and $B$ and ordered basis of $V$. We get a matrix $T$ w.r.t. $B$. Call it $A$.\n
If $B'$ is a different basis of $V$, $XAX\inv$ is the matrix of $T$ with respect to $B'$.\n

\defn{
	$A,A'$ are two $n\times{}n$ matrices. We say they are \textbf{similar} if $\exists{}X\in\GL_{n}(K)$ \st{} $A'=XAX\inv$.\n
}

\par\noindent
Note: this implies that if $A$ and $A'$ are matrices of $T$ w.r.t. two bases, then $A$ and $A'$ are similar.\n
Using this, we can define some numerical invariants of a linear transformation.\n

\defn{
	$\det(T)=\det(A)$, where $A$ is a matrix of $T$. This is well defined because similar matrices have equal determinants.\n
}

\defn{
	$\tr(T)=\tr(A)$, where $A$ is a matrix of $T$. This is well defined because $\tr(AB)=\tr(BA)$.\n
}

\ex{
	$\map{T:P_{\le{}2}}{P_{\le{}2}}{f}{\dd{f}{x}}$. Consider basis $(1,x,x^{2})$.\n
	Then
	\[
		A=\begin{bmatrix}
			0 & 1 & 0\\ 0 & 0 & 2\\ 0 & 0 & 0
		\end{bmatrix}\Rightarrow\tr(T)=0,\det(T)=0
	\]
}

\subsection*{Eigenvalues and Eigenvectors}

\par\noindent
Let $T$ be a linear operator on $V$.\n

\defn{
	An \textbf{eigenvector} for $T$ is a nonzero $v\in{}V$ \st{} $T(v)=\lambda{}v$, for some $\lambda\in{}K$. We say $\lambda$ is an \textbf{eigenvalue} for $T$.\n
}

\par\noindent
$\lambda$ is an eigenvalue for $T$
\begin{itemize}[label=$\Leftrightarrow$]
	\item $T-\lambda\Id$ has a nontrivial kernel.
	\item $T-\lambda\Id$ is not invertible.
	\item $\det(T-\lambda\Id)=0$.
\end{itemize}

\defn{
	The \textbf{characteristic polynomial} of $T$ is
	\[
		\mathcal{X}_{T}(t)=\det(T-t\Id)=(-t)^{m}\pm\tr(T)t^{m-1}+\cdots+\det(T)
	\]
	The eigenvalues of $T$ are the roots of $\mathcal{X}$.\n
}

\par\noindent
How to find eigenvectors of $T$:
\begin{enumerate}
	\item Compute $\mathcal{X}_{T}$.
	\item Find the roots of $\mathcal{X}_{T}$ (eigenvalues).
	\item For each eigenvalue $\lambda_{i}$, compute $\ker(T-\lambda_{i}\Id)$.
\end{enumerate}

\defn{
	Let $A$ be an $m\times{}m$ matrix. We can say $A$ is \textbf{diagonalizable} if it is similar to a diagonal matrix.\n
}

\par\noindent
Let $T:V\to{}V$ be an endomorphism. Pick a basis and supposed the matrix of $T$ is diagonalizable. So $\exists{}X\in\GL_{m}(K)$ \st{} $XAX'$ is diagonal.\n
Let $B'=(v_{1}',\ldots,v_{m}')$ be the basis with change of basis matrix $X$.\n
Then the matrix of $T$ with respect to $B'$ is
\[
A'=\begin{bmatrix}
	\lambda_{1} & 0 & \cdots & 0\\
	0 & \lambda_{2} & \cdots & 0\\
	\vdots & \vdots & \ddots & \vdots\\
	0 & 0 & \cdots & \lambda_{m}
\end{bmatrix}
\]

\par\noindent
According to
\begin{center}
	\begin{tikzcd}
		V \rar{T} \dar{i_{B}}[sloped,below]{\sim} & V \dar{i_{B'}}[sloped,below]{\sim}\\
		K^{m} \rar{T_{A}'} & K^{m}
	\end{tikzcd}
\end{center}
So $T_{A}'(e_{1})=A'e_{1}=\lambda{}e_{1}$, thus $T(v_{1}')=\lambda_{1}v_{1}'$.\n

\par\noindent
Conclusion: Let $T:V\to{}V$ be an endomorphism with matrix $A$ w.r.t. some basis. The following are equivalent:
\begin{enumerate}
	\item $A$ is diagonalizable.
	\item There is a basis of $V$ consisting of eigenvectors of $T$.
\end{enumerate}

\end{document}
\documentclass[10pt,letterpaper]{article}
\usepackage[utf8]{inputenc}
\usepackage[intlimits]{amsmath}
\usepackage{amsfonts}
\usepackage{amssymb}
\usepackage{ragged2e}
\usepackage[letterpaper, margin=1in]{geometry}
\usepackage{graphicx}
\usepackage{cancel}
\usepackage{mathtools}
\usepackage{tabularx}
\usepackage{arydshln}
\usepackage{tensor}
\usepackage{array}
\usepackage{xcolor}
\usepackage[boxed]{algorithm}
\usepackage[noend]{algpseudocode}
\usepackage{listings}
\usepackage{textcomp}
\usepackage[pdf,tmpdir,singlefile]{graphviz}
\usepackage{mathrsfs}
\usepackage{bbm}
\usepackage{tikz}
\usepackage{tikz-cd}
\usepackage{enumitem}
\usepackage{arydshln}
\usepackage{relsize}
\usepackage{multicol}
\usepackage{scalerel}

\usetikzlibrary{bayesnet}

%%%%%%%%%%%%%%%%%%%%%%%%%%%%%
% Formatting commands
%%%%%%%%%%%%%%%%%%%%%%%%%%%%%
\newcommand{\n}{\hfill\break}
\newcommand{\up}{\vspace{-\baselineskip}}
\newcommand{\hangblock}[2]{\par\noindent\settowidth{\hangindent}{\textbf{#1: }}\textbf{#1: }\!\!\!#2}
\newcommand{\lemma}[1]{\hangblock{Lemma}{#1}}
\newcommand{\defn}[1]{\hangblock{Defn}{#1}}
\newcommand{\thm}[1]{\hangblock{Thm}{#1}}
\newcommand{\cor}[1]{\hangblock{Cor}{#1}}
\newcommand{\prop}[1]{\hangblock{Prop}{#1}}
\newcommand{\ex}[1]{\hangblock{Ex}{#1}}
\newcommand{\exer}[1]{\hangblock{Exer}{#1}}
\newcommand{\fact}[1]{\hangblock{Fact}{#1}}
\newcommand{\remark}[1]{\hangblock{Remark}{#1}}
\newcommand{\proven}{\;$\square$\n}
\newcommand{\problem}[1]{\par\noindent{#1}\n}
\newcommand{\problempart}[2]{\par\noindent\indent{}\settowidth{\hangindent}{\textbf{(#1)} \indent{}}\textbf{(#1)} #2\n}
\newcommand{\ptxt}[1]{\textrm{\textnormal{#1}}}
\newcommand{\inlineeq}[1]{\centerline{$\displaystyle #1$}}
\newcommand{\pageline}{\noindent\rule{\textwidth}{0.1pt}}

%%%%%%%%%%%%%%%%%%%%%%%%%%%%%
% Math commands
%%%%%%%%%%%%%%%%%%%%%%%%%%%%%
% Set Theory
\newcommand{\card}[1]{\left|#1\right|}
\newcommand{\set}[1]{\left\{#1\right\}}
\newcommand{\setmid}{\;\middle|\;}
\newcommand{\ps}[1]{\mathcal{P}\left(#1\right)}
\newcommand{\pfinite}[1]{\mathcal{P}^{\ptxt{finite}}\left(#1\right)}
\newcommand{\naturals}{\mathbb{N}}
\newcommand{\N}{\naturals}
\newcommand{\integers}{\mathbb{Z}}
\newcommand{\Z}{\integers}
\newcommand{\rationals}{\mathbb{Q}}
\newcommand{\Q}{\rationals}
\newcommand{\reals}{\mathbb{R}}
\newcommand{\R}{\reals}
\newcommand{\complex}{\mathbb{C}}
\newcommand{\C}{\complex}
\newcommand{\halfPlane}{\mathbb{H}}
\let\H\relax
\newcommand{\H}{\halfPlane}
\newcommand{\comp}{^{\complement}}
\DeclareMathOperator{\Hom}{Hom}
\newcommand{\Ind}{\mathbbm{1}}
\newcommand{\cut}{\setminus}
\DeclareMathOperator{\elem}{elem}

% Graph Theory
\let\deg\relax
\DeclareMathOperator{\deg}{deg}
\newcommand{\degp}{\ptxt{deg}^{+}}
\newcommand{\degn}{\ptxt{deg}^{-}}
\newcommand{\precdot}{\mathrel{\ooalign{$\prec$\cr\hidewidth\hbox{$\cdot\mkern0.5mu$}\cr}}}
\newcommand{\succdot}{\mathrel{\ooalign{$\cdot\mkern0.5mu$\cr\hidewidth\hbox{$\succ$}\cr\phantom{$\succ$}}}}
\DeclareMathOperator{\cl}{cl}
\DeclareMathOperator{\affdim}{affdim}

% Probability
\newcommand{\parSymbol}{\P}
\newcommand{\Prob}{\mathbb{P}}
\renewcommand{\P}{\Prob}
\newcommand{\Avg}{\mathbb{E}}
\newcommand{\E}{\Avg}
\DeclareMathOperator{\Var}{Var}
\DeclareMathOperator{\cov}{cov}
\DeclareMathOperator{\Unif}{Unif}
\DeclareMathOperator{\Binom}{Binom}
\newcommand{\CI}{\mathrel{\text{\scalebox{1.07}{$\perp\mkern-10mu\perp$}}}}

% Standard Math
\newcommand{\inv}{^{-1}}
\newcommand{\abs}[1]{\left|#1\right|}
\newcommand{\ceil}[1]{\left\lceil{}#1\right\rceil{}}
\newcommand{\floor}[1]{\left\lfloor{}#1\right\rfloor{}}
\newcommand{\conj}[1]{\overline{#1}}
\newcommand{\of}{\circ}
\newcommand{\tri}{\triangle}
\newcommand{\inj}{\hookrightarrow}
\newcommand{\surj}{\twoheadrightarrow}
\newcommand{\ndiv}{\nmid}
\renewcommand{\epsilon}{\varepsilon}
\newcommand{\divides}{\mid}
\newcommand{\ndivides}{\nmid}
\DeclareMathOperator{\lcm}{lcm}
\DeclareMathOperator{\sgn}{sgn}
\newcommand{\map}[4]{\!\!\!\begin{array}[t]{rcl}#1 & \!\!\!\!\to & \!\!\!\!#2\\ #3 & \!\!\!\!\mapsto & \!\!\!\!#4\end{array}}
\newcommand{\bigsum}[2]{\smashoperator[lr]{\sum_{\scalebox{#1}{$#2$}}}}

% Linear Algebra
\newcommand{\Id}{\textrm{\textnormal{Id}}}
\newcommand{\im}{\textrm{\textnormal{im}}}
\newcommand{\norm}[1]{\abs{\abs{#1}}}
\newcommand{\tpose}{^{T}}
\newcommand{\iprod}[1]{\left<#1\right>}
\DeclareMathOperator{\tr}{tr}
\DeclareMathOperator{\trace}{tr}
\newcommand{\chgBasMat}[3]{\!\!\tensor*[_{#1}]{\left[#2\right]}{_{#3}}}
\newcommand{\vecBas}[2]{\tensor*[]{\left[#1\right]}{_{#2}}}
\DeclareMathOperator{\GL}{GL}
\DeclareMathOperator{\Mat}{Mat}
\DeclareMathOperator{\vspan}{span}
\DeclareMathOperator{\rank}{rank}
\newcommand{\V}[1]{\vec{#1}}
\DeclareMathOperator{\proj}{proj}
\DeclareMathOperator{\compProj}{comp}
\DeclareMathOperator{\row}{row}
\newcommand{\smallPMatrix}[1]{\paren{\begin{smallmatrix}#1\end{smallmatrix}}}
\newcommand{\smallBMatrix}[1]{\brack{\begin{smallmatrix}#1\end{smallmatrix}}}

% Multilinear Algebra
\newcommand{\Lsym}{\L}
\let\L\relax
\DeclareMathOperator{\L}{\mathscr{L}}
\DeclareMathOperator{\A}{\mathcal{A}}
\DeclareMathOperator{\Alt}{Alt}
\DeclareMathOperator{\Sym}{Sym}
\newcommand{\ot}{\otimes}
\newcommand{\ox}{\otimes}
\DeclareMathOperator{\asc}{asc}
\DeclareMathOperator{\asSet}{set}
\DeclareMathOperator{\sort}{sort}
\DeclareMathOperator{\ringA}{\mathring{A}}

% Topology
\newcommand{\closure}[1]{\overline{#1}}
\newcommand{\uball}{\mathcal{U}}
\DeclareMathOperator{\Int}{Int}
\DeclareMathOperator{\Ext}{Ext}
\DeclareMathOperator{\Bd}{Bd}
\DeclareMathOperator{\rInt}{rInt}
\DeclareMathOperator{\ch}{ch}
\DeclareMathOperator{\ah}{ah}
\newcommand{\LargerTau}{\mathlarger{\mathlarger{\mathlarger{\mathlarger{\tau}}}}}
\newcommand{\Tau}{\mathcal{T}}

% Analysis
\DeclareMathOperator{\Graph}{Graph}
\DeclareMathOperator{\epi}{epi}
\DeclareMathOperator{\hypo}{hypo}
\DeclareMathOperator{\supp}{supp}
\newcommand{\lint}[2]{\underset{#1}{\overset{#2}{{\color{black}\underline{{\color{white}\overline{{\color{black}\int}}\color{black}}}}}}}
\newcommand{\uint}[2]{\underset{#1}{\overset{#2}{{\color{white}\underline{{\color{black}\overline{{\color{black}\int}}\color{black}}}}}}}
\newcommand{\alignint}[2]{\underset{#1}{\overset{#2}{{\color{white}\underline{{\color{white}\overline{{\color{black}\int}}\color{black}}}}}}}
\newcommand{\extint}{\ptxt{ext}\int}
\newcommand{\extalignint}[2]{\ptxt{ext}\alignint{#1}{#2}}
\newcommand{\conv}{\ast}
\newcommand{\pd}[2]{\frac{\partial{}#1}{\partial{}#2}}
\newcommand{\del}{\nabla}
\DeclareMathOperator{\grad}{grad}
\DeclareMathOperator{\curl}{curl}
\let\div\relax
\DeclareMathOperator{\div}{div}
\DeclareMathOperator{\vol}{vol}

% Complex Analysis
\let\Re\relax
\DeclareMathOperator{\Re}{Re}
\let\Im\relax
\DeclareMathOperator{\Im}{Im}
\DeclareMathOperator{\Res}{Res}

% Abstract Algebra
\DeclareMathOperator{\ord}{ord}
\newcommand{\generated}[1]{\left<#1\right>}
\newcommand{\cycle}[1]{\smallPMatrix{#1}}
\newcommand{\id}{\ptxt{id}}
\newcommand{\iso}{\cong}
\DeclareMathOperator{\Aut}{Aut}
\DeclareMathOperator{\SL}{SL}
\DeclareMathOperator{\op}{op}
\newcommand{\isom}[4]{\!\!\!\begin{array}[t]{rcl}#1 & \!\!\!\!\overset{\sim}{\to} & \!\!\!\!#2\\ #3 & \!\!\!\!\mapsto & \!\!\!\!#4\end{array}}

% Convex Optimization
\newcommand{\sectionSymbol}{\S}
\let\S\relax
\newcommand{\S}{\mathbb{S}}

% Proofs
\newcommand{\st}{s.t.}
\newcommand{\unique}{!}
\newcommand{\iffdef}{\overset{\ptxt{def}}{\Leftrightarrow}}
\newcommand{\eqdef}{\overset{\ptxt{def}}{=}}
\newcommand{\eqVertical}{\rotatebox[origin=c]{90}{=}}
\newcommand{\mapsfrom}{\mathrel{\reflectbox{\ensuremath{\mapsto}}}}
\newcommand{\mapsdown}{\rotatebox[origin=c]{-90}{$\mapsto$}\mkern2mu}
\newcommand{\mapsup}{\rotatebox[origin=c]{90}{$\mapsto$}\mkern2mu}
\newcommand{\from}{\!\mathrel{\reflectbox{\ensuremath{\to}}}}

% Brackets
\newcommand{\paren}[1]{\left(#1\right)}
\renewcommand{\brack}[1]{\left[#1\right]}
\renewcommand{\brace}[1]{\left\{#1\right\}}
\newcommand{\ang}[1]{\left<#1\right>}

% Algorithms
\algrenewcommand{\algorithmiccomment}[1]{\hskip 1em \texttt{// #1}}
\algrenewcommand\algorithmicrequire{\textbf{Input:}}
\algrenewcommand\algorithmicensure{\textbf{Output:}}
\newcommand{\algP}{\ptxt{\textbf{P}}}
\newcommand{\algNP}{\ptxt{\textbf{NP}}}
\newcommand{\algNPC}{\ptxt{\textbf{NP-Complete}}}
\newcommand{\algNPH}{\ptxt{\textbf{NP-Hard}}}
\newcommand{\algEXP}{\ptxt{\textbf{EXP}}}

%%%%%%%%%%%%%%%%%%%%%%%%%%%%%
% Other commands
%%%%%%%%%%%%%%%%%%%%%%%%%%%%%
\newcommand{\flag}[1]{\textbf{\textcolor{red}{#1}}}
\newcommand{\uSym}{\u}
\let\u\relax
\newcommand{\u}[1]{\underline{#1}}
\newcommand{\bSym}{\b}
\let\b\relax
\newcommand{\b}[1]{\textbf{#1}}
\newcommand{\iSym}{\i}
\let\i\relax
\newcommand{\i}[1]{\textit{#1}}

%%%%%%%%%%%%%%%%%%%%%%%%%%%%%%%%%%%%%%%
% Make l's curvy in math environments %
%%%%%%%%%%%%%%%%%%%%%%%%%%%%%%%%%%%%%%%
\mathcode`l="8000
\begingroup
\makeatletter
\lccode`\~=`\l
\DeclareMathSymbol{\lsb@l}{\mathalpha}{letters}{`l}
\lowercase{\gdef~{\ifnum\the\mathgroup=\m@ne \ell \else \lsb@l \fi}}%
\endgroup

%%%%%%%%%%%%%%%%%%%%%%%%%
% Fix \vdots and \ddots %
%%%%%%%%%%%%%%%%%%%%%%%%%
\usepackage{letltxmacro}
\LetLtxMacro\orgvdots\vdots
\LetLtxMacro\orgddots\ddots

\makeatletter
\DeclareRobustCommand\vdots{%
	\mathpalette\@vdots{}%
}
\newcommand*{\@vdots}[2]{%
	% #1: math style
	% #2: unused
	\sbox0{$#1\cdotp\cdotp\cdotp\m@th$}%
	\sbox2{$#1.\m@th$}%
	\vbox{%
		\dimen@=\wd0 %
		\advance\dimen@ -3\ht2 %
		\kern.5\dimen@
		% remove side bearings
		\dimen@=\wd2 %
		\advance\dimen@ -\ht2 %
		\dimen2=\wd0 %
		\advance\dimen2 -\dimen@
		\vbox to \dimen2{%
			\offinterlineskip
			\copy2 \vfill\copy2 \vfill\copy2 %
		}%
	}%
}
\DeclareRobustCommand\ddots{%
	\mathinner{%
		\mathpalette\@ddots{}%
		\mkern\thinmuskip
	}%
}
\newcommand*{\@ddots}[2]{%
	% #1: math style
	% #2: unused
	\sbox0{$#1\cdotp\cdotp\cdotp\m@th$}%
	\sbox2{$#1.\m@th$}%
	\vbox{%
		\dimen@=\wd0 %
		\advance\dimen@ -3\ht2 %
		\kern.5\dimen@
		% remove side bearings
		\dimen@=\wd2 %
		\advance\dimen@ -\ht2 %
		\dimen2=\wd0 %
		\advance\dimen2 -\dimen@
		\vbox to \dimen2{%
			\offinterlineskip
			\hbox{$#1\mathpunct{.}\m@th$}%
			\vfill
			\hbox{$#1\mathpunct{\kern\wd2}\mathpunct{.}\m@th$}%
			\vfill
			\hbox{$#1\mathpunct{\kern\wd2}\mathpunct{\kern\wd2}\mathpunct{.}\m@th$}%
		}%
	}%
}
\makeatother

\newcommand{\B}{
	\begin{tikzpicture}
	\filldraw [fill=red, draw=black] (0, 0) rectangle (0.37, 0.45);
	\draw [line width=0.5mm, white ] (0.1,0.08) -- (0.1,0.38);
	\draw[line width=0.5mm, white ] (0.1, 0.35) .. controls (0.2, 0.35) and (0.4, 0.2625) .. (0.1, 0.225);
	\draw[line width=0.5mm, white ] (0.1, 0.225) .. controls (0.2, 0.225) and (0.4, 0.1625) .. (0.1, 0.1);
	\end{tikzpicture}
}

\author{Professor Andrew Snowden\\ \small\i{Transcribed by Thomas Cohn}}
\title{Math 493 Lecture 4}
\date{9/16/2019} % Can also use \today

\begin{document}
\maketitle
\setlength\RaggedRightParindent{\parindent}
\RaggedRight

\defn{
	Let $G$ be a group, $A,B\in{}G$ subsets. We define $AB=\set{ab|a\in{}A,b\in{}B}$. If $A=\set{a}$, then we write it $aB$ instead of $\set{a}B$.\n
}

\par\noindent
Warning: If $A,B$ are subgroups, $AB$ is not always a subgroup. If $ab\in{}AB$, $a'b'\in{}AB$, then\n
$(ab)(a'b')\ne{}aa'bb'$ in general.\n

\ex{
	$G=S_{3}$, $A=\generated{\cycle{1 & 2}}=\set{1,\cycle{1 & 2}}$, $B=\generated{\cycle{1 & 3}}=\set{1, \cycle{1 & 3}}$. Then\n
	$AB=\set{1,\cycle{2 & 3},\cycle{1 & 2},\cycle{1 & 2}\cycle{2 & 3}}=\set{1,\cycle{2 & 3},\cycle{1 & 2},\cycle{1 & 3 & 2}}$. So $AB$ is not a subgroup by Lagrange's Theorem.\n
}

\prop{
	$A,B,C\subseteq{}G$ subsets. Then $(AB)C=A(BC)$.\n
	Proof: Say $x\in(AB)C$. Then $x=(ab)c$ for some $a\in{}A,b\in{}B,c\in{}C$. So $x=a(bc)$, so $x\in{}A(BC)$.\proven
}

\par\noindent
Recall: a subgroup $N$ of $G$ is normal if $\forall{}g\in{}G,n\in{}N$, we have $gng\inv\in{}N$. This is true\n
$\Leftrightarrow{}gNg\inv\subseteq{}N,\forall{}g\in{}G$\n
$\Leftrightarrow{}gNg\inv=N,\forall{}g\in{}G$ because given $n\in{}N$, $g\inv{}ng\in{}N$, $g(g\inv{}ng)g\inv=n\in{}gNg\inv$\n
$\Leftrightarrow{}gN=Ng$, $\forall{}g\in{}G$ because $(gNg\inv)g=gN(g\inv{}g)=gN$\n

\par\noindent
Fix a normal subgroup $N\subset{}G$. Define $G/N$ to be the set of all cosets of $N$, $\set{gN|g\in{}G}$. Define a composition law on $G/N$ using product of subsets.\n

\par\noindent
Verify: $(gN)(hN)=gNhN=ghNN=ghN$. So it's a composition law. It's associative because multiplication of subsets is associative. $N$ is the identity, because $(gN)N=g(NN)=gN$, and\n
$N(gN)=NgN=gNN=gN$.\n

\par\noindent
Inverses: $(gN)(g\inv{}N)=gg\inv{}N=N$, and $(g\inv{}N)(gN)=g\inv{}gN=N$. So $g\inv{}N$ is the inverse of $gN$.\n

\par\noindent
Thus, $G/N$ is a group!\n

\par\noindent
We have a function $\map{\pi:G}{G/N}{g}{gN}$ and it is a group homomorphism (and surjective).\n
$\pi(g)\pi(h)=(gN)(hN)=(gh)N=\pi(gh)$.\n

\prop{
	$\ker(\pi)=N$.\n
	Proof: If $n\in{}N$, then $\pi(n)=nN=N$. So $N\subset\ker(\pi)$.\n
	Let $\pi(g)=N$. Then $gN=N$. So $g\in{}N$. So $\ker(\pi)\subset{}N$.\proven
}

\par\noindent
Given $g\in{}G$, put $\bar{g}=\pi(g)=gN$. Every element of $G/N$ has the form $\bar{g}$ for some $g$.\n
Warning: $\bar{g}=\bar{h}\Leftrightarrow{}gh\inv\in{}N$.\n

\ex{
	$G=\Z$, $N=n\Z,G/N=\Z/n\Z=\set{\bar{0},\bar{1},\ldots,\overline{n-1}}$.\n
	$\#G/N=n$. $\bar{a}+\bar{b}=\overline{a+b}$, $\bar{a}=\bar{b}$ iff $a\equiv{}b\pmod{n}$.\n
}

\subsection*{Mapping Property for Quotients}

\begin{tikzcd}
	G \arrow[rd, "\psi"] \arrow[d, "\pi"] & \\
	G/N \arrow[r, "\varphi"] & H
\end{tikzcd}
Given $\varphi$, we get $\psi$ by $\psi=\varphi\of\pi$. If $n\in{}N$, then $\psi(n)=\varphi(\pi(n))=1$, so $N\subset\ker(\psi)$.\n

\prop{
	Given a group homomorphism $\psi:G\to{}H$ \st{} $N\subset\ker(\psi)$, $\exists\unique\varphi:G/N\to{}H$ \st{} $\psi=\varphi\of\pi$. Moreover, $\varphi$ is surjective iff $\phi$ is surjective;\n
	in fact $\im(\varphi)=\im(\psi)$, and $\varphi$ is injective iff $\ker(\psi=N)$.\n
	Proof: attempt to define $\varphi(\bar{g})=\psi(g)$. Check that this is well defined:\n
	If $\bar{g}=\bar{h}$, then $\varphi(\bar{g})=\psi(g)\overset{\ptxt{?}}{=}\psi(h)$. Well, $\bar{g}=\bar{h}\Leftrightarrow{}g=hn$ for some $n\in{}N$. So\n
	$\psi(g)=\psi(hn)=\psi(h)\psi(n)=\psi(h)$, because $n\in\ker(\psi)$.\n
	Verify that $\varphi$ is a group homomorphism:\n
	\inlineeq{
		\varphi(\bar{g}\cdot\bar{h})=\varphi(\overline{gh})=\psi(gh)=\psi(g)\psi(h)=\varphi(\bar{g})\varphi(\bar{h})
	}
	$\varphi$ is unique because $\pi$ is surjective. Suppose $\varphi:G/N\to{}H$ \st{} $\psi=\varphi\of\pi$. Evaluate at $g$: $\varphi(\bar{g})=\psi(g)$. So all values of $\varphi$ are determined.\n
	\n
	$\im(\varphi)=\im(\psi)$:\n
	Say $x\in\im(\varphi)$. Then $x=\varphi(\bar{g})=\psi(g)\Rightarrow{}x\in\im(\Psi)$.\n
	Say $x\in\im(\psi)$. Then $x=\psi(g)=\varphi(\bar{g})\Rightarrow{}x\in\im(\varphi)$.\n
	\n
	Suppose $\ker(\psi)=N$. Say $\varphi(\bar{g})=1$. Then $\psi(g)=1$, so $g\in\ker(\psi)=N$. Thus, $\bar{g}=1$ in $G/N$. $\ker(\varphi)=1\Rightarrow\varphi$ is injective.\n
	\n
	Suppose $\varphi$ is injective, $g\in\ker(\psi)$. $\psi(g)=1$, so $\varphi(\bar{g})=1$, so $\bar{g}\in\ker(\varphi)$. Thus, $\bar{g}=\id$ in $G/N$, so $\bar{g}=N$. Thus, $\ker(\psi)\subseteq{}N$.\proven
}

\cor{
	(First Isomorphism Theorem) Suppose $\psi:G\to{}H$ is a homomorphism. Then we have a natural isomorphism $G/\ker(\psi)\overset{\sim}{\to}\im(\psi)$.\n
	Proof: Let $N=\ker(\psi)$ (a normal subgroup). Because $N\subseteq\ker(\psi)$, $\exists\unique\varphi:G/N\to{}H$ \st{} $\psi=\varphi\of\pi$.\n
	Then because $\im\varphi=\im\psi$ and $\varphi$ is injective, $\varphi$ is a bijection between $G/N$ and $\im\varphi=\im\psi$.\proven
}

\ex{
	Let $G$ be a group and let $g\in{}G$ of order $1\le{}n<\infty$. We have a group homomorphism $\map{\psi:\Z}{G}{m}{g^{m}}$. $\im(\psi)=\generated{g}$, $\ker(\psi)=n\Z$. So according to the first isomorphism theorem, we have $\isom{\varphi:\Z/n\Z}{\generated{g}}{\bar{m}}{g^{m}}$.\n
}

\cor{
	Any two cyclic groups of the same order are isomorphic.\n
	Proof: Any cyclic group of order $n$ is isomorphic to $\Z/n\Z$.\n
}

\par\noindent
Furthermore, any cyclic group of order $\infty$ is isomorphic to $\Z$.\n

\ex{
	Define $S^{1}=\set{z\in\C\mid\abs{z}=1}$ (the unit circle in the complex plane).\n
	Observe:
	\begin{itemize}
		\item $S_{1}$ is a group under multiplication.
		\item $\abs{1}=1$, so $1\in{}S^{1}$.
		\item $z,w\in{}S^{1}\Rightarrow\abs{zw}=1\Rightarrow\abs{z}\abs{w}=1$.
		\item $z\in{}S^{1}\Rightarrow\abs{z\inv}=\abs{z}\inv=1$.
	\end{itemize}\up\n
	We have a group homomorphism $\map{\psi:\R}{S^{1}}{x}{e^{2\pi{}ix}}$.
	\begin{itemize}
		\item $\abs{\psi(x)}=\abs{e^{2\pi{}ix}}=1$.
		\item $\psi(x+y)=e^{2\pi{}i(x+y)}=e^{2\pi{}ix}e^{2\pi{}iy}=\psi(x)\psi(y)$.
		\item $\ker(\psi)=\Z$.
	\end{itemize}\up\n
	Thus, by the first isomorphism theorem, $\isom{\R/\Z}{S^{1}}{\bar{x}}{e^{2\pi{}ix}}$.\n
}

\ex{
	$S_{n}/A_{n}\cong\Z/n\Z$ if $n\ge{}2$. We have $\sgn:S_{2}\to\set{\pm{}1}\cong\Z/2\Z$. $\sgn$ is surjective if $n\ge{}2$.\n
	So by the first isomorphism theorem, $S_{n}/\underbrace{\ker(\sgn)}_{=A_{n}}\cong\im(\sgn)\cong\Z/2\Z$.\n
}

\par\noindent
Fact: $\#G/N=[G:N]$. If $G$ is finite, then $\#G/N=\frac{\#G}{\#N}$.\n

\subsection*{Product Groups}

\par\noindent
Let $G,H$ be groups. Define $G\times{}H$ as a group (the direct product). Elements are ordered pairs $(g,h)$ for $g\in{}G,h\in{}H$. We have the composition law $(g,h)(g',h')=(gg',hh')$.\n

\exer{
	Check that this is a group.\n
}

\defn{
	Suppose $K$ is a group, and we have two subgroups $\bar{G},\bar{H}\subseteq{}K$. Then $K$ is the \textbf{internal product} (or \textbf{direct product}) of $\bar{G}$ and $\bar{H}$ if
	\begin{enumerate}
		\item $x\in\bar{G},y\in\bar{H}\Rightarrow{}xy=yx$.
		\item $\bar{G}\cap\bar{H}=\set{1}$.
		\item $K=\bar{G}\bar{H}$.
	\end{enumerate}\up\n
}

\par\noindent
$K=G\times{}H$. Let $\bar{G}=\set{(g,1)\mid{}g\in{}G}\subseteq{}K$ and $\bar{H}=\set{(1,h)\mid{}h\in{}H}\subseteq{}K$ subgroups.

\end{document}
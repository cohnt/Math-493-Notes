\documentclass[10pt,letterpaper]{article}
\usepackage[utf8]{inputenc}
\usepackage[intlimits]{amsmath}
\usepackage{amsfonts}
\usepackage{amssymb}
\usepackage{ragged2e}
\usepackage[letterpaper, margin=1in]{geometry}
\usepackage{graphicx}
\usepackage{cancel}
\usepackage{mathtools}
\usepackage{tabularx}
\usepackage{arydshln}
\usepackage{tensor}
\usepackage{array}
\usepackage{xcolor}
\usepackage[boxed]{algorithm}
\usepackage[noend]{algpseudocode}
\usepackage{listings}
\usepackage{textcomp}
\usepackage[pdf,tmpdir,singlefile]{graphviz}
\usepackage{mathrsfs}
\usepackage{bbm}
\usepackage{tikz}
\usepackage{tikz-cd}
\usepackage{enumitem}
\usepackage{arydshln}
\usepackage{relsize}
\usepackage{multicol}
\usepackage{scalerel}

\usetikzlibrary{bayesnet}

%%%%%%%%%%%%%%%%%%%%%%%%%%%%%
% Formatting commands
%%%%%%%%%%%%%%%%%%%%%%%%%%%%%
\newcommand{\n}{\hfill\break}
\newcommand{\up}{\vspace{-\baselineskip}}
\newcommand{\hangblock}[2]{\par\noindent\settowidth{\hangindent}{\textbf{#1: }}\textbf{#1: }\!\!\!#2}
\newcommand{\lemma}[1]{\hangblock{Lemma}{#1}}
\newcommand{\defn}[1]{\hangblock{Defn}{#1}}
\newcommand{\thm}[1]{\hangblock{Thm}{#1}}
\newcommand{\cor}[1]{\hangblock{Cor}{#1}}
\newcommand{\prop}[1]{\hangblock{Prop}{#1}}
\newcommand{\ex}[1]{\hangblock{Ex}{#1}}
\newcommand{\exer}[1]{\hangblock{Exer}{#1}}
\newcommand{\fact}[1]{\hangblock{Fact}{#1}}
\newcommand{\remark}[1]{\hangblock{Remark}{#1}}
\newcommand{\proven}{\;$\square$\n}
\newcommand{\problem}[1]{\par\noindent{#1}\n}
\newcommand{\problempart}[2]{\par\noindent\indent{}\settowidth{\hangindent}{\textbf{(#1)} \indent{}}\textbf{(#1)} #2\n}
\newcommand{\ptxt}[1]{\textrm{\textnormal{#1}}}
\newcommand{\inlineeq}[1]{\centerline{$\displaystyle #1$}}
\newcommand{\pageline}{\noindent\rule{\textwidth}{0.1pt}}

%%%%%%%%%%%%%%%%%%%%%%%%%%%%%
% Math commands
%%%%%%%%%%%%%%%%%%%%%%%%%%%%%
% Set Theory
\newcommand{\card}[1]{\left|#1\right|}
\newcommand{\set}[1]{\left\{#1\right\}}
\newcommand{\setmid}{\;\middle|\;}
\newcommand{\ps}[1]{\mathcal{P}\left(#1\right)}
\newcommand{\pfinite}[1]{\mathcal{P}^{\ptxt{finite}}\left(#1\right)}
\newcommand{\naturals}{\mathbb{N}}
\newcommand{\N}{\naturals}
\newcommand{\integers}{\mathbb{Z}}
\newcommand{\Z}{\integers}
\newcommand{\rationals}{\mathbb{Q}}
\newcommand{\Q}{\rationals}
\newcommand{\reals}{\mathbb{R}}
\newcommand{\R}{\reals}
\newcommand{\complex}{\mathbb{C}}
\newcommand{\C}{\complex}
\newcommand{\halfPlane}{\mathbb{H}}
\let\H\relax
\newcommand{\H}{\halfPlane}
\newcommand{\comp}{^{\complement}}
\DeclareMathOperator{\Hom}{Hom}
\newcommand{\Ind}{\mathbbm{1}}
\newcommand{\cut}{\setminus}
\DeclareMathOperator{\elem}{elem}

% Graph Theory
\let\deg\relax
\DeclareMathOperator{\deg}{deg}
\newcommand{\degp}{\ptxt{deg}^{+}}
\newcommand{\degn}{\ptxt{deg}^{-}}
\newcommand{\precdot}{\mathrel{\ooalign{$\prec$\cr\hidewidth\hbox{$\cdot\mkern0.5mu$}\cr}}}
\newcommand{\succdot}{\mathrel{\ooalign{$\cdot\mkern0.5mu$\cr\hidewidth\hbox{$\succ$}\cr\phantom{$\succ$}}}}
\DeclareMathOperator{\cl}{cl}
\DeclareMathOperator{\affdim}{affdim}

% Probability
\newcommand{\parSymbol}{\P}
\newcommand{\Prob}{\mathbb{P}}
\renewcommand{\P}{\Prob}
\newcommand{\Avg}{\mathbb{E}}
\newcommand{\E}{\Avg}
\DeclareMathOperator{\Var}{Var}
\DeclareMathOperator{\cov}{cov}
\DeclareMathOperator{\Unif}{Unif}
\DeclareMathOperator{\Binom}{Binom}
\newcommand{\CI}{\mathrel{\text{\scalebox{1.07}{$\perp\mkern-10mu\perp$}}}}

% Standard Math
\newcommand{\inv}{^{-1}}
\newcommand{\abs}[1]{\left|#1\right|}
\newcommand{\ceil}[1]{\left\lceil{}#1\right\rceil{}}
\newcommand{\floor}[1]{\left\lfloor{}#1\right\rfloor{}}
\newcommand{\conj}[1]{\overline{#1}}
\newcommand{\of}{\circ}
\newcommand{\tri}{\triangle}
\newcommand{\inj}{\hookrightarrow}
\newcommand{\surj}{\twoheadrightarrow}
\newcommand{\ndiv}{\nmid}
\renewcommand{\epsilon}{\varepsilon}
\newcommand{\divides}{\mid}
\newcommand{\ndivides}{\nmid}
\DeclareMathOperator{\lcm}{lcm}
\DeclareMathOperator{\sgn}{sgn}
\newcommand{\map}[4]{\!\!\!\begin{array}[t]{rcl}#1 & \!\!\!\!\to & \!\!\!\!#2\\ #3 & \!\!\!\!\mapsto & \!\!\!\!#4\end{array}}
\newcommand{\bigsum}[2]{\smashoperator[lr]{\sum_{\scalebox{#1}{$#2$}}}}

% Linear Algebra
\newcommand{\Id}{\textrm{\textnormal{Id}}}
\newcommand{\im}{\textrm{\textnormal{im}}}
\newcommand{\norm}[1]{\abs{\abs{#1}}}
\newcommand{\tpose}{^{T}\!}
\newcommand{\iprod}[1]{\left<#1\right>}
\DeclareMathOperator{\tr}{tr}
\DeclareMathOperator{\trace}{tr}
\newcommand{\chgBasMat}[3]{\!\!\tensor*[_{#1}]{\left[#2\right]}{_{#3}}}
\newcommand{\vecBas}[2]{\tensor*[]{\left[#1\right]}{_{#2}}}
\DeclareMathOperator{\GL}{GL}
\DeclareMathOperator{\Mat}{Mat}
\DeclareMathOperator{\vspan}{span}
\DeclareMathOperator{\rank}{rank}
\newcommand{\V}[1]{\vec{#1}}
\DeclareMathOperator{\proj}{proj}
\DeclareMathOperator{\compProj}{comp}
\DeclareMathOperator{\row}{row}
\newcommand{\smallPMatrix}[1]{\paren{\begin{smallmatrix}#1\end{smallmatrix}}}
\newcommand{\smallBMatrix}[1]{\brack{\begin{smallmatrix}#1\end{smallmatrix}}}

% Multilinear Algebra
\newcommand{\Lsym}{\L}
\let\L\relax
\DeclareMathOperator{\L}{\mathscr{L}}
\DeclareMathOperator{\A}{\mathcal{A}}
\DeclareMathOperator{\Alt}{Alt}
\DeclareMathOperator{\Sym}{Sym}
\newcommand{\ot}{\otimes}
\newcommand{\ox}{\otimes}
\DeclareMathOperator{\asc}{asc}
\DeclareMathOperator{\asSet}{set}
\DeclareMathOperator{\sort}{sort}
\DeclareMathOperator{\ringA}{\mathring{A}}

% Topology
\newcommand{\closure}[1]{\overline{#1}}
\newcommand{\uball}{\mathcal{U}}
\DeclareMathOperator{\Int}{Int}
\DeclareMathOperator{\Ext}{Ext}
\DeclareMathOperator{\Bd}{Bd}
\DeclareMathOperator{\rInt}{rInt}
\DeclareMathOperator{\ch}{ch}
\DeclareMathOperator{\ah}{ah}
\newcommand{\LargerTau}{\mathlarger{\mathlarger{\mathlarger{\mathlarger{\tau}}}}}
\newcommand{\Tau}{\mathcal{T}}

% Analysis
\DeclareMathOperator{\Graph}{Graph}
\DeclareMathOperator{\epi}{epi}
\DeclareMathOperator{\hypo}{hypo}
\DeclareMathOperator{\supp}{supp}
\newcommand{\lint}[2]{\underset{#1}{\overset{#2}{{\color{black}\underline{{\color{white}\overline{{\color{black}\int}}\color{black}}}}}}}
\newcommand{\uint}[2]{\underset{#1}{\overset{#2}{{\color{white}\underline{{\color{black}\overline{{\color{black}\int}}\color{black}}}}}}}
\newcommand{\alignint}[2]{\underset{#1}{\overset{#2}{{\color{white}\underline{{\color{white}\overline{{\color{black}\int}}\color{black}}}}}}}
\newcommand{\extint}{\ptxt{ext}\int}
\newcommand{\extalignint}[2]{\ptxt{ext}\alignint{#1}{#2}}
\newcommand{\conv}{\ast}
\newcommand{\pd}[2]{\frac{\partial{}#1}{\partial{}#2}}
\newcommand{\del}{\nabla}
\DeclareMathOperator{\grad}{grad}
\DeclareMathOperator{\curl}{curl}
\let\div\relax
\DeclareMathOperator{\div}{div}
\DeclareMathOperator{\vol}{vol}

% Complex Analysis
\let\Re\relax
\DeclareMathOperator{\Re}{Re}
\let\Im\relax
\DeclareMathOperator{\Im}{Im}
\DeclareMathOperator{\Res}{Res}

% Abstract Algebra
\DeclareMathOperator{\ord}{ord}
\newcommand{\generated}[1]{\left<#1\right>}
\newcommand{\cycle}[1]{\smallPMatrix{#1}}
\newcommand{\id}{\ptxt{id}}
\newcommand{\iso}{\cong}
\DeclareMathOperator{\Aut}{Aut}
\DeclareMathOperator{\SL}{SL}
\DeclareMathOperator{\op}{op}
\newcommand{\isom}[4]{\!\!\!\begin{array}[t]{rcl}#1 & \!\!\!\!\overset{\sim}{\to} & \!\!\!\!#2\\ #3 & \!\!\!\!\mapsto & \!\!\!\!#4\end{array}}
\newcommand{\F}{\mathbb{F}}

% Convex Optimization
\newcommand{\sectionSymbol}{\S}
\let\S\relax
\newcommand{\S}{\mathbb{S}}

% Proofs
\newcommand{\st}{s.t.}
\newcommand{\unique}{!}
\newcommand{\iffdef}{\overset{\ptxt{def}}{\Leftrightarrow}}
\newcommand{\eqdef}{\overset{\ptxt{def}}{=}}
\newcommand{\eqVertical}{\rotatebox[origin=c]{90}{=}}
\newcommand{\mapsfrom}{\mathrel{\reflectbox{\ensuremath{\mapsto}}}}
\newcommand{\mapsdown}{\rotatebox[origin=c]{-90}{$\mapsto$}\mkern2mu}
\newcommand{\mapsup}{\rotatebox[origin=c]{90}{$\mapsto$}\mkern2mu}
\newcommand{\from}{\!\mathrel{\reflectbox{\ensuremath{\to}}}}

% Brackets
\newcommand{\paren}[1]{\left(#1\right)}
\renewcommand{\brack}[1]{\left[#1\right]}
\renewcommand{\brace}[1]{\left\{#1\right\}}
\newcommand{\ang}[1]{\left<#1\right>}

% Algorithms
\algrenewcommand{\algorithmiccomment}[1]{\hskip 1em \texttt{// #1}}
\algrenewcommand\algorithmicrequire{\textbf{Input:}}
\algrenewcommand\algorithmicensure{\textbf{Output:}}
\newcommand{\algP}{\ptxt{\textbf{P}}}
\newcommand{\algNP}{\ptxt{\textbf{NP}}}
\newcommand{\algNPC}{\ptxt{\textbf{NP-Complete}}}
\newcommand{\algNPH}{\ptxt{\textbf{NP-Hard}}}
\newcommand{\algEXP}{\ptxt{\textbf{EXP}}}

%%%%%%%%%%%%%%%%%%%%%%%%%%%%%
% Other commands
%%%%%%%%%%%%%%%%%%%%%%%%%%%%%
\newcommand{\flag}[1]{\textbf{\textcolor{red}{#1}}}
\newcommand{\uSym}{\u}
\let\u\relax
\newcommand{\u}[1]{\underline{#1}}
\newcommand{\bSym}{\b}
\let\b\relax
\newcommand{\b}[1]{\textbf{#1}}
\newcommand{\iSym}{\i}
\let\i\relax
\newcommand{\i}[1]{\textit{#1}}

%%%%%%%%%%%%%%%%%%%%%%%%%%%%%%%%%%%%%%%
% Make l's curvy in math environments %
%%%%%%%%%%%%%%%%%%%%%%%%%%%%%%%%%%%%%%%
\mathcode`l="8000
\begingroup
\makeatletter
\lccode`\~=`\l
\DeclareMathSymbol{\lsb@l}{\mathalpha}{letters}{`l}
\lowercase{\gdef~{\ifnum\the\mathgroup=\m@ne \ell \else \lsb@l \fi}}%
\endgroup

%%%%%%%%%%%%%%%%%%%%%%%%%
% Fix \vdots and \ddots %
%%%%%%%%%%%%%%%%%%%%%%%%%
\usepackage{letltxmacro}
\LetLtxMacro\orgvdots\vdots
\LetLtxMacro\orgddots\ddots

\makeatletter
\DeclareRobustCommand\vdots{%
	\mathpalette\@vdots{}%
}
\newcommand*{\@vdots}[2]{%
	% #1: math style
	% #2: unused
	\sbox0{$#1\cdotp\cdotp\cdotp\m@th$}%
	\sbox2{$#1.\m@th$}%
	\vbox{%
		\dimen@=\wd0 %
		\advance\dimen@ -3\ht2 %
		\kern.5\dimen@
		% remove side bearings
		\dimen@=\wd2 %
		\advance\dimen@ -\ht2 %
		\dimen2=\wd0 %
		\advance\dimen2 -\dimen@
		\vbox to \dimen2{%
			\offinterlineskip
			\copy2 \vfill\copy2 \vfill\copy2 %
		}%
	}%
}
\DeclareRobustCommand\ddots{%
	\mathinner{%
		\mathpalette\@ddots{}%
		\mkern\thinmuskip
	}%
}
\newcommand*{\@ddots}[2]{%
	% #1: math style
	% #2: unused
	\sbox0{$#1\cdotp\cdotp\cdotp\m@th$}%
	\sbox2{$#1.\m@th$}%
	\vbox{%
		\dimen@=\wd0 %
		\advance\dimen@ -3\ht2 %
		\kern.5\dimen@
		% remove side bearings
		\dimen@=\wd2 %
		\advance\dimen@ -\ht2 %
		\dimen2=\wd0 %
		\advance\dimen2 -\dimen@
		\vbox to \dimen2{%
			\offinterlineskip
			\hbox{$#1\mathpunct{.}\m@th$}%
			\vfill
			\hbox{$#1\mathpunct{\kern\wd2}\mathpunct{.}\m@th$}%
			\vfill
			\hbox{$#1\mathpunct{\kern\wd2}\mathpunct{\kern\wd2}\mathpunct{.}\m@th$}%
		}%
	}%
}
\makeatother

\newcommand{\B}{
	\begin{tikzpicture}
	\filldraw [fill=red, draw=black] (0, 0) rectangle (0.37, 0.45);
	\draw [line width=0.5mm, white ] (0.1,0.08) -- (0.1,0.38);
	\draw[line width=0.5mm, white ] (0.1, 0.35) .. controls (0.2, 0.35) and (0.4, 0.2625) .. (0.1, 0.225);
	\draw[line width=0.5mm, white ] (0.1, 0.225) .. controls (0.2, 0.225) and (0.4, 0.1625) .. (0.1, 0.1);
	\end{tikzpicture}
}

\author{Thomas Cohn}
\title{Math 493 Lecture 5}
\date{9/18/2019} % Can also use \today

\begin{document}
\maketitle
\setlength\RaggedRightParindent{\parindent}
\RaggedRight

\par\noindent
We define $\R^{n}$ to be the set of column vectors of size $z$. It has two important operations: addition and scalar multiplication.\n

\par\noindent
Most things in linear algebra work with $\R$ replaced by $\C$ or $\Q$. $\R$, $\C$, and $\Q$ are examples of fields.\n

\defn{
	A \textbf{field} is a set $K$ equipped with $2$ composition laws, $+$ (addition) and $\cdot$ (multiplication) \st{}
	\begin{itemize}
		\item $(K,+)$ is an abelian group with identity element $0$.
		\item $(K^{\times},\cdot)$ is an abelian group with identity element $1$. ($K^{\times}\eqdef{}K\cut\set{0}$).
		\item $\forall{}a,b,c\in{}K$, $a\cdot(b+c)=(a\cdot{}b)+(a\cdot{}c)$ (multiplicative distribution).
	\end{itemize}\up\n
}

\ex{
	If $K$ is any field, define $K(T)$ to be the set of rational functions with coefficients in $K$. A rational function looks like\n
	\inlineeq{
		\frac{a_{n}T^{n}+\cdots+a_{0}}{b_{m}T^{m}+\cdots+b_{0}}\quad{}a_{i},b_{j}\in{}K
	}\n
}

\ex{
	$\Q[i]=\set{a+bi\mid{}a,b\in\Q}$ is field.\n
	$\Q[\sqrt{2}]=\set{a+b\sqrt{2}\mid{}a,b\in\Q}$ is a field.\n
}

\ex{
	For $p$ prime, $\mathbb{F}_{p}=\Z/p\Z$ is a field. Addition and multiplication are the usual modular operations.\n
	(See paper notes for justification.)
}

\par\noindent
Observe: If $K$ is a field, and $a,b\in{}K^{\times}$, then $ab\ne{}0$. This is because $K^{\times}$ is closed under multiplication, and $0\not\in{}K^{\times}$.\n

\ex{
	$\Z/6\Z$ is not a field.\n
	$\bar{2}\cdot\bar{3}=\bar{6}=\bar{0}$. But $\bar{2},\bar{3}\ne\bar{0}$.\n
}

\par\noindent
More generally, if $n$ is composite, $n=ab$, for $1<a,b<n$. So $\bar{a}\cdot\bar{b}=\bar{n}=\bar{0}$, but $\bar{a},\bar{b}\ne\bar{0}$.\n
Thus, $\Z/n\Z$ is not a field.\n

\ex{
	Suppose $K$ is a field, $a\in{}K$ is not a square (i.e. $a\ne{}b^{2}$, for any $b\in{}K$). Then we define $K(\sqrt{a})=\set{b+c\sqrt{a}\mid{}b,c\in{}K}$, with the obvious addition and multiplication. This is a field.\n
	Note: we get inversion by\n
	\inlineeq{
		\frac{1}{b+c\sqrt{a}}=\frac{b-c\sqrt{a}}{b^{2}-c^{2}a}
	}
	The denominator is nonzero because $b^{2}/c^{2}=(b/c)^{2}$ is a perfect square, and $a$ is not.\n
}

\ex{
	$K=\mathbb{F}_{3}=\set{0,1,2}$, $2=-1$ is not a square. $0^{2}=0$, $1^{2}=1$, $2^{2}=4=1$. So there's a field $\mathbb{F}_{3}(\sqrt{-1})$. $\#\mathbb{F}_{3}(\sqrt{-1})=9$.\n
}

\ex{
	$K=\F_{5}=\set{0,1,2,3,4}$. $-1$ is a square -- $-1=4=2^{2}$. $2$ is not so we get a field $\F_{5}(\sqrt{2})$.\n
	$\#\F_{5}(\sqrt{2})=25$.\n
}

\subsection*{Vector Spaces}

\par\noindent
Fix a field $K$.\n

\defn{
	A \textbf{vector space} over $K$ is a set $V$ equipped with two operations:
	\begin{itemize}
		\item $+:V\times{}V\to{}V$ (addition)
		\item $\cdot:K\times{}V\to{}V$ (scalar multiplication)
	\end{itemize}\up\n
	Such that
	\begin{itemize}
		\item $(V,+)$ is an abelian group (write $0$ for the identity element).
		\item Given $a,b\in{}K$, $v\in{}V$, then $a(bV)=(ab)V$.
		\item $1\cdot{}v=v$.
		\item Distributive law: for $a,b\in{}K$ and $v,w\in{}V$, $(a+b)v=av+bv$ and $a(v+w)=av+aw$.
	\end{itemize}\up\n
}

\ex{
	$V=K[t]$ (all polynomials with coefficients in $K$) is a vector space.\n
	$V=M_{n}(K)\cong{}K^{n^{2}}$ ($n\times{}n$ matrices in $K$) is a vector space.\n
}

\defn{
	$V,W$ vector spaces over $K$. A \textbf{linear map} is a function $T:V\to{}W$ \st{} $T(v_{1}+v_{2})=T(v_{1})+T(v_{2})$ and $T(av_{1})=aT(v_{1})$, for all $a\in{}K$ and $v_{1},v_{2}\in{}V$.\n
}

\defn{
	An \textbf{isomorphism} is a bijective linear map.\n
}

\ex{
	$\C$ is a vector space over $\R$. As an $\R$-vector space, $\C\cong\R^{2}$, where $a+bi\mapsto\smallBMatrix{a\\ b}$.\n
	More generally, if $K$ is a subfield of $L$, then $L$ is naturally a $K$-vector space.\n
}

\defn{
	Let $V$ be a $K$-vector space, and $S\subseteq{}V$. Define the \textbf{span} of $S$, denoted $\vspan(S)$, to be the set of all finite linear combinations of elements of $S$.\n
}

\defn{
	A set $S\subseteq{}V$ which is closed under addition and scalar multiplication is a \textbf{subspace} of $V$.\n
}

\par\noindent
Note: $\vspan(S)$ is closed under addition and scalar multiplication, so it's a subspace of $V$.\n

\defn{
	We say $S$ \textbf{spans} $V$, or is a \textbf{spanning set} if $\vspan(S)=V$.\n
}

\defn{
	We say $S$ is \textbf{linearly independent} if given $v_{1},\ldots,v_{n}\in{}S$ distinct, if $\sum_{i=1}^{n}a_{i}v_{i}=0$, then $a_{i}=0$, $\forall{}i$.\n
}

\defn{
	$S$ is a \textbf{basis} if it's a spanning set and linearly independent.\n
}

\ex{
	$V=K^{3}$, $S=\set{\smallBMatrix{1\\ 1\\ 0},\smallBMatrix{0\\ 1\\ 1}}$, $\vspan(S)=\set{\smallBMatrix{a\\ b\\ c}\mid{}b=a+c}$.\n
	$S$ is not a spanning set, but it is linearly independent.\n
}

\newpage
\prop{
	Let $S$ be a subset of $V$. The following are equivalent:
	\begin{enumerate}
		\item $S$ is a basis.
		\item $S$ is a minimal spanning set, i.e., $S$ is a spanning set, but no proper subset of $S$ is.
		\item $S$ is a maximal linearly independent set, i.e., $S$ is linearly independent, but no proper superset of $S$ is.
	\end{enumerate}\up\n
	Proof:\n
	$(1)\Rightarrow(2)$: $S$ is a basis. By definition, $S$ spans. Let $T\subsetneq{}S$ that spans, and let $v\in{}S\cut{}T$. Since $T$ spans, $\exists{}w_{1},\ldots,w_{n}\in{}T,a_{1},\ldots,a_{n}\in{}K$ \st{} $v=a_{1}w_{1}+\cdots+a_{n}w_{n}$. But $0=-v+a_{1}w_{1}+\cdots+a_{n}w_{n}$, so $S$ is not linearly independent. Oops!\n
	\n
	$(2)\Rightarrow(1)$: $S$ is a minimal spanning set. We need to show that $S$ is linearly independent, so suppose not. Then we have $a_{1}v_{1}+\cdots+a_{n}v_{n}=0$ with not all $a_{i}=0$, and $v_{i}\in{}S$ distinct. WOLOG $a_{1}=a$. Then $v_{1}=-a_{2}v_{2}-\cdots-a_{n}v_{n}$. We will show that $T=S\cut\set{v_{1}}$ spans.\n
	Let $x\in{}V$ be given. Since $S$ spans, $x=b_{1}w_{1}+\cdots+b_{m}w_{m}$. If no $w_{i}=v_{1}$, then $x\in\vspan(T)$. Otherwise, WOLOG $w_{m}=v_{1}$. Then\n
	$x=b_{1}w_{1}+\cdots+b_{m-1}w_{m-1}+b_{m}(-a_{2}v_{2}-\cdots-a_{n}v_{n})\in\vspan(S\cut\set{v_{1}})=\vspan(T)$.
}

\end{document}
\documentclass[10pt,letterpaper]{article}
\usepackage[utf8]{inputenc}
\usepackage[intlimits]{amsmath}
\usepackage{amsfonts}
\usepackage{amssymb}
\usepackage{ragged2e}
\usepackage[letterpaper, margin=1in]{geometry}
\usepackage{graphicx}
\usepackage{cancel}
\usepackage{mathtools}
\usepackage{tabularx}
\usepackage{arydshln}
\usepackage{tensor}
\usepackage{array}
\usepackage{xcolor}
\usepackage[boxed]{algorithm}
\usepackage[noend]{algpseudocode}
\usepackage{listings}
\usepackage{textcomp}
\usepackage[pdf,tmpdir,singlefile]{graphviz}
\usepackage{mathrsfs}
\usepackage{bbm}
\usepackage{tikz}
\usepackage{enumitem}
\usepackage{arydshln}
\usepackage{relsize}
\usepackage{multicol}

\usetikzlibrary{bayesnet}

%%%%%%%%%%%%%%%%%%%%%%%%%%%%%
% Formatting commands
%%%%%%%%%%%%%%%%%%%%%%%%%%%%%
\newcommand{\n}{\hfill\break}
\newcommand{\up}{\vspace{-\baselineskip}}
\newcommand{\hangblock}[2]{\par\noindent\settowidth{\hangindent}{\textbf{#1: }}\textbf{#1: }\!\!\!#2}
\newcommand{\lemma}[1]{\hangblock{Lemma}{#1}}
\newcommand{\defn}[1]{\hangblock{Defn}{#1}}
\newcommand{\thm}[1]{\hangblock{Thm}{#1}}
\newcommand{\cor}[1]{\hangblock{Cor}{#1}}
\newcommand{\prop}[1]{\hangblock{Prop}{#1}}
\newcommand{\ex}[1]{\hangblock{Ex}{#1}}
\newcommand{\exer}[1]{\hangblock{Exer}{#1}}
\newcommand{\fact}[1]{\hangblock{Fact}{#1}}
\newcommand{\remark}[1]{\hangblock{Remark}{#1}}
\newcommand{\proven}{\;$\square$\n}
\newcommand{\problem}[1]{\par\noindent{#1}\n}
\newcommand{\problempart}[2]{\par\noindent\indent{}\settowidth{\hangindent}{\textbf{(#1)} \indent{}}\textbf{(#1)} #2\n}
\newcommand{\ptxt}[1]{\textrm{\textnormal{#1}}}
\newcommand{\inlineeq}[1]{\centerline{$\displaystyle #1$}}
\newcommand{\pageline}{\noindent\rule{\textwidth}{0.1pt}}

%%%%%%%%%%%%%%%%%%%%%%%%%%%%%
% Math commands
%%%%%%%%%%%%%%%%%%%%%%%%%%%%%
% Set Theory
\newcommand{\card}[1]{\left|#1\right|}
\newcommand{\set}[1]{\left\{#1\right\}}
\newcommand{\setmid}{\;\middle|\;}
\newcommand{\ps}[1]{\mathcal{P}\left(#1\right)}
\newcommand{\pfinite}[1]{\mathcal{P}^{\ptxt{finite}}\left(#1\right)}
\newcommand{\naturals}{\mathbb{N}}
\newcommand{\N}{\naturals}
\newcommand{\integers}{\mathbb{Z}}
\newcommand{\Z}{\integers}
\newcommand{\rationals}{\mathbb{Q}}
\newcommand{\Q}{\rationals}
\newcommand{\reals}{\mathbb{R}}
\newcommand{\R}{\reals}
\newcommand{\complex}{\mathbb{C}}
\newcommand{\C}{\complex}
\newcommand{\halfPlane}{\mathbb{H}}
\let\H\relax
\newcommand{\H}{\halfPlane}
\newcommand{\comp}{^{\complement}}
\DeclareMathOperator{\Hom}{Hom}
\newcommand{\Ind}{\mathbbm{1}}
\newcommand{\cut}{\setminus}
\DeclareMathOperator{\elem}{elem}

% Graph Theory
\let\deg\relax
\DeclareMathOperator{\deg}{deg}
\newcommand{\degp}{\ptxt{deg}^{+}}
\newcommand{\degn}{\ptxt{deg}^{-}}
\newcommand{\precdot}{\mathrel{\ooalign{$\prec$\cr\hidewidth\hbox{$\cdot\mkern0.5mu$}\cr}}}
\newcommand{\succdot}{\mathrel{\ooalign{$\cdot\mkern0.5mu$\cr\hidewidth\hbox{$\succ$}\cr\phantom{$\succ$}}}}
\DeclareMathOperator{\cl}{cl}
\DeclareMathOperator{\affdim}{affdim}

% Probability
\newcommand{\parSymbol}{\P}
\newcommand{\Prob}{\mathbb{P}}
\renewcommand{\P}{\Prob}
\newcommand{\Avg}{\mathbb{E}}
\newcommand{\E}{\Avg}
\DeclareMathOperator{\Var}{Var}
\DeclareMathOperator{\cov}{cov}
\DeclareMathOperator{\Unif}{Unif}
\DeclareMathOperator{\Binom}{Binom}
\newcommand{\CI}{\mathrel{\text{\scalebox{1.07}{$\perp\mkern-10mu\perp$}}}}

% Standard Math
\newcommand{\inv}{^{-1}}
\newcommand{\abs}[1]{\left|#1\right|}
\newcommand{\ceil}[1]{\left\lceil{}#1\right\rceil{}}
\newcommand{\floor}[1]{\left\lfloor{}#1\right\rfloor{}}
\newcommand{\conj}[1]{\overline{#1}}
\newcommand{\of}{\circ}
\newcommand{\tri}{\triangle}
\newcommand{\inj}{\hookrightarrow}
\newcommand{\surj}{\twoheadrightarrow}
\newcommand{\ndiv}{\nmid}
\renewcommand{\epsilon}{\varepsilon}
\newcommand{\divides}{\mid}
\newcommand{\ndivides}{\nmid}
\DeclareMathOperator{\lcm}{lcm}
\DeclareMathOperator{\sgn}{sgn}
\newcommand{\map}[4]{\!\!\!\begin{array}[t]{rcl}#1 & \!\!\!\!\to & \!\!\!\!#2\\ #3 & \!\!\!\!\mapsto & \!\!\!\!#4\end{array}}
\newcommand{\bigsum}[2]{\smashoperator[lr]{\sum_{\scalebox{#1}{$#2$}}}}

% Linear Algebra
\newcommand{\Id}{\textrm{\textnormal{Id}}}
\newcommand{\im}{\textrm{\textnormal{im}}}
\newcommand{\norm}[1]{\abs{\abs{#1}}}
\newcommand{\tpose}{^{T}}
\newcommand{\iprod}[1]{\left<#1\right>}
\DeclareMathOperator{\trace}{tr}
\newcommand{\chgBasMat}[3]{\!\!\tensor*[_{#1}]{\left[#2\right]}{_{#3}}}
\newcommand{\vecBas}[2]{\tensor*[]{\left[#1\right]}{_{#2}}}
\DeclareMathOperator{\GL}{GL}
\DeclareMathOperator{\Mat}{Mat}
\DeclareMathOperator{\vspan}{span}
\DeclareMathOperator{\rank}{rank}
\newcommand{\V}[1]{\vec{#1}}
\DeclareMathOperator{\proj}{proj}
\DeclareMathOperator{\compProj}{comp}
\DeclareMathOperator{\row}{row}
\newcommand{\smallPMatrix}[1]{\paren{\begin{smallmatrix}#1\end{smallmatrix}}}
\newcommand{\smallBMatrix}[1]{\brack{\begin{smallmatrix}#1\end{smallmatrix}}}

% Multilinear Algebra
\newcommand{\Lsym}{\L}
\let\L\relax
\DeclareMathOperator{\L}{\mathscr{L}}
\DeclareMathOperator{\A}{\mathcal{A}}
\DeclareMathOperator{\Alt}{Alt}
\DeclareMathOperator{\Sym}{Sym}
\newcommand{\ot}{\otimes}
\newcommand{\ox}{\otimes}
\DeclareMathOperator{\asc}{asc}
\DeclareMathOperator{\asSet}{set}
\DeclareMathOperator{\sort}{sort}
\DeclareMathOperator{\ringA}{\mathring{A}}

% Topology
\newcommand{\closure}[1]{\overline{#1}}
\newcommand{\uball}{\mathcal{U}}
\DeclareMathOperator{\Int}{Int}
\DeclareMathOperator{\Ext}{Ext}
\DeclareMathOperator{\Bd}{Bd}
\DeclareMathOperator{\rInt}{rInt}
\DeclareMathOperator{\ch}{ch}
\DeclareMathOperator{\ah}{ah}
\newcommand{\LargerTau}{\mathlarger{\mathlarger{\mathlarger{\mathlarger{\tau}}}}}
\newcommand{\Tau}{\mathcal{T}}

% Analysis
\DeclareMathOperator{\Graph}{Graph}
\DeclareMathOperator{\epi}{epi}
\DeclareMathOperator{\hypo}{hypo}
\DeclareMathOperator{\supp}{supp}
\newcommand{\lint}[2]{\underset{#1}{\overset{#2}{{\color{black}\underline{{\color{white}\overline{{\color{black}\int}}\color{black}}}}}}}
\newcommand{\uint}[2]{\underset{#1}{\overset{#2}{{\color{white}\underline{{\color{black}\overline{{\color{black}\int}}\color{black}}}}}}}
\newcommand{\alignint}[2]{\underset{#1}{\overset{#2}{{\color{white}\underline{{\color{white}\overline{{\color{black}\int}}\color{black}}}}}}}
\newcommand{\extint}{\ptxt{ext}\int}
\newcommand{\extalignint}[2]{\ptxt{ext}\alignint{#1}{#2}}
\newcommand{\conv}{\ast}
\newcommand{\pd}[2]{\frac{\partial{}#1}{\partial{}#2}}
\newcommand{\del}{\nabla}
\DeclareMathOperator{\grad}{grad}
\DeclareMathOperator{\curl}{curl}
\let\div\relax
\DeclareMathOperator{\div}{div}
\DeclareMathOperator{\vol}{vol}

% Complex Analysis
\let\Re\relax
\DeclareMathOperator{\Re}{Re}
\let\Im\relax
\DeclareMathOperator{\Im}{Im}
\DeclareMathOperator{\Res}{Res}

% Abstract Algebra
\DeclareMathOperator{\ord}{ord}
\newcommand{\generated}[1]{\left<#1\right>}
\newcommand{\cycle}[1]{\begin{pmatrix}#1\end{pmatrix}}
\newcommand{\id}{\ptxt{id}}
\newcommand{\iso}{\cong}
\DeclareMathOperator{\Aut}{Aut}
\DeclareMathOperator{\SL}{SL}

% Convex Optimization
\newcommand{\sectionSymbol}{\S}
\let\S\relax
\newcommand{\S}{\mathbb{S}}

% Proofs
\newcommand{\st}{s.t.}
\newcommand{\unique}{!}
\newcommand{\iffdef}{\overset{\ptxt{def}}{\Leftrightarrow}}
\newcommand{\eqdef}{\overset{\ptxt{def}}{=}}
\newcommand{\eqVertical}{\rotatebox[origin=c]{90}{=}}
\newcommand{\mapsfrom}{\mathrel{\reflectbox{\ensuremath{\mapsto}}}}
\newcommand{\mapsdown}{\rotatebox[origin=c]{-90}{$\mapsto$}\mkern2mu}
\newcommand{\mapsup}{\rotatebox[origin=c]{90}{$\mapsto$}\mkern2mu}
\newcommand{\from}{\!\mathrel{\reflectbox{\ensuremath{\to}}}}

% Brackets
\newcommand{\paren}[1]{\left(#1\right)}
\renewcommand{\brack}[1]{\left[#1\right]}
\renewcommand{\brace}[1]{\left\{#1\right\}}
\newcommand{\ang}[1]{\left<#1\right>}

% Algorithms
\algrenewcommand{\algorithmiccomment}[1]{\hskip 1em \texttt{// #1}}
\algrenewcommand\algorithmicrequire{\textbf{Input:}}
\algrenewcommand\algorithmicensure{\textbf{Output:}}
\newcommand{\algP}{\ptxt{\textbf{P}}}
\newcommand{\algNP}{\ptxt{\textbf{NP}}}
\newcommand{\algNPC}{\ptxt{\textbf{NP-Complete}}}
\newcommand{\algNPH}{\ptxt{\textbf{NP-Hard}}}
\newcommand{\algEXP}{\ptxt{\textbf{EXP}}}

%%%%%%%%%%%%%%%%%%%%%%%%%%%%%
% Other commands
%%%%%%%%%%%%%%%%%%%%%%%%%%%%%
\newcommand{\flag}[1]{\textbf{\textcolor{red}{#1}}}
\newcommand{\uSym}{\u}
\let\u\relax
\newcommand{\u}[1]{\underline{#1}}
\newcommand{\bSym}{\b}
\let\b\relax
\newcommand{\b}[1]{\textbf{#1}}
\newcommand{\iSym}{\i}
\let\i\relax
\newcommand{\i}[1]{\textit{#1}}

%%%%%%%%%%%%%%%%%%%%%%%%%%%%%
% Make l's curvy in math environments
%%%%%%%%%%%%%%%%%%%%%%%%%%%%%
\mathcode`l="8000
\begingroup
\makeatletter
\lccode`\~=`\l
\DeclareMathSymbol{\lsb@l}{\mathalpha}{letters}{`l}
\lowercase{\gdef~{\ifnum\the\mathgroup=\m@ne \ell \else \lsb@l \fi}}%
\endgroup

\newcommand{\B}{
    \begin{tikzpicture}
    \filldraw [fill=red, draw=black] (0, 0) rectangle (0.37, 0.45);
    \draw [line width=0.5mm, white ] (0.1,0.08) -- (0.1,0.38);
    \draw[line width=0.5mm, white ] (0.1, 0.35) .. controls (0.2, 0.35) and (0.4, 0.2625) .. (0.1, 0.225);
    \draw[line width=0.5mm, white ] (0.1, 0.225) .. controls (0.2, 0.225) and (0.4, 0.1625) .. (0.1, 0.1);
    \end{tikzpicture}
}

\author{Professor Andrew Snowden\\ \small\i{Transcribed by Thomas Cohn}}
\title{Math 493 Lecture 2}
\date{9/9/2019} % Can also use \today

\begin{document}
\maketitle
\setlength\RaggedRightParindent{\parindent}
\RaggedRight

\defn{
	$S_{n}$ is the symmetric group on $n$ letters. As a group, it can be considered to be the set of bijections on $\set{1,\ldots,n}$ under composition.
}

\par\noindent
$\ord(S_{n})=n!$, because an element of $S_{n}$ is a permutation. There are $n$ choices for the first number, $n-1$ choices for the second number, etc.\n

\ex{
	\begin{itemize}
		\item $\card{S_{2}}=2!=2$.
		\item $\card{S_{3}}=3!=6$.
		\item $\card{S_{4}}=4!=24$.
		\item $\card{S_{5}}=5!=120$.
	\end{itemize}
}

\subsection*{Cycle Notation}

\defn{
	Say $a_{1},\ldots,a_{r}\in\set{1,\ldots,n}$ distinct. Define the \textbf{$r$-cycle} $\cycle{a_{1} & a_{2} & \cdots & a_{r}}$ as the element of $S_{n}$ defined by $a_{1}\mapsto{}a_{2}\mapsto\cdots\mapsto{}a_{r}\mapsto{}a_{1}$, and $a_{i}\mapsto{}a_{i}$ for all $a_{i}\not\in\set{a_{1},\ldots,a_{r}}$.\n
}

\hangblock{Fact}{
	Every element of $S_{n}$ can be written as a product of disjoint cycles.\n
	Proof (sketch): Suppose $\sigma\in{}S_{n}$. $1\mapsto\sigma(1),\sigma(1)\mapsto\sigma^{2}(1),\ldots,\sigma^{r-1}(1)\mapsto{}1$.\footnote{Note that $\sigma^{r-1}(1)$ cannot map to some $\sigma^{k}(1)$, because elements of $S_{n}$ are bijections, and $\sigma^{k-1}(1)\mapsto\sigma^{k}(1)$.} Then successively repeat for the smallest element not already in a cycle.\n
}

\ex{
	$\cycle{1 & 2 & 3}\cycle{2 & 3 & 5}=\cycle{1 & 2}\cycle{3 & 5}$, becuase\n
	\inlineeq{
		\begin{array}{l}
			1\mapsto{}2\\
			2\mapsto{}1\\
			3\mapsto{}5\\
			4\mapsto{}4\\
			5\mapsto{}3
		\end{array}
	}
}

\ex{
	The elements of $S_{2}$ are
	\begin{itemize}
		\item $1=\id$
		\item $\cycle{1 & 2}$
	\end{itemize}
}

\ex{
	The elements of $S_{3}$ are
	\begin{itemize}
		\item $1=\id$
		\item $\cycle{1 & 2}$
		\item $\cycle{1 & 3}$
		\item $\cycle{2 & 3}$
		\item $\cycle{1 & 2 & 3}$
		\item $\cycle{1 & 3 & 2}$
	\end{itemize}
}

\par\noindent
Note that a $2$-cycle is just a transposition of two elements.\n

\fact{
	The order of an $r$-cycle is $r$, because for any $\sigma$, $\sigma^{r}=\id$, and $r$ is minimal.\n
}

\defn{
	For $G$ and $H$ groups, an \textbf{isomorphism} between $G$ and $H$ is a bijection $f:G\to{}H$ \st{} $f(xy)=f(x)f(y)$, $\forall{}x,y\in{}G$. We say $G$ and $H$ are \textbf{isomorphic}, written $G\iso{}H$, if such an isomorphism exists.
}

\ex{
	In $S_{5}$, we consider $G=\generated{\cycle{1 & 2}}=\set{\id,\cycle{1 & 2}}$ and $H=\generated{\cycle{3 & 5}}=\set{\id,\cycle{3 & 5}}$.\n
	$f:G\to{}H$ where $\cycle{1 & 2}\mapsto\cycle{3 & 5}$, $\id\mapsto\id$ is an isomorphism, so $G\iso{}H$.\n
}

\remark{
	\begin{itemize}
		\item If $f:G\to{}H$ is an isomorphism, $f\inv:H\to{}G$ is an isomorphism. So if $G\iso{}H$, then $H\iso{}G$.
		\item If $f:G\to{}H$, $g:H\to{}K$ are isomorphisms, then $g\of{}f:G\to{}K$ is an isomorphism. So if $G\iso{}H$ and $H\iso{}K$, then $G\iso{}K$.
	\end{itemize}
}

\par\noindent Note: $\id:G\to{}G$ is an isomorphism, so $G\iso{}G$. However, there are usually other isomorphisms on $G$.\n

\defn{
	An \textbf{automorphism} of $G$ is an isomorphism from $G$ to $G$. The set of all automorphisms of $G$ is denoted $\Aut(G)$, and is a group under composition.\n
}

\ex{
	$G$ is a group. Consider $\map{f:G}{G}{x}{x\inv}$. This is a bijection.\n
	$f(xy)=(xy)\inv=y\inv{}x\inv$, and $f(x)f(y)=x\inv{}y\inv$, so they're not equal in general (in fact, they're equal if and only if $G$ is abelian).\n
	So $f$ is an automorphism if and only if $G$ is abelian.\n
}

\ex{
	$G=\generated{\cycle{1 & 2 & 3}}\subseteq{}S_{3}$. That is, $G=\set{1, \cycle{1 & 2 & 3}, \cycle{1 & 3 & 2}}$.\n
	$\map{f:G}{G}{x}{x\inv}$. So $1\mapsto{}1$, $\cycle{1 & 2 & 3}\mapsto\cycle{1 & 3 & 2}$, and $\cycle{1 & 3 & 2}\mapsto\cycle{1 & 2 & 3}$.\n
}

\ex{
	$\sigma\in{}S_{3}$. Define $f:S_{3}\to{}S_{3}$, where\n
	$\begin{array}{l}
		\cycle{1 & 2} \mapsto \cycle{\sigma(1),\sigma(2)}\\
		\cycle{1 & 3} \mapsto \cycle{\sigma(1),\sigma(3)}\\
		\cycle{2 & 3} \mapsto \cycle{\sigma(2),\sigma(3)}\\
		\cycle{1 & 2 & 3} \mapsto \cycle{\sigma(1), \sigma(2), \sigma(3)}\\
		\cycle{1 & 3 & 2} \mapsto \cycle{\sigma(1), \sigma(3), \sigma(2)}\\
	\end{array}$\n
	This is an automorphism.\n
}

\defn{
	Let $G$ be a group, and $g\in{}G$. Define $\map{\gamma_{g}:G}{G}{x}{gxg\inv}$.\n
	This is called the \textbf{conjugate} of $x$ by $g$.\n
}

\hangblock{Claim}{
	$\gamma_{g}\in\Aut(G)$.
	Proof: $\gamma_{g}(\gamma_{g\inv}(x))=g\gamma_{g\inv}(x)g\inv=gg\inv{}x(g\inv)\inv{}g=x$.\n
	So $\gamma_{g}\of\gamma_{g}\inv=\id$ and $\gamma_{g\inv}\of\gamma_{g}=\id$, so $\gamma_{g}$ is a bijection, and $\gamma_{g}\inv=\gamma_{g\inv}$.\n
	$\gamma_{g}(xy)=gxyg\inv=gx(g\inv{}g)yg\inv=(gxg\inv)(gyg\inv)=\gamma_{g}(x)\gamma_{g}(y)$.\n
	Thus, $\gamma_{g}$ is an isomorphism, so $\gamma_{g}\in\Aut(G)$.\proven
}

\lemma{
	If $\sigma\in{}S_{n}$, $a_{1},\ldots,a_{r}\in\set{1,\ldots,n}$ distinct, then $\sigma\cycle{a_{1} & \cdots & a_{r}}\sigma\inv=\cycle{\sigma(a_{1}) & \cdots & \sigma(a_{r})}$.\n
}

\par\noindent
If $G$ is abelian, then $\gamma_{g}=\id$, $\forall{}g\in{}G$.\n

\ex{
	$G=\R^{2}$ under addition. $A$ is an invertible $2\times{}2$ real matrix. So $\map{f:G}{G}{x}{Ax}$ is an automorphism, because $f(x+y)=A(x+y)=Ax+Ay=f(x)+f(y)$, and because $A$ is invertible, so $f$ is indeed a bijection.\n
}

\defn{
	automorphisms defined by conjugation are called \textbf{inner automorphisms}.\n
}

\ex{
	$\SL_{n}(\R)$ is the subgroup of $\GL_{n}(\R)$ consisting of matrices with determinant $1$.\n
	$\map{f:\SL_{n}(\R)}{\SL_{n}(\R)}{x}{\tpose{}x\inv=(x\tpose)\inv}$ is an automorphism.\n
	\n
	If $f$ were inner, then $\exists{}g\in\SL_{n}(\R)$ \st{} $f=\gamma_{g}$, i.e., $\tpose{}x\inv=gxg\inv$, $\forall{}x$.\n
	So $f$ is inner if and only if $n\le{}2$.\n
}

\defn{
	Let $G$ and $H$ be groups. A (\textbf{group}) \textbf{homomorphism} from $G$ to $H$ is a function $f:G\to{}H$ \st{} $f(xy)=f(x)f(y)$, $\forall{}x,y\in{}G$.\n
}

\ex{
	$\map{\gamma:G}{\Aut(G)}{g}{\gamma_{g}}$ is a group homomorphism.\n
	Proof:
	\begin{align*}
		\gamma_{g}(\gamma_{h}(x)) & =g\gamma_{h}(x)g\inv\\
		& =g(hxh\inv)g\inv\\
		& =(gh)x(h\inv{}g\inv)\\
		& =(gh)x(gh)\inv\\
		& =\gamma_{gh}(x)
	\end{align*}
	So $\gamma_{gh}=\gamma_{g}\of\gamma_{h}$.\proven
}

\hangblock{Remark}{
	Is $\gamma:S_{n}\to\Aut(S_{n})$ an isomorphism? Sometimes, but the conditions are weird.\n
}

\ex{
	$G$ is a group, $g\in{}G$. $\map{f:\Z}{G}{n}{g^{n}}$ is a homomorphism.\n
	$f(n+m)=g^{n+m}=\underbrace{g\cdots{}g}_{n+m}=\underbrace{g\cdots{}g}_{n}\underbrace{g\cdots{}g}_{m}=g^{n}g^{m}=f(n)f(m)$.\n
}

\par\noindent
Note:
\begin{itemize}
	\item $f$ is injective $\Leftrightarrow$ $\ord(g)=\infty$. More generally, $f(i)=f(j)\Leftrightarrow\ord(g)\divides{}i-j$.
	\item $f$ is surjective $\Leftrightarrow$ $g$ generates $G$.
\end{itemize}

\defn{
	Let $f:G\to{}H$ be a group homomorphism. The \textbf{image} of $f$ is $im(f)=\set{y\in{}H|\exists{}x\in{}G\ptxt{ \st{} }y=f(x)}$.\n
}

\fact{
	$\im(f)$ is a subgroup.\n
	Proof:
	\begin{itemize}
		\item $1\in\im(f)$, because $1=f(1)$.
		\item If $y\in\im(f)$, then $y=f(x)$, so $y\inv=f(x\inv)\in\im(f)$.
		\item $y,y'\in\im(f)\Rightarrow{}y=f(x),y'=f(x')\Rightarrow{}yy'=f(xx')\in\im(f)$.
	\end{itemize}\up\n
	\proven
}

\lemma{
	If $f$ is a homomorphism, then
	\begin{itemize}
		\item $f(1)=1$. Proof: $1\cdot{}1=1$, so $f(1)=f(1\cdot{}1)=f(1)\cdot{}f(1)$. Thus, $f(1)=1$.\proven\up
		\item $f(x\inv)=f(x)\inv$. Proof: $x\cdot{}x\inv=1$, so $f(x)f(x\inv)=f(xx\inv)=f(1)=1$.\proven\up
	\end{itemize}
}

\defn{
	The \textbf{kernel} of $f$ is $\ker(f)=\set{x\in{}G|f(x)=1}$.\n
}

\fact{
	$\ker(f)$ is a subgroup of $G$.\n
	Proof:
	\begin{itemize}
		\item $f(1)=1$, so $1\in\ker(f)$.
		\item If $x\in\ker(f)$, Then $f(x)=1$, so $f(x\inv)=f(x)\inv=1\inv=1$. Thus, $x\inv\in\ker(f)$.
		\item If $x,x'\in\ker(f)$, Then $f(xx')=f(x)f(x')=1\cdot{}1=1$, so $xx'\in\ker(f)$.
	\end{itemize}\up\n
	\proven
}

\defn{
	A subgroup $K$ of $G$ is called \textbf{normal} if $\forall{}g\in{}G,x\in{}K$, $gxg\inv\in{}K$.\n
}

\fact{
	$\ker(f)$ is normal.\n
	Proof: Let $x\in\ker(f)$, $g\in{}G$. Then $f(gxg\inv)=f(g)f(x)f(g\inv)=f(g)\cdot{}1\cdot{}f(g)\inv=1$.\n
	So $gxg\inv\in\ker(f)$.\proven
}

\ex{
	$G$ is a group, $g\in{}G$. Consider $\map{f:\Z}{G}{n}{g^{n}}$. $\ker(f)=\set{n\in\Z|g^{n}=1}$.\n
	This is equal to $d\Z$, where $d=\ord(g)$.\n
}

\prop{
	Let $f:G\to{}H$ be a group homomorphism. Then $f$ is injective $\Leftrightarrow$ $\ker(f)=\set{1}$.\n
	Proof: If $f$ is injective, then $\ker(f)=\set{1}$.\n
	If $\ker(f)=\set{1}$, let $f(x)=f(y)$. Then $f(xy\inv)=f(x)f(y\inv)=f(x)f(y)\inv=1$. So $xy\inv\in\ker(f)$, so $xy\inv=1$, so $x=y$.\proven
}

\end{document}
\documentclass[10pt,letterpaper]{article}
\usepackage[utf8]{inputenc}
\usepackage[intlimits]{amsmath}
\usepackage{amsfonts}
\usepackage{amssymb}
\usepackage{ragged2e}
\usepackage[letterpaper, margin=1in]{geometry}
\usepackage{graphicx}
\usepackage{cancel}
\usepackage{mathtools}
\usepackage{tabularx}
\usepackage{arydshln}
\usepackage{tensor}
\usepackage{array}
\usepackage{xcolor}
\usepackage[boxed]{algorithm}
\usepackage[noend]{algpseudocode}
\usepackage{listings}
\usepackage{textcomp}
\usepackage[pdf,tmpdir,singlefile]{graphviz}
\usepackage{mathrsfs}
\usepackage{bbm}
\usepackage{tikz}
\usepackage{enumitem}
\usepackage{arydshln}
\usepackage{relsize}
\usepackage{multicol}

\usetikzlibrary{bayesnet}

%%%%%%%%%%%%%%%%%%%%%%%%%%%%%
% Formatting commands
%%%%%%%%%%%%%%%%%%%%%%%%%%%%%
\newcommand{\n}{\hfill\break}
\newcommand{\up}{\vspace{-\baselineskip}}
\newcommand{\lemma}[1]{\par\noindent\settowidth{\hangindent}{\textbf{Lemma: }}\textbf{Lemma: }#1}
\newcommand{\defn}[1]{\par\noindent\settowidth{\hangindent}{\textbf{Defn: }}\textbf{Defn: }#1\n}
\newcommand{\thm}[1]{\par\noindent\settowidth{\hangindent}{\textbf{Thm: }}\textbf{Thm: }#1\n}
\newcommand{\prop}[1]{\par\noindent\settowidth{\hangindent}{\textbf{Prop: }}\textbf{Prop: }#1\n}
\newcommand{\cor}[1]{\par\noindent\settowidth{\hangindent}{\textbf{Cor: }}\textbf{Cor: }#1\n}
\newcommand{\ex}[1]{\par\noindent\settowidth{\hangindent}{\textbf{Ex: }}\textbf{Ex: }#1\n}
\newcommand{\exer}[1]{\par\noindent\settowidth{\hangindent}{\textbf{Exer: }}\textbf{Exer: }#1\n}
\newcommand{\proven}{\;$\square$\n}
\newcommand{\problem}[1]{\par\noindent{#1}\n}
\newcommand{\problempart}[2]{\par\noindent\indent{}\settowidth{\hangindent}{\textbf{(#1)} \indent{}}\textbf{(#1)} #2\n}
\newcommand{\ptxt}[1]{\textrm{\textnormal{#1}}}
\newcommand{\inlineeq}[1]{\centerline{$\displaystyle #1$}}
\newcommand{\pageline}{\noindent\rule{\textwidth}{0.1pt}}

%%%%%%%%%%%%%%%%%%%%%%%%%%%%%
% Math commands
%%%%%%%%%%%%%%%%%%%%%%%%%%%%%
% Set Theory
\newcommand{\card}[1]{\left|#1\right|}
\newcommand{\set}[1]{\left\{#1\right\}}
\newcommand{\setmid}{\;\middle|\;}
\newcommand{\ps}[1]{\mathcal{P}\left(#1\right)}
\newcommand{\pfinite}[1]{\mathcal{P}^{\ptxt{finite}}\left(#1\right)}
\newcommand{\naturals}{\mathbb{N}}
\newcommand{\N}{\naturals}
\newcommand{\integers}{\mathbb{Z}}
\newcommand{\Z}{\integers}
\newcommand{\rationals}{\mathbb{Q}}
\newcommand{\Q}{\rationals}
\newcommand{\reals}{\mathbb{R}}
\newcommand{\R}{\reals}
\newcommand{\complex}{\mathbb{C}}
\newcommand{\C}{\complex}
\newcommand{\halfPlane}{\mathbb{H}}
\let\H\relax
\newcommand{\H}{\halfPlane}
\newcommand{\comp}{^{\complement}}
\DeclareMathOperator{\Hom}{Hom}
\newcommand{\Ind}{\mathbbm{1}}
\newcommand{\cut}{\setminus}
\DeclareMathOperator{\elem}{elem}

% Graph Theory
\let\deg\relax
\DeclareMathOperator{\deg}{deg}
\newcommand{\degp}{\ptxt{deg}^{+}}
\newcommand{\degn}{\ptxt{deg}^{-}}
\newcommand{\precdot}{\mathrel{\ooalign{$\prec$\cr\hidewidth\hbox{$\cdot\mkern0.5mu$}\cr}}}
\newcommand{\succdot}{\mathrel{\ooalign{$\cdot\mkern0.5mu$\cr\hidewidth\hbox{$\succ$}\cr\phantom{$\succ$}}}}
\DeclareMathOperator{\cl}{cl}
\DeclareMathOperator{\affdim}{affdim}

% Probability
\newcommand{\parSymbol}{\P}
\newcommand{\Prob}{\mathbb{P}}
\renewcommand{\P}{\Prob}
\newcommand{\Avg}{\mathbb{E}}
\newcommand{\E}{\Avg}
\DeclareMathOperator{\Var}{Var}
\DeclareMathOperator{\cov}{cov}
\DeclareMathOperator{\Unif}{Unif}
\DeclareMathOperator{\Binom}{Binom}
\newcommand{\CI}{\mathrel{\text{\scalebox{1.07}{$\perp\mkern-10mu\perp$}}}}

% Standard Math
\newcommand{\inv}{^{-1}}
\newcommand{\abs}[1]{\left|#1\right|}
\newcommand{\ceil}[1]{\left\lceil{}#1\right\rceil{}}
\newcommand{\floor}[1]{\left\lfloor{}#1\right\rfloor{}}
\newcommand{\conj}[1]{\overline{#1}}
\newcommand{\of}{\circ}
\newcommand{\tri}{\triangle}
\newcommand{\inj}{\hookrightarrow}
\newcommand{\surj}{\twoheadrightarrow}
\newcommand{\ndiv}{\nmid}
\renewcommand{\epsilon}{\varepsilon}
\newcommand{\divides}{\mid}
\newcommand{\ndivides}{\nmid}
\DeclareMathOperator{\lcm}{lcm}
\DeclareMathOperator{\sgn}{sgn}
\newcommand{\map}[4]{\!\!\!\begin{array}[t]{rcl}#1 & \!\!\!\!\to & \!\!\!\!#2\\ #3 & \!\!\!\!\mapsto & \!\!\!\!#4\end{array}}
\newcommand{\bigsum}[2]{\smashoperator[lr]{\sum_{\scalebox{#1}{$#2$}}}}

% Linear Algebra
\newcommand{\Id}{\textrm{\textnormal{Id}}}
\newcommand{\im}{\textrm{\textnormal{im}}}
\newcommand{\norm}[1]{\abs{\abs{#1}}}
\newcommand{\tpose}{^{T}}
\newcommand{\iprod}[1]{\left<#1\right>}
\DeclareMathOperator{\trace}{tr}
\newcommand{\chgBasMat}[3]{\!\!\tensor*[_{#1}]{\left[#2\right]}{_{#3}}}
\newcommand{\vecBas}[2]{\tensor*[]{\left[#1\right]}{_{#2}}}
\DeclareMathOperator{\GL}{GL}
\DeclareMathOperator{\Mat}{Mat}
\DeclareMathOperator{\vspan}{span}
\DeclareMathOperator{\rank}{rank}
\newcommand{\V}[1]{\vec{#1}}
\DeclareMathOperator{\proj}{proj}
\DeclareMathOperator{\compProj}{comp}
\DeclareMathOperator{\row}{row}
\newcommand{\smallPMatrix}[1]{\paren{\begin{smallmatrix}#1\end{smallmatrix}}}
\newcommand{\smallBMatrix}[1]{\brack{\begin{smallmatrix}#1\end{smallmatrix}}}

% Multilinear Algebra
\newcommand{\Lsym}{\L}
\let\L\relax
\DeclareMathOperator{\L}{\mathscr{L}}
\DeclareMathOperator{\A}{\mathcal{A}}
\DeclareMathOperator{\Alt}{Alt}
\DeclareMathOperator{\Sym}{Sym}
\newcommand{\ot}{\otimes}
\newcommand{\ox}{\otimes}
\DeclareMathOperator{\asc}{asc}
\DeclareMathOperator{\asSet}{set}
\DeclareMathOperator{\sort}{sort}
\DeclareMathOperator{\ringA}{\mathring{A}}

% Topology
\newcommand{\closure}[1]{\overline{#1}}
\newcommand{\uball}{\mathcal{U}}
\DeclareMathOperator{\Int}{Int}
\DeclareMathOperator{\Ext}{Ext}
\DeclareMathOperator{\Bd}{Bd}
\DeclareMathOperator{\rInt}{rInt}
\DeclareMathOperator{\ch}{ch}
\DeclareMathOperator{\ah}{ah}
\newcommand{\LargerTau}{\mathlarger{\mathlarger{\mathlarger{\mathlarger{\tau}}}}}
\newcommand{\Tau}{\mathcal{T}}

% Analysis
\DeclareMathOperator{\Graph}{Graph}
\DeclareMathOperator{\epi}{epi}
\DeclareMathOperator{\hypo}{hypo}
\DeclareMathOperator{\supp}{supp}
\newcommand{\lint}[2]{\underset{#1}{\overset{#2}{{\color{black}\underline{{\color{white}\overline{{\color{black}\int}}\color{black}}}}}}}
\newcommand{\uint}[2]{\underset{#1}{\overset{#2}{{\color{white}\underline{{\color{black}\overline{{\color{black}\int}}\color{black}}}}}}}
\newcommand{\alignint}[2]{\underset{#1}{\overset{#2}{{\color{white}\underline{{\color{white}\overline{{\color{black}\int}}\color{black}}}}}}}
\newcommand{\extint}{\ptxt{ext}\int}
\newcommand{\extalignint}[2]{\ptxt{ext}\alignint{#1}{#2}}
\newcommand{\conv}{\ast}
\newcommand{\pd}[2]{\frac{\partial{}#1}{\partial{}#2}}
\newcommand{\del}{\nabla}
\DeclareMathOperator{\grad}{grad}
\DeclareMathOperator{\curl}{curl}
\let\div\relax
\DeclareMathOperator{\div}{div}
\DeclareMathOperator{\vol}{vol}

% Complex Analysis
\let\Re\relax
\DeclareMathOperator{\Re}{Re}
\let\Im\relax
\DeclareMathOperator{\Im}{Im}
\DeclareMathOperator{\Res}{Res}

% Abstract Algebra
\DeclareMathOperator{\ord}{ord}
\newcommand{\generated}[1]{\left<#1\right>}

% Proofs
\newcommand{\st}{s.t.}
\newcommand{\unique}{!}
\newcommand{\iffdef}{\overset{\ptxt{def}}{\Leftrightarrow}}
\newcommand{\eqdef}{\overset{\ptxt{def}}{=}}
\newcommand{\eqVertical}{\rotatebox[origin=c]{90}{=}}
\newcommand{\mapsfrom}{\mathrel{\reflectbox{\ensuremath{\mapsto}}}}
\newcommand{\mapsdown}{\rotatebox[origin=c]{-90}{$\mapsto$}\mkern2mu}
\newcommand{\mapsup}{\rotatebox[origin=c]{90}{$\mapsto$}\mkern2mu}
\newcommand{\from}{\!\mathrel{\reflectbox{\ensuremath{\to}}}}

% Brackets
\newcommand{\paren}[1]{\left(#1\right)}
\renewcommand{\brack}[1]{\left[#1\right]}
\renewcommand{\brace}[1]{\left\{#1\right\}}
\newcommand{\ang}[1]{\left<#1\right>}

% Algorithms
\algrenewcommand{\algorithmiccomment}[1]{\hskip 1em \texttt{// #1}}
\algrenewcommand\algorithmicrequire{\textbf{Input:}}
\algrenewcommand\algorithmicensure{\textbf{Output:}}
\newcommand{\algP}{\ptxt{\textbf{P}}}
\newcommand{\algNP}{\ptxt{\textbf{NP}}}
\newcommand{\algNPC}{\ptxt{\textbf{NP-Complete}}}
\newcommand{\algNPH}{\ptxt{\textbf{NP-Hard}}}
\newcommand{\algEXP}{\ptxt{\textbf{EXP}}}

%%%%%%%%%%%%%%%%%%%%%%%%%%%%%
% Other commands
%%%%%%%%%%%%%%%%%%%%%%%%%%%%%
\newcommand{\flag}[1]{\textbf{\textcolor{red}{#1}}}
\newcommand{\uSym}{\u}
\let\u\relax
\newcommand{\u}[1]{\underline{#1}}
\newcommand{\bSym}{\b}
\let\b\relax
\newcommand{\b}[1]{\textbf{#1}}
\newcommand{\iSym}{\i}
\let\i\relax
\newcommand{\i}[1]{\textit{#1}}

%%%%%%%%%%%%%%%%%%%%%%%%%%%%%
% Make l's curvy in math environments
%%%%%%%%%%%%%%%%%%%%%%%%%%%%%
\mathcode`l="8000
\begingroup
\makeatletter
\lccode`\~=`\l
\DeclareMathSymbol{\lsb@l}{\mathalpha}{letters}{`l}
\lowercase{\gdef~{\ifnum\the\mathgroup=\m@ne \ell \else \lsb@l \fi}}%
\endgroup

\newcommand{\B}{
    \begin{tikzpicture}
    \filldraw [fill=red, draw=black] (0, 0) rectangle (0.37, 0.45);
    \draw [line width=0.5mm, white ] (0.1,0.08) -- (0.1,0.38);
    \draw[line width=0.5mm, white ] (0.1, 0.35) .. controls (0.2, 0.35) and (0.4, 0.2625) .. (0.1, 0.225);
    \draw[line width=0.5mm, white ] (0.1, 0.225) .. controls (0.2, 0.225) and (0.4, 0.1625) .. (0.1, 0.1);
    \end{tikzpicture}
}

\author{Thomas Cohn}
\title{Math 493 Lecture 1}
\date{9/4/2019} % Can also use \today

\begin{document}
\maketitle
\setlength\RaggedRightParindent{\parindent}
\RaggedRight

\defn{\!\!\!
	Let $S$ be a set. A \textbf{composition law} (or \textbf{binary operation}) on $S$ is a function $S\times{}S\overset{f}{\to}S$. We typically write $xy$, $x\cdot{}y$, $x+y$, $x\star{}y$, etc. instead of $f(x,y)$ ($f$ is implicit).
}

\ex{\!\!\!
	\begin{itemize}
		\item $S=\Z$, $x\cdot{}y=x+y$ (usual addition)
		\item $S=\Z$, $x\cdot{}y=xy$ (usual multiplication)
		\item $S=\R$, $x\cdot{}y=\frac{x+y}{2}$
		\item $S=\set{f:X\to{}X}$, $f\cdot{}g=f\of{}g$
		\item $S=M_{n}(\R)$, i.e., the set of $n\times{}n$ real matrices, with matrix addition or multiplication as the composition law.
	\end{itemize}
}

\par\noindent
This is very general, so it's not much to study.\n

\defn{\!\!\!
	A composition law is \textbf{associative} if $(x\cdot{}y)\cdot{}z=x\cdot(y\cdot{}z)$, $\forall{}x,y,z\in{}S$.
}

\par\noindent
All of the above examples (except the average one) are associative.\n

\par\noindent
If we have an associative composition law, and $x_{1},\ldots,x_{n}\in{}S$, we can make sense of $x_{1}\cdot{}x_{2}\cdot\ldots\cdot{}x_{n}$. We don't have to have parentheses.\n

\ex{\!\!\!
	$x_{1}\cdot{}x_{2}\cdot{}x_{3}\cdot{}x_{4}=x_{1}\cdot(x_{2}\cdot(x_{3}\cdot{}x_{4}))=(x_{1}\cdot{}x_{2})\cdot(x_{3}\cdot{}x_{4})=((x_{1}\cdot{}x_{2})\cdot{}x_{3})\cdot{}x_{4}$.
}

\defn{\!\!\!
	A composition law is \textbf{commutative} if $x\cdot{}y=y\cdot{}x$, $\forall{}x,y\in{}S$.
}

\defn{\!\!\!
	An element $e\in{}S$ is an \textbf{identity} for a composition law if $x\cdot{}e=e\cdot{}x=x$, $\forall{}x\in{}S$. $e$ is often denoted $1$ or $0$ (depending on context).
}

\par\noindent
All but the average example above have an identity. If an identity exists, it is unique -- assume $e$ and $e'$ are identity elements. Then $e=e\cdot{}e'=e'$.\n

\defn{\!\!\!
	Suppose we have an identity element $e\in{}S$, and our composition law is associative. Given $x\in{}S$, we say $y\in{}S$ is an \textbf{inverse} to $x$ if $x\cdot{}y=y\cdot{}x=e$. If such a $y$ exists, we say $x$ is \textbf{invertible}.
}

\par\noindent
The inverse to $x$ is unique if it exists. Assume $y$ and $y'$ are inverses of $x$. Then\n
$yxy'=y(xy')=ye=y$\n
$yxy'=(yx)y'=ey'=y'$\n
So $y=y'$.\n

\par\noindent
We'll denote the inverse of $x$ as $x\inv$ or $-x$ if it exists, depending on context.\n

\prop{\!\!\!
	Suppose $x$ and $y$ are both invertible. Then so is $xy$, and $(xy)\inv=y\inv{}x\inv$.\n
	Proof: $(xy)(y\inv{}x\inv)=x(yy\inv)x\inv=xex\inv=xx\inv=e$.\n
	And $(y\inv{}x\inv)xy=y\inv(x\inv{}x)y=y\inv{}ey=y\inv{}y=e$.\proven
}

\defn{\!\!\!
	A \textbf{group} is a pair $(G,\cdot)$ where $G$ is a set and $\cdot$ is a composition law on $G$ \st{}
	\begin{enumerate}
		\item $\cdot$ is associative.
		\item An identity element exists.
		\item All elements are invertible.
	\end{enumerate}
}

\defn{\!\!\!
	A commutative group is also called an \textbf{abelian group}.
}

\ex{\!\!\!
	\begin{itemize}
		\item $(\Z,+)$ is an abelian group.
		\item $(\Z,\cdot)$ is not a group.
		\item $(\Q\cut\set{0},\cdot)$ is an abelian group.
		\item $X$ set, $S=\set{f:X\to{}X|f\ptxt{ is a bijection}}$. $(S,\of)$ is a group.
		\item $\GL_{n}(\R)=\set{\ptxt{invertible matrices in }M_{n}(\R)}$ is a group under matrix multiplication.\footnote{$\GL_{n}$ is the \textbf{General Linear Group}.}
	\end{itemize}
}

\defn{\!\!\!
	Let $G$ be a group. A \textbf{subgroup} of $G$ is a subset $H\subset{}G$ \st{}
	\begin{enumerate}
		\item $H$ is closed under the composition law, i.e., $x,y\in{}H\Rightarrow{}xy\in{}H$.
		\item $H$ is closed under inverses, i.e., $x\in{}H\Rightarrow{}x\inv\in{}H$.
		\item $e\in{}H$ (or equivalently, $H$ is nonemepty).
	\end{enumerate}
}

\ex{\!\!\!
	$G=\Z$. Trivial subgroups $H=\Z$, $H=\set{0}$.\n
	$H=\set{\ptxt{even integers}}=2\Z\subseteq{}G$ is a subgroup.\n
	$H=m\Z=\set{\ptxt{all integers divisible by }m}$ is a subgroup.\n
	$H=\set{n\ge{}0|n\in\Z}$ is \textit{not} a subgroup.
}

\prop{\!\!\!
	Every subgroup of $\Z$ is of the form $m\Z$ for some $m\ge{}0$, and if $H\subseteq\Z$ subgroup, $\exists\unique{}m\ge{}0$ \st{} $H=m\Z$.\n
	Proof: Given $H\subseteq\Z$. If $H=\set{0}$, then $m=0$.\n
	Assume now that $H\ne\set{0}$. So $\exists{}n\ne{}0$ in $H$. Then either $n$ or $-n$ is positive, and both are in $H$. Let $m$ be the minimal positive integer in $H$.\n
	Claim: $H=m\Z$. Well, $m\in{}H$ by assumption, so $\forall{}k\ge{}0$, $km\in{}H$. With inverse, we have $m\Z\subseteq{}H$. Suppose we have $n>0\;\in{}H$. We can write $n=qm+r$, with $q,r\ge{}0$, $r<m$. Well, $n,qm\in{}H$, so $r=n+(-qm)\in{}H$. So $r=0$. Thus, $H\subseteq{}m\Z$, so $H=m\Z$.\n
	If $n<0$, $-n\in{}m\Z$, so $n\in{}m\Z$.\proven
}

\par\noindent
Observe: $H,K\subseteq\Z$ subgroups. $H+K=\set{x+y|x\in{}H,y\in{}K}$ is a subgroup.\n

\par\noindent
Let $n,m>0$. Then $n\Z+m\Z$ is a subgroup of $\Z$. By the definition of subgroups, $\exists\unique{}d>0$ \st{} $n\Z+m\Z=d\Z$. $d$ is in fact the GCD of $n$ and $m$.\n

\defn{\!\!\!
	Let $G$ be a group, $x\in{}G$. $H=\set{\ldots,x^{-2},x^{-1},x^{0}=e,x^{1},x^{2},\ldots}=\set{x^{n}|n\in\Z}$ is a subgroup of $G$. $H$ is the smallest subgroup of $G$ containing $x$, and it is called the subgroup of $G$ \textbf{generated} by $x$. A group that is generated by a single element is called \textbf{cyclic}.
}

\par\noindent
Consider $K=\set{n\in\Z|x^{n}=e}$.\n

\lemma{\!\!\!
	$K$ is a subgroup of $\Z$.\n
	Proof:
	\begin{enumerate}
		\item $n,m\in{}K\Rightarrow{}x^{n+m}=x^{n}\cdot{}x^{m}=e\cdot{}e=e\Rightarrow{}n+m\in{}K$.
		\item $n\in{}K\Rightarrow{}x^{-n}=(x^{n})^{-1}=e^{-1}=e\Rightarrow{}-n\in{}K$.
		\item $0\in{}K$ because $x^{0}=e$.
	\end{enumerate}
	\proven
}

\par\noindent
Note: $x^{n}=x^{m}$ if and only if $x^{n-m}=e$ if and only if $n-m\in{}K$.\n

\par\noindent
Two cases:
\begin{enumerate}
	\item $K=0$. Then $x^{n}=x^{m}$ if and only if $n=m$, so all pairs of $x$ are distinct, so $H$ is infinite.
	\item $K\ne{}0$. Then $k=d\Z$, for some $d>0$. $x^{n}=x^{m}$ if and only if $n-m\in{}d\Z$ if and only if\break $n\equiv{}m\pmod{d}$.
\end{enumerate}

\defn{\!\!\!
	$G$ is a group. The \textbf{order} of $G$, denoted $\card{G}$ or $\#G$, is the cardinality of $G$.
}

\defn{\!\!\!
	$G$ is a group, and $x\in{}G$. The \textbf{order} of $x$, denoted $\ord(x)$, is the order of the subgroup generated by $x$.
}

\par\noindent
$\ord(x)=\infty\Leftrightarrow\forall{}n\ne{}0,x^{n}\ne{}e$.\n
$\ord(x)=d\Leftrightarrow{}x^{d}=e$ and $d$ is minimal.\n

\ex{\!\!\!
	$G=\GL_{2}(\R)$, $x=\smallBMatrix{1 & 1\\ 0 & 1}$. So $x^{2}=\smallBMatrix{1 & 1\\ 0 & 1}\smallBMatrix{1 & 1\\ 0 & 1}=\smallBMatrix{1 & 2\\ 0 & 1}$, $x^{3}=xx^{2}=\smallBMatrix{1 & 1\\ 0 & 1}\smallBMatrix{1 & 2\\ 0 & 1}=\smallBMatrix{1 & 3\\ 0 & 1}$. $x^{n}=\smallBMatrix{1 & n\\ 0 & 1}$, $\forall{}n\in\Z$.\n
	$\generated{x}=\set{x^{n}|n\in\Z}=\set{\smallBMatrix{1 & n\\ 0 & 1}\middle|n\in\Z}$.\n
	$\ord(x)=\infty$.
}

\ex{\!\!\!
	$G=\GL_{3}(\R)$. $x=\smallBMatrix{0 & 0 & 1\\ 1 & 0 & 0\\ 0 & 1 & 0}$, $x^{2}=\smallBMatrix{0 & 1 & 0\\ 0 & 0 & 1\\ 1 & 0 & 0}$, $x^{3}=\smallBMatrix{1 & 0 & 0\\ 0 & 1 & 0\\ 0 & 0 & 1}$.\n
	$\ord(x)=3$.
}

\end{document}
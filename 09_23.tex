\documentclass[10pt,letterpaper]{article}
\usepackage[utf8]{inputenc}
\usepackage[intlimits]{amsmath}
\usepackage{amsfonts}
\usepackage{amssymb}
\usepackage{ragged2e}
\usepackage[letterpaper, margin=1in]{geometry}
\usepackage{graphicx}
\usepackage{cancel}
\usepackage{mathtools}
\usepackage{tabularx}
\usepackage{arydshln}
\usepackage{tensor}
\usepackage{array}
\usepackage{xcolor}
\usepackage[boxed]{algorithm}
\usepackage[noend]{algpseudocode}
\usepackage{listings}
\usepackage{textcomp}
\usepackage[pdf,tmpdir,singlefile]{graphviz}
\usepackage{mathrsfs}
\usepackage{bbm}
\usepackage{tikz}
\usepackage{tikz-cd}
\usepackage{enumitem}
\usepackage{arydshln}
\usepackage{relsize}
\usepackage{multicol}
\usepackage{scalerel}

\usetikzlibrary{bayesnet}
\setlist{noitemsep}

%%%%%%%%%%%%%%%%%%%%%%%%%%%%%
% Formatting commands
%%%%%%%%%%%%%%%%%%%%%%%%%%%%%
\newcommand{\n}{\hfill\break}
\newcommand{\up}{\vspace{-\baselineskip}}
\newcommand{\hangblock}[2]{\par\noindent\settowidth{\hangindent}{\textbf{#1: }}\textbf{#1: }\!\!\!#2}
\newcommand{\lemma}[1]{\hangblock{Lemma}{#1}}
\newcommand{\defn}[1]{\hangblock{Defn}{#1}}
\newcommand{\thm}[1]{\hangblock{Thm}{#1}}
\newcommand{\cor}[1]{\hangblock{Cor}{#1}}
\newcommand{\prop}[1]{\hangblock{Prop}{#1}}
\newcommand{\ex}[1]{\hangblock{Ex}{#1}}
\newcommand{\exer}[1]{\hangblock{Exer}{#1}}
\newcommand{\fact}[1]{\hangblock{Fact}{#1}}
\newcommand{\remark}[1]{\hangblock{Remark}{#1}}
\newcommand{\proven}{\;$\square$\n}
\newcommand{\problem}[1]{\par\noindent{#1}\n}
\newcommand{\problempart}[2]{\par\noindent\indent{}\settowidth{\hangindent}{\textbf{(#1)} \indent{}}\textbf{(#1)} #2\n}
\newcommand{\ptxt}[1]{\textrm{\textnormal{#1}}}
\newcommand{\inlineeq}[1]{\centerline{$\displaystyle #1$}}
\newcommand{\pageline}{\noindent\rule{\textwidth}{0.1pt}}

%%%%%%%%%%%%%%%%%%%%%%%%%%%%%
% Math commands
%%%%%%%%%%%%%%%%%%%%%%%%%%%%%
% Set Theory
\newcommand{\card}[1]{\left|#1\right|}
\newcommand{\set}[1]{\left\{#1\right\}}
\newcommand{\setmid}{\;\middle|\;}
\newcommand{\ps}[1]{\mathcal{P}\left(#1\right)}
\newcommand{\pfinite}[1]{\mathcal{P}^{\ptxt{finite}}\left(#1\right)}
\newcommand{\naturals}{\mathbb{N}}
\newcommand{\N}{\naturals}
\newcommand{\integers}{\mathbb{Z}}
\newcommand{\Z}{\integers}
\newcommand{\rationals}{\mathbb{Q}}
\newcommand{\Q}{\rationals}
\newcommand{\reals}{\mathbb{R}}
\newcommand{\R}{\reals}
\newcommand{\complex}{\mathbb{C}}
\newcommand{\C}{\complex}
\newcommand{\halfPlane}{\mathbb{H}}
\let\H\relax
\newcommand{\H}{\halfPlane}
\newcommand{\comp}{^{\complement}}
\DeclareMathOperator{\Hom}{Hom}
\newcommand{\Ind}{\mathbbm{1}}
\newcommand{\cut}{\setminus}
\DeclareMathOperator{\elem}{elem}

% Graph Theory
\let\deg\relax
\DeclareMathOperator{\deg}{deg}
\newcommand{\degp}{\ptxt{deg}^{+}}
\newcommand{\degn}{\ptxt{deg}^{-}}
\newcommand{\precdot}{\mathrel{\ooalign{$\prec$\cr\hidewidth\hbox{$\cdot\mkern0.5mu$}\cr}}}
\newcommand{\succdot}{\mathrel{\ooalign{$\cdot\mkern0.5mu$\cr\hidewidth\hbox{$\succ$}\cr\phantom{$\succ$}}}}
\DeclareMathOperator{\cl}{cl}
\DeclareMathOperator{\affdim}{affdim}

% Probability
\newcommand{\parSymbol}{\P}
\newcommand{\Prob}{\mathbb{P}}
\renewcommand{\P}{\Prob}
\newcommand{\Avg}{\mathbb{E}}
\newcommand{\E}{\Avg}
\DeclareMathOperator{\Var}{Var}
\DeclareMathOperator{\cov}{cov}
\DeclareMathOperator{\Unif}{Unif}
\DeclareMathOperator{\Binom}{Binom}
\newcommand{\CI}{\mathrel{\text{\scalebox{1.07}{$\perp\mkern-10mu\perp$}}}}

% Standard Math
\newcommand{\inv}{^{-1}}
\newcommand{\abs}[1]{\left|#1\right|}
\newcommand{\ceil}[1]{\left\lceil{}#1\right\rceil{}}
\newcommand{\floor}[1]{\left\lfloor{}#1\right\rfloor{}}
\newcommand{\conj}[1]{\overline{#1}}
\newcommand{\of}{\circ}
\newcommand{\tri}{\triangle}
\newcommand{\inj}{\hookrightarrow}
\newcommand{\surj}{\twoheadrightarrow}
\newcommand{\ndiv}{\nmid}
\renewcommand{\epsilon}{\varepsilon}
\newcommand{\divides}{\mid}
\newcommand{\ndivides}{\nmid}
\DeclareMathOperator{\lcm}{lcm}
\DeclareMathOperator{\sgn}{sgn}
\newcommand{\map}[4]{\!\!\!\begin{array}[t]{rcl}#1 & \!\!\!\!\to & \!\!\!\!#2\\ #3 & \!\!\!\!\mapsto & \!\!\!\!#4\end{array}}
\newcommand{\bigsum}[2]{\smashoperator[lr]{\sum_{\scalebox{#1}{$#2$}}}}

% Linear Algebra
\newcommand{\Id}{\textrm{\textnormal{Id}}}
\newcommand{\im}{\textrm{\textnormal{im}}}
\newcommand{\norm}[1]{\abs{\abs{#1}}}
\newcommand{\tpose}{^{T}\!}
\newcommand{\iprod}[1]{\left<#1\right>}
\DeclareMathOperator{\tr}{tr}
\DeclareMathOperator{\trace}{tr}
\newcommand{\chgBasMat}[3]{\!\!\tensor*[_{#1}]{\left[#2\right]}{_{#3}}}
\newcommand{\vecBas}[2]{\tensor*[]{\left[#1\right]}{_{#2}}}
\DeclareMathOperator{\GL}{GL}
\DeclareMathOperator{\Mat}{Mat}
\DeclareMathOperator{\vspan}{span}
\DeclareMathOperator{\rank}{rank}
\newcommand{\V}[1]{\vec{#1}}
\DeclareMathOperator{\proj}{proj}
\DeclareMathOperator{\compProj}{comp}
\DeclareMathOperator{\row}{row}
\newcommand{\smallPMatrix}[1]{\paren{\begin{smallmatrix}#1\end{smallmatrix}}}
\newcommand{\smallBMatrix}[1]{\brack{\begin{smallmatrix}#1\end{smallmatrix}}}

% Multilinear Algebra
\newcommand{\Lsym}{\L}
\let\L\relax
\DeclareMathOperator{\L}{\mathscr{L}}
\DeclareMathOperator{\A}{\mathcal{A}}
\DeclareMathOperator{\Alt}{Alt}
\DeclareMathOperator{\Sym}{Sym}
\newcommand{\ot}{\otimes}
\newcommand{\ox}{\otimes}
\DeclareMathOperator{\asc}{asc}
\DeclareMathOperator{\asSet}{set}
\DeclareMathOperator{\sort}{sort}
\DeclareMathOperator{\ringA}{\mathring{A}}

% Topology
\newcommand{\closure}[1]{\overline{#1}}
\newcommand{\uball}{\mathcal{U}}
\DeclareMathOperator{\Int}{Int}
\DeclareMathOperator{\Ext}{Ext}
\DeclareMathOperator{\Bd}{Bd}
\DeclareMathOperator{\rInt}{rInt}
\DeclareMathOperator{\ch}{ch}
\DeclareMathOperator{\ah}{ah}
\newcommand{\LargerTau}{\mathlarger{\mathlarger{\mathlarger{\mathlarger{\tau}}}}}
\newcommand{\Tau}{\mathcal{T}}

% Analysis
\DeclareMathOperator{\Graph}{Graph}
\DeclareMathOperator{\epi}{epi}
\DeclareMathOperator{\hypo}{hypo}
\DeclareMathOperator{\supp}{supp}
\newcommand{\lint}[2]{\underset{#1}{\overset{#2}{{\color{black}\underline{{\color{white}\overline{{\color{black}\int}}\color{black}}}}}}}
\newcommand{\uint}[2]{\underset{#1}{\overset{#2}{{\color{white}\underline{{\color{black}\overline{{\color{black}\int}}\color{black}}}}}}}
\newcommand{\alignint}[2]{\underset{#1}{\overset{#2}{{\color{white}\underline{{\color{white}\overline{{\color{black}\int}}\color{black}}}}}}}
\newcommand{\extint}{\ptxt{ext}\int}
\newcommand{\extalignint}[2]{\ptxt{ext}\alignint{#1}{#2}}
\newcommand{\conv}{\ast}
\newcommand{\pd}[2]{\frac{\partial{}#1}{\partial{}#2}}
\newcommand{\del}{\nabla}
\DeclareMathOperator{\grad}{grad}
\DeclareMathOperator{\curl}{curl}
\let\div\relax
\DeclareMathOperator{\div}{div}
\DeclareMathOperator{\vol}{vol}

% Complex Analysis
\let\Re\relax
\DeclareMathOperator{\Re}{Re}
\let\Im\relax
\DeclareMathOperator{\Im}{Im}
\DeclareMathOperator{\Res}{Res}

% Abstract Algebra
\DeclareMathOperator{\ord}{ord}
\newcommand{\generated}[1]{\left<#1\right>}
\newcommand{\cycle}[1]{\smallPMatrix{#1}}
\newcommand{\id}{\ptxt{id}}
\newcommand{\iso}{\cong}
\DeclareMathOperator{\Aut}{Aut}
\DeclareMathOperator{\SL}{SL}
\DeclareMathOperator{\op}{op}
\newcommand{\isom}[4]{\!\!\!\begin{array}[t]{rcl}#1 & \!\!\!\!\overset{\sim}{\to} & \!\!\!\!#2\\ #3 & \!\!\!\!\mapsto & \!\!\!\!#4\end{array}}
\newcommand{\F}{\mathbb{F}}

% Convex Optimization
\newcommand{\sectionSymbol}{\S}
\let\S\relax
\newcommand{\S}{\mathbb{S}}
\DeclareMathOperator{\dist}{dist}
\DeclareMathOperator{\dom}{dom}
\DeclareMathOperator{\diag}{diag}
\DeclareMathOperator{\ones}{\mathbbm{1}}

% Proofs
\newcommand{\st}{s.t.}
\newcommand{\unique}{!}
\newcommand{\iffdef}{\overset{\ptxt{def}}{\Leftrightarrow}}
\newcommand{\eqdef}{\overset{\ptxt{def}}{=}}
\newcommand{\eqVertical}{\rotatebox[origin=c]{90}{=}}
\newcommand{\mapsfrom}{\mathrel{\reflectbox{\ensuremath{\mapsto}}}}
\newcommand{\mapsdown}{\rotatebox[origin=c]{-90}{$\mapsto$}\mkern2mu}
\newcommand{\mapsup}{\rotatebox[origin=c]{90}{$\mapsto$}\mkern2mu}
\newcommand{\from}{\!\mathrel{\reflectbox{\ensuremath{\to}}}}

% Brackets
\newcommand{\paren}[1]{\left(#1\right)}
\renewcommand{\brack}[1]{\left[#1\right]}
\renewcommand{\brace}[1]{\left\{#1\right\}}
\newcommand{\ang}[1]{\left<#1\right>}

% Algorithms
\algrenewcommand{\algorithmiccomment}[1]{\hskip 1em \texttt{// #1}}
\algrenewcommand\algorithmicrequire{\textbf{Input:}}
\algrenewcommand\algorithmicensure{\textbf{Output:}}
\newcommand{\algP}{\ptxt{\textbf{P}}}
\newcommand{\algNP}{\ptxt{\textbf{NP}}}
\newcommand{\algNPC}{\ptxt{\textbf{NP-Complete}}}
\newcommand{\algNPH}{\ptxt{\textbf{NP-Hard}}}
\newcommand{\algEXP}{\ptxt{\textbf{EXP}}}

%%%%%%%%%%%%%%%%%%%%%%%%%%%%%
% Other commands
%%%%%%%%%%%%%%%%%%%%%%%%%%%%%
\newcommand{\flag}[1]{\textbf{\textcolor{red}{#1}}}
\newcommand{\uSym}{\u}
\let\u\relax
\newcommand{\u}[1]{\underline{#1}}
\newcommand{\bSym}{\b}
\let\b\relax
\newcommand{\b}[1]{\textbf{#1}}
\newcommand{\iSym}{\i}
\let\i\relax
\newcommand{\i}[1]{\textit{#1}}

%%%%%%%%%%%%%%%%%%%%%%%%%%%%%%%%%%%%%%%
% Make l's curvy in math environments %
%%%%%%%%%%%%%%%%%%%%%%%%%%%%%%%%%%%%%%%
\mathcode`l="8000
\begingroup
\makeatletter
\lccode`\~=`\l
\DeclareMathSymbol{\lsb@l}{\mathalpha}{letters}{`l}
\lowercase{\gdef~{\ifnum\the\mathgroup=\m@ne \ell \else \lsb@l \fi}}%
\endgroup

%%%%%%%%%%%%%%%%%%%%%%%%%
% Fix \vdots and \ddots %
%%%%%%%%%%%%%%%%%%%%%%%%%
\usepackage{letltxmacro}
\LetLtxMacro\orgvdots\vdots
\LetLtxMacro\orgddots\ddots

\makeatletter
\DeclareRobustCommand\vdots{%
	\mathpalette\@vdots{}%
}
\newcommand*{\@vdots}[2]{%
	% #1: math style
	% #2: unused
	\sbox0{$#1\cdotp\cdotp\cdotp\m@th$}%
	\sbox2{$#1.\m@th$}%
	\vbox{%
		\dimen@=\wd0 %
		\advance\dimen@ -3\ht2 %
		\kern.5\dimen@
		% remove side bearings
		\dimen@=\wd2 %
		\advance\dimen@ -\ht2 %
		\dimen2=\wd0 %
		\advance\dimen2 -\dimen@
		\vbox to \dimen2{%
			\offinterlineskip
			\copy2 \vfill\copy2 \vfill\copy2 %
		}%
	}%
}
\DeclareRobustCommand\ddots{%
	\mathinner{%
		\mathpalette\@ddots{}%
		\mkern\thinmuskip
	}%
}
\newcommand*{\@ddots}[2]{%
	% #1: math style
	% #2: unused
	\sbox0{$#1\cdotp\cdotp\cdotp\m@th$}%
	\sbox2{$#1.\m@th$}%
	\vbox{%
		\dimen@=\wd0 %
		\advance\dimen@ -3\ht2 %
		\kern.5\dimen@
		% remove side bearings
		\dimen@=\wd2 %
		\advance\dimen@ -\ht2 %
		\dimen2=\wd0 %
		\advance\dimen2 -\dimen@
		\vbox to \dimen2{%
			\offinterlineskip
			\hbox{$#1\mathpunct{.}\m@th$}%
			\vfill
			\hbox{$#1\mathpunct{\kern\wd2}\mathpunct{.}\m@th$}%
			\vfill
			\hbox{$#1\mathpunct{\kern\wd2}\mathpunct{\kern\wd2}\mathpunct{.}\m@th$}%
		}%
	}%
}
\makeatother

\newcommand{\B}{
	\begin{tikzpicture}
	\filldraw [fill=red, draw=black] (0, 0) rectangle (0.37, 0.45);
	\draw [line width=0.5mm, white ] (0.1,0.08) -- (0.1,0.38);
	\draw[line width=0.5mm, white ] (0.1, 0.35) .. controls (0.2, 0.35) and (0.4, 0.2625) .. (0.1, 0.225);
	\draw[line width=0.5mm, white ] (0.1, 0.225) .. controls (0.2, 0.225) and (0.4, 0.1625) .. (0.1, 0.1);
	\end{tikzpicture}
}

\author{Thomas Cohn}
\title{Math 493 Lecture 6}
\date{9/23/19} % Can also use \today

\begin{document}
\maketitle
\setlength\RaggedRightParindent{\parindent}
\RaggedRight

\par\noindent
Continuing from last time...\n

\par\noindent
$(1)\Rightarrow(3)$: $S$ is a basis. We need to show no superset of $S$ is linearly independent. Say $x\in{}V\cut{}S$. $S$ spans, so $X=\sum_{i=1}^{n}a_{i}v_{i}$ for some $a_{i}\in{}K$, $v_{i}\in{}S$. So $x-a_{1}v_{1}-a_{2}v_{2}-\cdots-a_{n}v_{n}=0$. Thus, $S\cup\set{x}$ is not linearly independent.\n

\par\noindent
$(3)\Rightarrow(1)$: $S$ is a maximal linearly independent set. Suppose $S$ did not span. Let $x\not\in\vspan(S)$. Then we claim $S\cup\set{x}$ is linearly independent. Suppose we have $bx+a_{1}v_{1}+\cdots+a_{n}v_{n}=0$, with $a_{i},b\in{}K$, $v_{i}\in{}S$. If $b=0$, then we have a linear relation on the elements of $S$, so $a_{i}=0$, because $S$ is linearly independent. If $b\ne{}0$, then $x=\frac{-1}{b}(a_{1}v_{1}+\cdots+a_{n}v_{n})$, so $x\in\vspan(S)$. Oops!\n

\par\noindent
\proven

\prop{
	$V$ is a vector space, $B\subseteq{}V$. Then any maximally independent subset of $B$ spans $B$.\n
	Proof: Let $S\subset{}B$ be a maximal linearly independent set. Suppose $x\in{}B$ \st{} $x\not\in\vspan(S)$. Then $S\cup\set{x}$ is linearly independent. This contradicts $S$ being a maximal linearly independent subset of $B$.\proven
}

\lemma{
	(Zorn's Lemma) Suppose $S$ is a partially ordered set with binary relation $\succeq$ \st{}
	\begin{enumerate}
		\item $\succeq$ is reflexive: $x\succeq{}x,\forall{}x\in{}S$.
		\item $\succeq$ is antisymmetric: $x\succeq{}y,y\succeq{}x\Rightarrow{}x=y,\forall{}x,y\in{}S$.
		\item $\succeq$ is transitive: $x\succeq{}y,y\succeq{}z\Rightarrow{}x\succeq{}z,\forall{}x,y,z\in{}S$.
	\end{enumerate}
	Assume $S\ne\emptyset$ and every chain is bounded, i.e., given $x_{1}\succeq{}x_{2}\succeq\cdots\in{}S$, $\exists{}y\in{}S$ \st{} $x_{i}\succeq{}y,\forall{}i$, then there exists a maximal element $z\in{}X$, i.e., $z\succeq{}x\Rightarrow{}z=x$.\n
}

\cor{
	Let $A\subseteq{}B\subseteq{}V$, with $A$ linearly independent and $B$ spanning. Then there is a basis $S$ contained in $B$ which contains $A$.\n
	Proof: define $X$ to be the set of all independent subsets of $B$ containing $A$.
	\begin{itemize}
		\item $X$ is partially ordered by inclusion.
		\item $X$ is nonempty because $A\in{}X$.
	\end{itemize}\up\n
	We will apply Zorn's Lemma, by showing all chains are bounded. Let $C_{1}\subseteq{}C_{2}\subseteq\cdots\in{}X$. We need to prove $\exists{}C\in{}C$ \st{} $C_{i}\subseteq{}C$, $\forall{}i$.\n
	Take $C=\bigcup_{i\ge{}1}C_{i}\subseteq{}B$. We need to show $C$ is linearly independent. Suppose $a_{n}x_{1}+\cdots+a_{n}x_{n}$ is a linear relation with $x_{i}\in{}C$. Then $\exists{}m$ \st{} $x_{1},\ldots,x_{n}\in{}C_{m}$. So the relation is trivial, because $C_{m}\in{}X$, so $C$ bounds the independent sets.\n
	Thus, by Zorn's Lemma, there is a maximal element of $X$.\proven
}

\cor{
	If $A\subseteq{}V$ is any independent set, there is a basis containing $A$.\n
	Proof: take $B=V$.\proven
}

\cor{
	If $B\subseteq{}V$ is any spanning set, there is a basis contained in $B$.\n
	Proof: Take $A=\emptyset$.\proven
}

\prop{
	(Basis Exchange Lemma) $V$ vector space, $B$ basis of $V$, $C$ spanning set of $V$. Given $x\in{}B$, $\exists{}y\in{}C$ \st{} $(B\cut\set{x})\cup\set{y}$ is a basis.\n
	Proof: $B\cut\set{x}$ is not maximally independent, but is independent. So it must not span.\n
	So $C\not\subset\vspan(B\cut\set{x})$ (if we did have containment, then we'd have $\vspan(C)\subseteq\vspan(B\cut\set{x})$). So $\exists{}y\in{}C$ \st{} $y\not\in\vspan(B\cut\set{x})$. We need to show $B'=(B\cut\set{x})\cup\set{y}$ is a basis, because $y\in\vspan(B)$, $y=a_{1}z_{1}+\cdots+a_{n}z_{n}$ with $z_{i}\in{}B$.\n
	\n
	Must have $z_{i}=x$, $a_{i}\ne{}0$, for some $i$. Say $i=1$ WOLOG (otherwise, $y\in\vspan(B\cut\set{x})$). $y=a_{1}x+a_{2}z_{2}+\cdots+a_{n}z_{n}$. $x\in\vspan(B')$, because $x=\frac{1}{a}(y-a_{1}z_{1}-a_{2}z_{2}-\cdots-a_{n}z_{n})\in\vspan(B')$.\n
	\n
	$B\cut\set{x}\subseteq\vspan(B')\Rightarrow{}B\subseteq\vspan(B)'$. So $V=\vspan(B)\subseteq\vspan(B')$, and thus, $B'$ spans $V$.\n
	\n
	Claim: $B'$ is independent. Consider a linear relation $cy+d_{1}v_{1}+\cdots+d_{m}v_{m}=0$, with $c,d_{i}\in{}K$, $v_{1},\ldots,v_{m}\in{}B\cut\set{x}$.
	\begin{enumerate}[label=Case{\arabic*}: ,leftmargin=5\parindent]
		\item $c=0$. Then this is a relation between elements of $B\cut\set{x}$, which we know to be independent. So each $d_{i}=0$.
		\item $c\ne{}0$. Then $c(a_{1}x+a_{2}z_{2}+\cdots+a_{n}z_{n})+d_{1}v_{1}+\cdots=d_{m}v_{m}=0$. This is a linear relation between elements of $B$. So the coefficients must all be $0$.
	\end{enumerate}\up\n
	Therefore, $B'$ is a basis.\proven
}

\defn{
	A vector space $V$ is \textbf{finite dimensional} if it has a finite spanning set.\n
}

\thm{
	(Dimension Theorem) Suppose $V$ is a finite dimensional vector space. Then all bases are finite and have the same size.\n
	Proof: Note that, by definition, $V$ has a finite spanning set. This contains a basis, so there exists a finite basis.\n
	Say $B=\set{x_{1},\ldots,x_{n}}$ is a basis. $C$ is some other basis. Define $B_{0}=B$. Given $B_{k}$, define $B_{k+1}$ by replacing $x_{k}\in{}B_{k}$ with something from $C$, while maintaining it as a basis.\n
	Thus, $B_{k}\subseteq{}C$, so $C=B_{k}$. We have $\#C=\#B_{k}=\#B_{0}=\#B$.\proven
}

\defn{
	Suppose $V$ is finite dimensional. Then we say the \textbf{dimension} of $V$, denoted $\dim{}V$, to be the size of a basis.\n
}

\ex{
	$V=K^{n}$. For $1\le{}i\le{}n$, let $e_{i}$ be the vector of zeros, with $1$ in the $i$th position. $\set{e_{1},\ldots,e_{n}}$ forms a basis of $V$, so $\dim{}V=n$.\n
}

\ex{
	$V$ is the set of all polynomials in $X$ with coefficients in $K$. This is not finite dimensional.\n
}

\ex{
	Let $V$ be the set of all polynomials of degree $\le{}d$. Then $\set{1,x,x^{2},\ldots,x^{n}}$ is a basis, so $\dim{}V=n+1$.\n
}

\par\noindent
Let $V$ be a vector space, and let $W$ be a subspace. We can then form $V/W$, the quotient space.
\begin{itemize}
	\item As an abelian group, it's the usual quotient group.
	\item Given $a\in{}K$, $v\in{}V$, define $a(v\cdot{}W)=av+W$. Check that this is well defined:\n
	Suppose $v+W=v'+W$. Then $v-v'\in{}W$, so $a(v-v')=av-av'\in{}W$. Thus, $av+W=av'+W$.
\end{itemize}

\par\noindent
We have the quotient map $\map{\pi:V}{V/W}{v}{\bar{v}=v+W}$.\n
This is a linear map, because it is a group homomorphism and compatible with scalar multiplication.\n

\prop{
	Let $V$ be a finite dimensional vector space, and $W\subseteq{}V$ a subspace. Then $W$ and $V/W$ are finite dimensional, and $\dim{}V=\dim{}W+\dim{}V/W$.\n
	Proof: Let $S$ be a basis for $W$. We can extend $S$ to be a basis $T$ of $V$. Thus, $T$ (and $S$) are finite. $S=\set{x_{1},\ldots,x_{n}}$. $T=\set{x_{1},\ldots,x_{n},y_{1},\ldots,y_{m}}$. So $\dim{}W=n$, $\dim{}V=n+m$.\n
	\n
	We claim $\bar{y}_{1},\ldots,\bar{y}_{m}$ form a basis of $V/W$. Let $\bar{v}\in{}V/W$. We can write\n
	$v=a_{1}x_{1}+\cdots+a_{n}x_{n}+b_{1}y_{1}+\cdots+b_{m}y_{m}$. Apply $\pi$, so $\bar{v}=b_{1}\bar{y}_{1}+\cdots+b_{m}\bar{y}_{m}$. Thus, $\bar{y}_{1},\ldots,\bar{y}_{m}$ spans.\n
	Say $b_{1}\bar{y}_{1}+\cdots+b_{m}\bar{y}_{m}=0$. Consider $b_{1}y_{1}+\cdots+b_{m}y_{m}\in\ker(\pi)=W$. This is equal to $a_{1}x_{1}+\cdots+a_{n}x_{n}$ for some $a_{1},\ldots,a_{n}$. So $-a_{1}x_{1}-\cdots-a_{n}x_{n}+b_{1}y_{1}+\cdots+b_{m}y_{m}=0$. So all coefficients are $0$, so $\bar{y}_{1},\ldots,\bar{y}_{m}$ are linearly independent.\n
	Therefore, they're a basis.\proven
}

\end{document}
\documentclass[10pt,letterpaper]{article}
\usepackage[utf8]{inputenc}
\usepackage[intlimits]{amsmath}
\usepackage{amsfonts}
\usepackage{amssymb}
\usepackage{ragged2e}
\usepackage[letterpaper, margin=1in]{geometry}
\usepackage{graphicx}
\usepackage{cancel}
\usepackage{mathtools}
\usepackage{tabularx}
\usepackage{arydshln}
\usepackage{tensor}
\usepackage{array}
\usepackage{xcolor}
\usepackage[boxed]{algorithm}
\usepackage[noend]{algpseudocode}
\usepackage{listings}
\usepackage{textcomp}
\usepackage[pdf,tmpdir,singlefile]{graphviz}
\usepackage{mathrsfs}
\usepackage{bbm}
\usepackage{tikz}
\usepackage{tikz-cd}
\usepackage{enumitem}
\usepackage{arydshln}
\usepackage{relsize}
\usepackage{multicol}
\usepackage{scalerel}

\usetikzlibrary{bayesnet}

%%%%%%%%%%%%%%%%%%%%%%%%%%%%%
% Formatting commands
%%%%%%%%%%%%%%%%%%%%%%%%%%%%%
\newcommand{\n}{\hfill\break}
\newcommand{\up}{\vspace{-\baselineskip}}
\newcommand{\hangblock}[2]{\par\noindent\settowidth{\hangindent}{\textbf{#1: }}\textbf{#1: }\!\!\!#2}
\newcommand{\lemma}[1]{\hangblock{Lemma}{#1}}
\newcommand{\defn}[1]{\hangblock{Defn}{#1}}
\newcommand{\thm}[1]{\hangblock{Thm}{#1}}
\newcommand{\cor}[1]{\hangblock{Cor}{#1}}
\newcommand{\prop}[1]{\hangblock{Prop}{#1}}
\newcommand{\ex}[1]{\hangblock{Ex}{#1}}
\newcommand{\exer}[1]{\hangblock{Exer}{#1}}
\newcommand{\fact}[1]{\hangblock{Fact}{#1}}
\newcommand{\remark}[1]{\hangblock{Remark}{#1}}
\newcommand{\proven}{\;$\square$\n}
\newcommand{\problem}[1]{\par\noindent{#1}\n}
\newcommand{\problempart}[2]{\par\noindent\indent{}\settowidth{\hangindent}{\textbf{(#1)} \indent{}}\textbf{(#1)} #2\n}
\newcommand{\ptxt}[1]{\textrm{\textnormal{#1}}}
\newcommand{\inlineeq}[1]{\centerline{$\displaystyle #1$}}
\newcommand{\pageline}{\noindent\rule{\textwidth}{0.1pt}}

%%%%%%%%%%%%%%%%%%%%%%%%%%%%%
% Math commands
%%%%%%%%%%%%%%%%%%%%%%%%%%%%%
% Set Theory
\newcommand{\card}[1]{\left|#1\right|}
\newcommand{\set}[1]{\left\{#1\right\}}
\newcommand{\setmid}{\;\middle|\;}
\newcommand{\ps}[1]{\mathcal{P}\left(#1\right)}
\newcommand{\pfinite}[1]{\mathcal{P}^{\ptxt{finite}}\left(#1\right)}
\newcommand{\naturals}{\mathbb{N}}
\newcommand{\N}{\naturals}
\newcommand{\integers}{\mathbb{Z}}
\newcommand{\Z}{\integers}
\newcommand{\rationals}{\mathbb{Q}}
\newcommand{\Q}{\rationals}
\newcommand{\reals}{\mathbb{R}}
\newcommand{\R}{\reals}
\newcommand{\complex}{\mathbb{C}}
\newcommand{\C}{\complex}
\newcommand{\halfPlane}{\mathbb{H}}
\let\H\relax
\newcommand{\H}{\halfPlane}
\newcommand{\comp}{^{\complement}}
\DeclareMathOperator{\Hom}{Hom}
\newcommand{\Ind}{\mathbbm{1}}
\newcommand{\cut}{\setminus}
\DeclareMathOperator{\elem}{elem}

% Graph Theory
\let\deg\relax
\DeclareMathOperator{\deg}{deg}
\newcommand{\degp}{\ptxt{deg}^{+}}
\newcommand{\degn}{\ptxt{deg}^{-}}
\newcommand{\precdot}{\mathrel{\ooalign{$\prec$\cr\hidewidth\hbox{$\cdot\mkern0.5mu$}\cr}}}
\newcommand{\succdot}{\mathrel{\ooalign{$\cdot\mkern0.5mu$\cr\hidewidth\hbox{$\succ$}\cr\phantom{$\succ$}}}}
\DeclareMathOperator{\cl}{cl}
\DeclareMathOperator{\affdim}{affdim}

% Probability
\newcommand{\parSymbol}{\P}
\newcommand{\Prob}{\mathbb{P}}
\renewcommand{\P}{\Prob}
\newcommand{\Avg}{\mathbb{E}}
\newcommand{\E}{\Avg}
\DeclareMathOperator{\Var}{Var}
\DeclareMathOperator{\cov}{cov}
\DeclareMathOperator{\Unif}{Unif}
\DeclareMathOperator{\Binom}{Binom}
\newcommand{\CI}{\mathrel{\text{\scalebox{1.07}{$\perp\mkern-10mu\perp$}}}}

% Standard Math
\newcommand{\inv}{^{-1}}
\newcommand{\abs}[1]{\left|#1\right|}
\newcommand{\ceil}[1]{\left\lceil{}#1\right\rceil{}}
\newcommand{\floor}[1]{\left\lfloor{}#1\right\rfloor{}}
\newcommand{\conj}[1]{\overline{#1}}
\newcommand{\of}{\circ}
\newcommand{\tri}{\triangle}
\newcommand{\inj}{\hookrightarrow}
\newcommand{\surj}{\twoheadrightarrow}
\newcommand{\ndiv}{\nmid}
\renewcommand{\epsilon}{\varepsilon}
\newcommand{\divides}{\mid}
\newcommand{\ndivides}{\nmid}
\DeclareMathOperator{\lcm}{lcm}
\DeclareMathOperator{\sgn}{sgn}
\newcommand{\map}[4]{\!\!\!\begin{array}[t]{rcl}#1 & \!\!\!\!\to & \!\!\!\!#2\\ #3 & \!\!\!\!\mapsto & \!\!\!\!#4\end{array}}
\newcommand{\bigsum}[2]{\smashoperator[lr]{\sum_{\scalebox{#1}{$#2$}}}}

% Linear Algebra
\newcommand{\Id}{\textrm{\textnormal{Id}}}
\newcommand{\im}{\textrm{\textnormal{im}}}
\newcommand{\norm}[1]{\abs{\abs{#1}}}
\newcommand{\tpose}{^{T}\!}
\newcommand{\iprod}[1]{\left<#1\right>}
\DeclareMathOperator{\tr}{tr}
\DeclareMathOperator{\trace}{tr}
\newcommand{\chgBasMat}[3]{\!\!\tensor*[_{#1}]{\left[#2\right]}{_{#3}}}
\newcommand{\vecBas}[2]{\tensor*[]{\left[#1\right]}{_{#2}}}
\DeclareMathOperator{\GL}{GL}
\DeclareMathOperator{\Mat}{Mat}
\DeclareMathOperator{\vspan}{span}
\DeclareMathOperator{\rank}{rank}
\newcommand{\V}[1]{\vec{#1}}
\DeclareMathOperator{\proj}{proj}
\DeclareMathOperator{\compProj}{comp}
\DeclareMathOperator{\row}{row}
\newcommand{\smallPMatrix}[1]{\paren{\begin{smallmatrix}#1\end{smallmatrix}}}
\newcommand{\smallBMatrix}[1]{\brack{\begin{smallmatrix}#1\end{smallmatrix}}}

% Multilinear Algebra
\newcommand{\Lsym}{\L}
\let\L\relax
\DeclareMathOperator{\L}{\mathscr{L}}
\DeclareMathOperator{\A}{\mathcal{A}}
\DeclareMathOperator{\Alt}{Alt}
\DeclareMathOperator{\Sym}{Sym}
\newcommand{\ot}{\otimes}
\newcommand{\ox}{\otimes}
\DeclareMathOperator{\asc}{asc}
\DeclareMathOperator{\asSet}{set}
\DeclareMathOperator{\sort}{sort}
\DeclareMathOperator{\ringA}{\mathring{A}}

% Topology
\newcommand{\closure}[1]{\overline{#1}}
\newcommand{\uball}{\mathcal{U}}
\DeclareMathOperator{\Int}{Int}
\DeclareMathOperator{\Ext}{Ext}
\DeclareMathOperator{\Bd}{Bd}
\DeclareMathOperator{\rInt}{rInt}
\DeclareMathOperator{\ch}{ch}
\DeclareMathOperator{\ah}{ah}
\newcommand{\LargerTau}{\mathlarger{\mathlarger{\mathlarger{\mathlarger{\tau}}}}}
\newcommand{\Tau}{\mathcal{T}}

% Analysis
\DeclareMathOperator{\Graph}{Graph}
\DeclareMathOperator{\epi}{epi}
\DeclareMathOperator{\hypo}{hypo}
\DeclareMathOperator{\supp}{supp}
\newcommand{\lint}[2]{\underset{#1}{\overset{#2}{{\color{black}\underline{{\color{white}\overline{{\color{black}\int}}\color{black}}}}}}}
\newcommand{\uint}[2]{\underset{#1}{\overset{#2}{{\color{white}\underline{{\color{black}\overline{{\color{black}\int}}\color{black}}}}}}}
\newcommand{\alignint}[2]{\underset{#1}{\overset{#2}{{\color{white}\underline{{\color{white}\overline{{\color{black}\int}}\color{black}}}}}}}
\newcommand{\extint}{\ptxt{ext}\int}
\newcommand{\extalignint}[2]{\ptxt{ext}\alignint{#1}{#2}}
\newcommand{\conv}{\ast}
\newcommand{\pd}[2]{\frac{\partial{}#1}{\partial{}#2}}
\newcommand{\del}{\nabla}
\DeclareMathOperator{\grad}{grad}
\DeclareMathOperator{\curl}{curl}
\let\div\relax
\DeclareMathOperator{\div}{div}
\DeclareMathOperator{\vol}{vol}

% Complex Analysis
\let\Re\relax
\DeclareMathOperator{\Re}{Re}
\let\Im\relax
\DeclareMathOperator{\Im}{Im}
\DeclareMathOperator{\Res}{Res}

% Abstract Algebra
\DeclareMathOperator{\ord}{ord}
\newcommand{\generated}[1]{\left<#1\right>}
\newcommand{\cycle}[1]{\smallPMatrix{#1}}
\newcommand{\id}{\ptxt{id}}
\newcommand{\iso}{\cong}
\DeclareMathOperator{\Aut}{Aut}
\DeclareMathOperator{\SL}{SL}
\DeclareMathOperator{\op}{op}
\newcommand{\isom}[4]{\!\!\!\begin{array}[t]{rcl}#1 & \!\!\!\!\overset{\sim}{\to} & \!\!\!\!#2\\ #3 & \!\!\!\!\mapsto & \!\!\!\!#4\end{array}}
\newcommand{\F}{\mathbb{F}}

% Convex Optimization
\newcommand{\sectionSymbol}{\S}
\let\S\relax
\newcommand{\S}{\mathbb{S}}
\DeclareMathOperator{\dist}{dist}
\DeclareMathOperator{\dom}{dom}
\DeclareMathOperator{\diag}{diag}
\DeclareMathOperator{\ones}{\mathbbm{1}}

% Proofs
\newcommand{\st}{s.t.}
\newcommand{\unique}{!}
\newcommand{\iffdef}{\overset{\ptxt{def}}{\Leftrightarrow}}
\newcommand{\eqdef}{\overset{\ptxt{def}}{=}}
\newcommand{\eqVertical}{\rotatebox[origin=c]{90}{=}}
\newcommand{\mapsfrom}{\mathrel{\reflectbox{\ensuremath{\mapsto}}}}
\newcommand{\mapsdown}{\rotatebox[origin=c]{-90}{$\mapsto$}\mkern2mu}
\newcommand{\mapsup}{\rotatebox[origin=c]{90}{$\mapsto$}\mkern2mu}
\newcommand{\from}{\!\mathrel{\reflectbox{\ensuremath{\to}}}}

% Brackets
\newcommand{\paren}[1]{\left(#1\right)}
\renewcommand{\brack}[1]{\left[#1\right]}
\renewcommand{\brace}[1]{\left\{#1\right\}}
\newcommand{\ang}[1]{\left<#1\right>}

% Algorithms
\algrenewcommand{\algorithmiccomment}[1]{\hskip 1em \texttt{// #1}}
\algrenewcommand\algorithmicrequire{\textbf{Input:}}
\algrenewcommand\algorithmicensure{\textbf{Output:}}
\newcommand{\algP}{\ptxt{\textbf{P}}}
\newcommand{\algNP}{\ptxt{\textbf{NP}}}
\newcommand{\algNPC}{\ptxt{\textbf{NP-Complete}}}
\newcommand{\algNPH}{\ptxt{\textbf{NP-Hard}}}
\newcommand{\algEXP}{\ptxt{\textbf{EXP}}}

%%%%%%%%%%%%%%%%%%%%%%%%%%%%%
% Other commands
%%%%%%%%%%%%%%%%%%%%%%%%%%%%%
\newcommand{\flag}[1]{\textbf{\textcolor{red}{#1}}}
\newcommand{\uSym}{\u}
\let\u\relax
\newcommand{\u}[1]{\underline{#1}}
\newcommand{\bSym}{\b}
\let\b\relax
\newcommand{\b}[1]{\textbf{#1}}
\newcommand{\iSym}{\i}
\let\i\relax
\newcommand{\i}[1]{\textit{#1}}

%%%%%%%%%%%%%%%%%%%%%%%%%%%%%%%%%%%%%%%
% Make l's curvy in math environments %
%%%%%%%%%%%%%%%%%%%%%%%%%%%%%%%%%%%%%%%
\mathcode`l="8000
\begingroup
\makeatletter
\lccode`\~=`\l
\DeclareMathSymbol{\lsb@l}{\mathalpha}{letters}{`l}
\lowercase{\gdef~{\ifnum\the\mathgroup=\m@ne \ell \else \lsb@l \fi}}%
\endgroup

%%%%%%%%%%%%%%%%%%%%%%%%%
% Fix \vdots and \ddots %
%%%%%%%%%%%%%%%%%%%%%%%%%
\usepackage{letltxmacro}
\LetLtxMacro\orgvdots\vdots
\LetLtxMacro\orgddots\ddots

\makeatletter
\DeclareRobustCommand\vdots{%
	\mathpalette\@vdots{}%
}
\newcommand*{\@vdots}[2]{%
	% #1: math style
	% #2: unused
	\sbox0{$#1\cdotp\cdotp\cdotp\m@th$}%
	\sbox2{$#1.\m@th$}%
	\vbox{%
		\dimen@=\wd0 %
		\advance\dimen@ -3\ht2 %
		\kern.5\dimen@
		% remove side bearings
		\dimen@=\wd2 %
		\advance\dimen@ -\ht2 %
		\dimen2=\wd0 %
		\advance\dimen2 -\dimen@
		\vbox to \dimen2{%
			\offinterlineskip
			\copy2 \vfill\copy2 \vfill\copy2 %
		}%
	}%
}
\DeclareRobustCommand\ddots{%
	\mathinner{%
		\mathpalette\@ddots{}%
		\mkern\thinmuskip
	}%
}
\newcommand*{\@ddots}[2]{%
	% #1: math style
	% #2: unused
	\sbox0{$#1\cdotp\cdotp\cdotp\m@th$}%
	\sbox2{$#1.\m@th$}%
	\vbox{%
		\dimen@=\wd0 %
		\advance\dimen@ -3\ht2 %
		\kern.5\dimen@
		% remove side bearings
		\dimen@=\wd2 %
		\advance\dimen@ -\ht2 %
		\dimen2=\wd0 %
		\advance\dimen2 -\dimen@
		\vbox to \dimen2{%
			\offinterlineskip
			\hbox{$#1\mathpunct{.}\m@th$}%
			\vfill
			\hbox{$#1\mathpunct{\kern\wd2}\mathpunct{.}\m@th$}%
			\vfill
			\hbox{$#1\mathpunct{\kern\wd2}\mathpunct{\kern\wd2}\mathpunct{.}\m@th$}%
		}%
	}%
}
\makeatother

\newcommand{\B}{
	\begin{tikzpicture}
	\filldraw [fill=red, draw=black] (0, 0) rectangle (0.37, 0.45);
	\draw [line width=0.5mm, white ] (0.1,0.08) -- (0.1,0.38);
	\draw[line width=0.5mm, white ] (0.1, 0.35) .. controls (0.2, 0.35) and (0.4, 0.2625) .. (0.1, 0.225);
	\draw[line width=0.5mm, white ] (0.1, 0.225) .. controls (0.2, 0.225) and (0.4, 0.1625) .. (0.1, 0.1);
	\end{tikzpicture}
}

\author{Professor Andrew Snowden\\ \small\i{Transcribed by Thomas Cohn}}
\title{Math 493 Lecture 7}
\date{9/25/19} % Can also use \today

\begin{document}
\maketitle
\setlength\RaggedRightParindent{\parindent}
\RaggedRight

\subsection*{Direct Sums}

\par\noindent
Let $V$ be a vector space, $W_{1},\ldots,W_{r}\subseteq{}V$ subspaces.\n

\defn{
	$W_{1},\ldots,W_{r}$ are \textbf{independent} if $w_{1}=\ldots+w_{r}=0$, for $w_{i}\in{}W_{i}$, then $w_{i}=0$.\n
}

\defn{
	We let $W_{1}+\cdots+W_{r}=\set{w_{1}=\cdots+w_{r}\mid{}w_{i}\in{}W_{i}}$.\n
}

\par\noindent
Observations:
\begin{enumerate}
	\item $W_{1}+\cdots+W_{r}$ is a subspace.
	\item Suppose $v_{1},\ldots,v_{r}\in{}V$ are nonzero. Put $W_{i}=\vspan(v_{i})=\set{av_{i}\mid{}a\in{}K}$. Then $W_{1},\ldots,W_{r}$ are independent if and only if $v_{1},\ldots,v_{r}$ are linearly independent. Additionally, $W_{1}+\cdots+W_{r}=\vspan(v_{1},\ldots,v_{r})$.
	\item $r=2$: $W_{1}$ and $W_{2}$ are independent if and only if $W_{1}\cap{}W_{2}=\set{0}$.\n
	Reason: say $W_{1}$ and $W_{2}$ are independent, $v\in{}W_{1}\cap{}W_{2}$. Then $v+(-v)=0$, and we have $v\in{}W_{1}$, $-v\in{}W_{2}$. So $v=0$.
\end{enumerate}

\defn{
	$V$ is the (\textbf{internal}) \textbf{direct sum} of $W_{1},\ldots,W_{r}$, written $V=W_{1}\oplus\cdots\oplus{}W_{r}$ if $W_{1},\ldots,W_{r}$ are independent and $W_{1}+\cdots+W_{r}=V$.\n
}

\par\noindent
Observe: $V=W_{1}\oplus\cdots\oplus{}W_{r}$ if and only if every $v\in{}V$ can be written uniquely in the form $w_{1}=\cdots+w_{r}$, with $w_{i}\in{}W_{i}$.\n
Reason: suppose $v=w_{1}+w_{2}+\cdots+w_{r}=w_{1}'+w_{2}'+\cdots+w_{r}'$. Then $0=(w_{1}-w_{1}')+\cdots+(w_{r}-w_{r}')$ (each $w_{i}-w_{i}'\in{}W_{i}$). Because the $W_{i}$ are independent, we must have $w_{i}-w_{i}'=0$, so $w_{i}=w_{i}'$.\n

\ex{
	$K=\C$, $V=M_{n\times{}n}(\C)$. $W_{1}=\set{m\in{}V\mid{}\tpose{}m=m}$, $W_{2}=\set{m\in{}V\mid{}\tpose{}m=-m}$.\n
	Claim: $V=W_{1}\oplus{}W_{2}$.
	\begin{itemize}
		\item $V=W_{1}+W_{2}$: Given $m\in{}V$, $m=\paren{\frac{m+\tpose{}m}{2}}+\paren{\frac{m-\tpose{}m}{2}}$. $\frac{m+\tpose{}m}{2}\in{}W_{1}$ and $\frac{m-\tpose{}m}{2}\in{}W_{2}$.
		\item $W_{1}\cap{}W_{2}=\set{0}$. If $m\in{}W_{1}\cap{}W_{2}$, then $m=\tpose{}m=-\tpose{}m$. So $m=0$.
	\end{itemize}
}

\par\noindent
Let $V$ be a vector space, and $U\subseteq{}V$ a subspace.\n

\defn{
	A subspace $W$ of $V$ is called a \textbf{complement} to $U$ if $V=U\oplus{}W$.\n
}

\prop{
	Every subspace $U$ has at least one complement.\n
	Proof: Pick a basis $S$ (possibly infinite) of $U$. Extend $S$ to a basis $T$ of $V$. Define $W=\vspan(T\cut{}S)$. Claim that $V=U\oplus{}W$.\n
	Well, $V=U+W$. Let $v\in{}V$. Write $v=a_{1}x_{1}+\cdots+a_{n}v_{n}$, $a_{i}\in{}K$, $x_{i}\in{}T$. Assume $x_{i}\in{}S$ for $1\le{}i\le{}k$, $x_{i}\in{}T\cut{}S$ for $k+1\le{}i\le{}n$.\n
	Now, independence. Suppose $u+w=0$, $u\in{}U$, $w\in{}W$. Write
	\begin{align*}
		& u=a_{1}x_{1}+\cdots+a_{n}x_{n}\quad(a_{i}\in{}K,x_{i}\in{}S)\\
		& w=b_{1}y_{1}+\cdots+b_{m}y_{m}\quad(b_{i}\in{}K,y_{i}\in{}T\cut{}S)
	\end{align*}
	So
	\[
		u+w=a_{1}x_{1}=\cdots+a_{n}x_{n}+b_{1}y_{1}=\cdots+b_{m}y_{m}
	\]
	$T$ is a basis, and $v_{i}\in{}T$, $w_{i}\in{}T$, so $a_{i}=0$ and $b_{j}=0$, $\forall{}i,j$. So $u=0$ and $w=0$.\proven
}

\ex{
	$V=\C^{2},U=\vspan\paren{\smallBMatrix{1\\ 0}}=\set{\smallBMatrix{a\\ 0}\mid{}a\in\C}$.\n
	Claim: if $w=\smallBMatrix{b\\ 1}$, for any $b\in\C$, then $W=\vspan(w)$ is a complement of $U$.\n
	Reason: $\smallBMatrix{1\\ 0}$, $\smallBMatrix{b\\ 1}$ are a basis for $\C^{2}$.\n
	\n
	In fact, any line other than the $x$-axis is a complement to $U$.\n
}

\prop{
	$V$ is a vector space, $U,W\subseteq{}V$ subspaces. Let $\pi:V\to{}V/U$ be the quotient map. Then $W$ is complement to $U$ if $\pi|_{W}:W\to{}V/U$ is an isomorphism.\n
	Proof: $\ker(\pi|_{W})=\set{w\in{}W\mid\pi(w)=0}=\set{w\in{}W\mid{}w\in\ker(\pi)}=W\cap\ker(\pi)=W\cap{}U$.\n
	$\pi|_{W}$ is injective $\Leftrightarrow{}W\cap{}U=\set{0}\Leftrightarrow{}W,V$ independent.\n
	\n
	Suppose $\bar{v}\in\im(\pi|_{W})$, $\bar{v}=\bar{w}$ where $w\in{}W$. So $\overline{v-w}=0$, thus, $v-w\in{}U$. $v=w+u$, with $w\in{}W$ and $u\in{}U$. Conversely, if $v=w+u$, $w\in{}W$, $u\in{}U$, then $\bar{v}=\bar{w}$ because $\bar{u}=0$.\n
	\n
	$\im(\pi|_{W})=\set{\bar{v}\mid{}v\in{}U+W}$. So $\pi|_{W}$ is surjective if and only if $U+W=V$.\proven
}

\cor{
	Suppose $V$ is finite dimensional, and $U,W$ are complements. Then $\dim{}V=\dim{}U+\dim{}W$.\n
	Proof: $\dim{}V=\dim{}U+\dim{}V/U=\dim{}U+\dim{}W$, because $W\cong{}V/U$.\proven
}

\subsection*{External Direct Sums}

\par\noindent
Let $U$ and $W$ be vector spaces over $K$.

\defn{
	The (\textbf{external}) \textbf{direct sum} $U\oplus{}W$ is the set of all ordered pairs $(u,w)$ with $u\in{}U,w\in{}W$.\n
}

\par\noindent The external direct sum is a vector space:
\begin{itemize}
	\item $(u,w)+(u',w')=(u+u',w+w')$
	\item $a(u,v)=(au,av)$
\end{itemize}

\par\noindent
Let $\bar{u}=\set{(u,0)\mid{}u\in{}U}\subseteq{}U\oplus{}W$ and $\bar{w}=\set{(0,w)\mid{}w\in{}W}\subseteq{}U\oplus{}W$.\n
Then $U\oplus{}W$ is the internal direct sum of $\bar{u}$ and $\bar{w}$.\n

\subsection*{Linear Transformations}

\par\noindent
Let $T:V\to{}W$ be a linear transformation.\n

\defn{
	$\ker(T)=\set{v\in{}V\mid{}T(v)=0}$.\n
}

\defn{
	$\im(T)=\set{w\in{}W\mid{}\exists{}v\in{}V\ptxt{ \st{} }T(v)=w}$.\n
}

\par\noindent
Facts:
\begin{enumerate}
	\item $\ker(T)$ is a subspace of $V$.
	\item $\im(T)$ is a subspace of $W$.
	\item $T$ is injective if and only if $\ker(T)=\set{0}$.
	\item First isomorphism theorem holds: $T$ induces an isomorphism $V/\ker(T)\to\im(T)$.
\end{enumerate}

\defn{
	Suppose $V$ is a finite dimensional vector space. The \textbf{rank} of $T$ is $\dim(\im(T))$. The \textbf{nullity} of $T$ is $\dim(\ker(T))$.\n
}

\thm{
	(Rank-Nullity) $\rank(T)+\operatorname{nullity}(T)=\dim{}V$.\n
	Proof: $\dim{}V=\dim{}V/\ker(T)+\dim(\ker(T))=\dim(\im(T))+\dim(\ker(T))$, by the first isomorphism theorem.\proven
}

\ex{
	$V=P_{\le{}d}=\set{\ptxt{polynomials of degree }\le{}d}$, $K=\C$.\n
	$\map{T=V}{V}{f}{\frac{df}{dx}}$ is a linear transformation.\n
	Then $\dim{}P_{\le{}d}=\operatorname{nullity}(T)+\rank(T)=1+d=d+1$.\n
	Note: if we work over $\F_{p}$, then $\frac{d}{dx}(x^{p})=px^{p-1}=0$, so nullity can be greater than $1$.\n
}

\par\noindent
Let $A$ be an $n\times{}m$ matrix (i.e. $n$ rows, $m$ columns) over $K$. Define a linear transformation $T_{A}:K^{m}\to{}K^{n}$ by $T_{A}(v)=Av$.\n

\prop{
	Every linear transformation $T:K^{m}\to{}K^{n}$ has the form $T_{A}$ for a unique matrix $A$.\n
	Proof: write
	\[
		A=\begin{bmatrix}
			| & & |\\
			v_{1} & \cdots & v_{m}\\
			| & & |
		\end{bmatrix}
	\]
	with $v_{i}\in{}K^{n}$. Then $T_{A}(e_{i})=v_{i}$.\n
	If $T_{A}=T_{B}$, write
	\[
		B=\begin{bmatrix}
			| & & |\\
			w_{1} & \cdots & w_{m}\\
			| & & |
		\end{bmatrix}
	\]
	$T_{A}(e_{i})=T_{B}(e_{i})$, so $v_{i}=w_{i}$. Thus, $A=B$.\n
	\n
	Given an arbitrary $T$, put $v_{i}=T(e_{i})$ and
	\[
		A=\begin{bmatrix}
			| & & |\\
			v_{1} & \cdots & v_{m}\\
			| & & |
		\end{bmatrix}
	\]
	Then $T(e_{i})=v_{i}=T_{A}(e_{i})$, so $T=T_{A}$.\n
	Let $v=\sum{}a_{i}e_{i}$. Then $T(v)=T(\sum{}a_{i}e_{i})=\sum{}a_{i}T(e_{i})=\sum{}a_{i}T_{A}(e_{i})=T_{A}(\sum{}a_{i}e_{i})=T_{A}(v)$.\proven
}

\end{document}